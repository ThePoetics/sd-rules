\section{Disciplines}
\label{sec:disciplines}
Disciplines are the unique and awe-inspiring powers that have helped to ensure that 
kindred remain at the top of the world's social hierarchy from time immemorial.  
Each Clan has innate access to three specific Disciplines, which has helped to 
define their entire lineage into the archetypes we know today.

All kindred may learn their three disciplines without any outside aid; though there 
may be limits as to how quickly a character may advance, it rarely takes more than 
the expenditure of experience points earned from coming to game.  Learning a 
Discipline that is not native to your Clan requires a teacher who will sign your 
Experience spend sheet verifying that they have taught your character a Discipline.  
To learn a foreign Clan-specific discipline such as Obtenebration, Protean, or Serpentis, 
not only must you have a teacher but you must drink a trait of blood for each level 
learned, at the time of the purchase.  In either case someone can only teach below the 
level they currently possess, e.g. a Ventrue with level four Dominate can only teach 
levels one through three.

The first two levels of any discipline are considered ``basic'' powers, the second two 
``intermediate,'' and the final listed level is ``advanced.''  No player-character may 
ever learn Elder (level 6+) Disciplines through any means.

Disciplines that affect the mind of a target may often be used against a sleeping victim, 
at Storyteller discretion.  Rarely however will such powers be successful against a kindred 
who is in torpor or whose consciousness is located elsewhere (as through \emph{Possession}, 
for example), as there is no mind to affect.  Similarly unless specified no Mental or Social 
power will be effective against a deceased target.

Some powers require physical touch before activation; see the chapter on Challenges 
(page~\pageref{sec:challenges}) for more information on those specific mechanics.  For powers 
that physically alter a character such as certain levels of Protean, Serpentis, or Vicissitude, 
unless otherwise stated all changes can be reverted at will, taking the full listed amount of 
time required to change.  Transformations cannot normally be halted once started, and only one 
transformative power can be active at a time, except where dictated.

\subsection{Animalism}
This power represents control over the animal kingdom, and is retested by the 
\emph{Animal Ken} Ability.  Usually possessed by the Tzimisce, Gangrel, Nosferatu, 
and Ravnos Clans, it is only able to affect animals with vermin-level intelligence or 
better; this Discipline is wholly unusable on insects or similar lower-functioning 
creatures.  Statistics for and availability of particular animals are at Storyteller 
discretion.

\begin{description}
	\item[1 -- Feral Whispers:]  By looking into the eyes of an animal 
	you may speak verbally with it.  With a Static Social challenge versus 6-8 traits 
	you may issue simple commands which will be followed to the best of the 
	animal's usually limited ability for the next scene or hour.
	\item[2 -- Beckoning:]  By calling out to animals you may summon them to 
	your presence.  Spend a Social Trait for each animal desired and role-play 
	making the squawks, barks, howls, or other noises indicative of the chosen 
	species.  If present, the nearest applicable animals will hear the call and 
	respond. Alternatively you may endeavor to call to a specific animal, such as a favored pet.
	\item[3 -- Quell the Beast:]  Touching or meeting the gaze of a victim 
	allows you to temper their passions.  With a successful contested Social 
	Challenge you force your subject to be \emph{Submissive x2} for an 
	hour or scene, also negating their ability to spend temporary 
	Willpower.  If your target is in the midst of Frenzy or R\"{o}tschreck this 
	power instead instantly snaps them back to lucidity with no other effect, without 
	requiring any additional tests on the part of the frenzying kindred.  In 
	either case, using this power against another supernatural creature, 
	requires you to spend a Willpower before engaging in the challenge. 
	Multiple uses of this power are not cumulative.
	\item[4 -- Subsume the Spirit:]  By locking eyes with an animal 
	you may move your consciousness into it, causing your body to fall 
	limp.  This power requires no test but necessitates that you spend a Social 
	trait to enact it.  You are limited by the physical constraints of your new 
	form, but by spending additional Social traits you may transfer the ability to 
	use social and mental powers, as seen on the chart below.  
	
	For each Social trait spent on this possession you gain a semi-permanent Negative 
	trait of \emph{Feral} until you spend a Willpower to remove it on a one-for-one basis.  
	These traits and your expenditures to remove them should be logged with the Storytellers.
	
	You are not aware of anything that happens to your body, and any damage you suffer while 
	in animal form is sympathetically transferred to your torpid corpse.  To leave the body 
	you must declare so at the beginning of the combat turn.  If your host is still alive at 
	the end of that turn you return and wake normally.  If you take damage during the 
	turn you escape, throw a Simple Test:  a tie means you behave as that animal for 
	the rest of the scene after your return, while a loss results in you instantly frenzying.  
	If your animal is killed before you can escape, you fall Incapacitated with lethal damage.  
	
	If your defenseless, nearly-torpid body is staked your consciousness instantly returns to it, 
	but you are rendered helpless per normal.  If your body is torpored your mind returns and you 
	fall to slumber.  If however you are slain while your mind is possessing the animal, you cannot 
	leave it (without use of a power like \emph{Psychic Projection}) and your soul dissipates with 
	the coming dawn.
\end{description}

\begin{center}
\begin{tabular}{ | l l |}
	\hline
	\multicolumn{2}{| c |}{\textbf{Subsume the Spirit}} \\
	\hline
	1 Trait & Simple Possession \\
	2 Traits & Can use Auspex \\
	3 Traits & and Presence and Animalism \\
	4 Traits & and Dementation and Dominate \\
	\hline
\end{tabular}
\end{center}

\begin{description}
	\item[5 -- Drawing Out the Beast:]  In lieu of making a Self-Control/Instinct test to avoid 
	Frenzy you may issue a Social Challenge to a human or vampire within range.  If you succeed, 
	your beast enters them and they suffer your frenzy instead, leaving you under the effects of 
	\emph{Quell the Beast} and also unable to frenzy until your Beast returns at the frenzy's 
	conclusion.  If the target leaves the scene before the frenzy is exhausted however you will 
	continue to suffer these effects until you act in a monstrous enough fashion in the Beast's 
	presence to ``convince'' the Beast to come back.
	
	If the target in the grips of your Beast dies in your presence you immediately suffer the full 
	wrath of the Frenzy and cannot use Willpower to attempt to control it.
\end{description}

\subsection{Auspex}
Sharpening the senses to superhuman levels, the \emph{Investigation} Ability 
is used to retest this gift used by the Malkavians, Toreador, and Tremere.  When 
challenging someone using Obfuscate, the Auspex user, who is the defender, 
may add their level in Auspex (1-5) as bonus traits.  This bonus may also be applied at 
Storyteller discretion when using Auspex to aid in the search or investigation of an area.

\begin{description}
	\item[1 -- Heightened Senses:]  Improving some or all of your senses to 
	twice their normal effectiveness you may detect uses of Obfuscate and 
	observe goings-on from far distances.  Should your senses be overwhelmed 
	however by loud noises or lights, you will be be unable to use this power 
	for the remainder of the scene unless you spend a Willpower to do so.  In addition 
	you will be stunned for some moments, as determined by a Storyteller.  This power 
	may help a kindred find their way in nearly absolute darkness (see Chapter~\ref{sec:combat}).
	\item [2 -- Aura Perception:]  By staring at someone for a full three seconds 
	(one combat turn) you may ask one of the following questions and receive a 
	truthful answer by winning a Static Mental challenge against their current Mental 
	traits:  What is your current mood or emotional state?  What sort of creature are 
	you (human, ghoul, vampire, other)?  Are you under the effects of magic?  
	Have you committed diablerie (see page~\pageref{sec:additional})?  Was the last thing 
	you said a lie (applicable on mortals only)?  Are you currently under the effects of 
	an active Derangement?  This power has no function on dead mortals or torpid vampire 
	bodies, including those using  powers like \emph{Possession} or \emph{Psychic Projection}.
	
	In addition if you suspect a ghost or psychic entity is in the area you may spend a Mental 
	trait and then make this test, which allows you to see that an aura is present, but not 
	enough to identify an individual or ask any of the above questions.  
	\item[3 -- Spirit's Touch:]  By touching an object and concentrating for a full 
	turn you may sense traces of emotion left within an object.  Spend a Mental 
	trait to ask one of the following questions per application of this power:  Who 
	was the last person to significantly touch this item?  Was this object used in 
	any emotionally-charged events?  What strong emotions drove someone holding 
	this object?  If you are attempting to gain insight about someone who was 
	using Obfuscate, a separate static Mental challenge may be required to learn their 
	true identity, at Storyteller discretion.
	\item[4 -- Telepathy:]  By projecting your thoughts into a visible subject's 
	mind, requiring a Contested Mental challenge retested with \emph{Investigation}, 
	you may establish a connection after spending a full turn in concentration.  
	This power only functions while the two participating characters are in close 
	proximity, typically line-of-sight or line-of-effect.  If either character leaves 
	the immediate area contact is broken. The minds of supernatural creatures are more 
	difficult to break into and require the expenditure of a Mental trait before you 
	attempt the challenge.
	
	If successful you enter the subject's surface thoughts and may ask the following 
	questions, in addition to those allowed by \emph{Aura Perception}:  What 
	is the appearance of a person/place/item about which you are speaking?  What 
	have you omitted from the answer to a recent question, if any?  What is the true 
	answer to a question you lied about?  What memories do you have about (a specific 
	element) of current conversation? 
	
	While in their surface thoughts you may also project your own thoughts, 
	allowing for two-way communication.  While just listening to someone's surface thoughts 
	is a passive and unobtrusive activity, directly interacting with them through 
	communication will alert them to the foreign presence. Subjects who are aware of your mental 
	presence and are uncooperative require you to issue a Contested Mental challenge 
	(defended against with \emph{Subterfuge}) for each question.
	
	\emph{Telepathy} also allows you to probe deeper, delving into the hidden recesses 
	of your subject's mind.  This process is invasive and clearly announces your presence.  
	Asking any of these questions requires a separate Mental Challenge, against which the 
	defender may bid their full permanent Mental Trait total, using \emph{Subterfuge} to 
	retest.  If asking these questions of a supernatural creature, each question requires 
	the expenditure of a Mental Trait before the challenge:  What is one of your Flaws?  
	What is one of your Negative Traits?  What is one of your Derangements?  In all cases 
	the subject chooses the information to reveal if there are multiple possible answers, 
	and you must spend the Mental trait even if the target is willing.
	
	If you suspect a telepathic conversation is occurring in your vicinity you may 
	make a Static Mental challenge against the initiator of the Telepathy to listen in 
	to the ongoing conversation.  
	
	Telepathy is often role-played in more grand or invasive fashion by invoking the first 
	Golden Rule between players, but the core mechanics of the power are presented here and 
	will stand in the case of conflict; all other effects are by mutual player agreement only.
	\item[5 -- Psychic Projection:]  Leaving your body behind by spending a Willpower, your 
	spirit may move in any direction the speed of thought and through any physical barrier.  Your 
	Psychic form looks like the idealized version of yourself, and cannot be concealed or obscured 
	through disguise or powers such as Obfuscate.  To materialize in the physical plane costs an 
	additional Willpower, and such manifestation lasts for one turn.  While manifesting 
	you may speak normally and can be heard and may use Mental or Social 
	Disciplines by spending a Mental trait before making the attempt. 
	
	While projecting you are totally unaware of injury that may befall your body though you 
	can always find your way back by virtue of a tiny silver cord that links your 
	consciousness to your torpid form.  Other psychic entities may endeavor to follow 
	or even attack your cord however, as you may theirs in a battle of Willpower.  
	
	If your defenseless body is staked you immediately rush back to it, but are rendered helpless as 
	normal.  If you are torpored you likewise return to your now-inert body.  If however your body is 
	slain while you are active in the Psychic realms, you become instantly aware of it and are trapped 
	there, unable to affect the physical world save as described above.  In this sad case your soul 
	dissolves with the local dawn.
	
	Projecting characters are still limited to their normal senses of sight and smell, though 
	they cannot touch or interact with any physical object, even when manifesting.  Darkness 
	penalties (see Chapter~\ref{sec:combat}) apply as normal.  While traveling at great speeds 
	you are not able to observe your surroundings clearly or perhaps at all.  The hand sign 
	representing this power's use is a closed fist held vertically below the chin.
\end{description}

\subsection{Celerity}
Practiced by the Assamite, Brujah, and Toreador, Celerity allows a kindred to move 
faster than humanly possible, allowing for truly superhuman feats.  Activating Celerity 
costs one Blood trait per combat turn, regardless of the level(s) used.  Mental and Social 
actions may not be taken during Celerity follow-up rounds.  One cannot normally use Obfuscate 
while in Celerity, and all uses of Celerity other than \emph{Alacrity} are obvious to all 
observers.  All challenges taken during Celerity rounds use their standard retests.  Curiously 
Celerity does not cause thrown objects to travel faster or harder than normal.

\begin{description}
	\item[1 -- Alacrity:]  A single action may be taken at any time during the combat turn, 
	including pre-empting other challenges.  You may take one step, interact with 
	the environment, draw a single weapon, or step between two combatants.  You may neither 
	step out of a challenge nor initiate direct challenges with others.
	\item[2 -- Swiftness:]  This level of movement breaks the Masquerade as you 
	receive a single follow-up after the Everyman round, moving faster than is humanly 
	possible.
	\item[3 -- Rapidity:]  In tests of Dexterity while you have Celerity active you 
	may now throw the bomb which defeats rock and paper but loses to scissors.  You 
	must declare that you can throw the bomb before the challenge is thrown.  See 
	Chapter~\ref{sec:challenges} for more information.
	\item[4 -- Legerity:]  Moving as a blur you may take a second follow-up round 
	which comes after the Swiftness round.
	\item[5 -- Fleetness:]  In challenges of Dexterity with Celerity active, you now 
	win all ties.  You must declare the use of this power before a challenge is thrown.  
	If multiple contenders are using win all ties powers, compare traits as normal.
\end{description}

\subsection{Dementation}
A strange and mostly-unknown discipline only recently discovered in Sabbat Malkavians, 
Dementation is a power that inflicts confusion and psychological torture on its 
victims.  The appropriate retest is \emph{Empathy}.

\begin{description}
	\item[1 -- Passion:]  With a Social challenge you may dull your victim's senses, 
	granting them the Negative Trait \emph{Submissive} or ramp them up, granting them 
	\emph{Impatient}, either one for a full scene or hour.
	\item[2 -- The Haunting:]  Terrifying nightmares follow your target as you make their 
	reality truly terrifying.  Choose what sense is affected, spend a blood, and succeed 
	in a Social challenge to bestow the Derangement \emph{Schizophrenia} on your victim 
	(see page~\pageref{sec:derangements}) for the next scene or hour.
	\item[3 -- Eyes of Chaos:]  By studying the world or its inhabitants you may gain 
	insight as to its inner workings.  If you study an individual for a full turn and spend 
	a Mental trait you may learn their Nature with a successful Contested Mental challenge.  
	Failing this challenge wracks you with the Toreador Clan disadvantage until you are 
	disturbed as you study the person's intricate patterns. If you study the world itself 
	and spend a Mental trait you cannot be surprised in combat for the next scene or hour.
	\item[4 -- Voice of Madness:]  By speaking aloud to your victims in a reasonable 
	tone you can cause panic and rage.  Spend a Blood and then speak for a full turn.  
	With a mass Social challenge you may force all kindred in the area who are able to hear 
	you to test for R\"{o}tschreck using Self-Control/Instinct versus four traits.  You must also 
	test, though only against three traits.  Mortals are automatically feared and flee in blind 
	panic, likely not remembering their actions or the situation at hand.
	\item[5 -- Total Insanity:]  You must create an approved ``derangement deck'' to use this 
	power, which contains the approved \emph{Sanguine Dreams} Derangements that can be afflicted 
	on your target.  By capturing your target's attention for a full turn and spending a Blood you 
	may make a Social challenge.  Success cripples the target's mind as they suffer 
	five active Derangements at once for the next scene or hour, drawn randomly 
	from your deck.
\end{description}

\subsection{Dominate}
A powerful mind-controlling discipline wielded by the Giovanni, Lasombra, Tremere, and Ventrue, 
\emph{Intimidation} is used to retest these Mental challenges.  No-one can be Dominated to act 
in a self-destructive or suicidal manner, and your commands must be verbal and typically in a 
language the subject understands.  Dominate is completely ineffective against vampires of a 
lower generation; when using this Discipline against another kindred, if the challenge succeeds 
you must declare your generation to the target.  If your \emph{vitae} is less potent than theirs, 
the Dominate does not take effect, though your character does not know why it failed.  Dominate 
functions on a character's actions and cannot be used to affect their emotions.

\begin{description}
	\item[1 -- Command:]  You may give a single-word command of action to someone who meets your 
	gaze, which must be simple and easily understood.  The Command will be followed for 
	no longer than 10 minutes.
	\item[2 -- Mesmerism:]  You may now lay a more complicated Dominate within your subject, 
	even including a trigger or activation hook, e.g. ``when you next see the Prince, ask him 
	for a kerchief.''  This Dominate may still only contain one action to be performed.  Only 
	one Mesmerism may be implanted in a subject at a time, with new Dominators making a Static 
	Mental challenge against the original one's traits.  If successful the new Mesmerism supplants 
	the old, regardless of the Dominators' Generations. As with \emph{Command}, actions dictated 
	through \emph{Mesmerism} cannot last more than 10 minutes.
	\item[3 -- Forgetful Mind:]  By speaking with your subject for up to 15 minutes and with a 
	successful Mental challenge you may rewrite memories of up to the same length, filling them 
	with as much or as little detail as you would care to.  If you believe a particular 
	memory has been falsified with this power you may similarly challenge to bring forth 
	the original memories.  In either case you must specify the time and date, such as 
	``8:35pm on the 1st of July, this year.''  In the event a memory has been altered, a similar Simple 
	test must be thrown as above.  At Storyteller discretion the use of this power 
	in conjunction with \emph{Telepathy} may reveal additional altered memories. 
	\item[4 -- Conditioning:]  Bending the mind of your victim to your will, a captive can be 
	turned into a useful tool.  After an hour of intense interrogation make a Mental challenge.  
	The Conditioning only succeeds if this process continues, every night, until you have accrued 
	more successes than they have permanent Willpower.   Those so conditioned no longer receive a 
	test to resist your future Dominates and you may issue them without making eye contact; being 
	in their presence is sufficient.  You may layer multiple Mesmerisms in the subject, up to their 
	permanent Willpower.  A subject so conditioned receives a free retest against all uses of 
	Dominate by other individuals while they remain your thrall.  In addition, a Mesmerism from 
	another Dominator does not replace or override those already in place unless it directly 
	contradicts an implanted directive, though the above Static test is required in any case.
	
	While Conditioned mortals lose all sense of individuality and blindly follow instructions, 
	kindred thralls act more as Manchurian Candidates, waiting for the triggers to follow your 
	prescribed demands.  By wilfully avoiding your presence it is possible for a thrall to break 
	out of your influence after several weeks.  Needless to say, a victim can only be Conditioned by 
	one Dominator at a time.
	\item[5 -- Possession:]  By touching a mortal you may move your consciousness into their 
	body, your own falling lifeless to the floor.  Spend a Willpower and issue a Mental 
	challenge.  If successful you must spend one Mental trait to engage the Possession, and 
	may spend more to allow additional powers to transfer with you, per the chart below.  You 
	are not aware of anything that happens to your physical body, and any damage your host suffers 
	while you possess it is sympathetically transferred to your torpid corpse.
	
	To leave the body you must declare so at the beginning of the combat turn. If your host is 
	still alive at the end of that turn you return and wake normally. If the body is killed 
	before you can escape, you fall Incapacitated with lethal damage. If you take damage during 
	the turn you escape, throw a Simple Test; a loss results in you instantly frenzying as you return 
	to your kindred body.
	
	If your defenseless, nearly-torpid body is staked your consciousness instantly returns to it, 
	but you are rendered helpless per normal.  If your body is torpored your mind returns and you 
	fall to slumber.  If however you are slain while your mind is possessing the mortal, you cannot 
	leave it (without use of a power like \emph{Psychic Projection}) and your soul dissipates with 
	the coming dawn.
\end{description}

\begin{center}
\begin{tabular}{ | l l |}
	\hline
	\multicolumn{2}{| c |}{\textbf{Possession}} \\
	\hline
	1 Trait & Simple Possession \\
	2 Traits & Can use Auspex \\
	3 Traits & and Presence and Dominate \\
	4 Traits & and Dementation and Animalism \\
	\hline
\end{tabular}
\end{center}

\subsection{Fortitude}
The vampire's reputation for being indestructible is likely owed to uses of this power, 
which protects Gangrel, Ravnos, and Ventrue from being injured.  While there are no 
applicable retests for the use of Fortitude itself, all tests to absorb damage with Stamina 
traits are retested with the \emph{Survival} Ability.

\begin{description}
	\item[1 -- Endurance:]  Replace your Wounded and Incapacitated health levels with additional 
	Bruised levels; you do not suffer the normal effects for being severely injured until reaching 
	Torpor.
	\item[2 -- Mettle:]  Gain an additional Healthy health level.
	\item[3 -- Resilience:]  When you suffer aggravated damage you may make a Simple test to 
	reduce a single wound to lethal.  If you expend a Stamina trait you win on ties.
	\item[4 -- Resistance:]  When you suffer lethal or bashing damage you may make a Simple test 
	to reduce a single wound to nothing.  If you expend a Stamina trait you win on ties.  This 
	power stacks with Resilience, allowing you to test a single aggravated wound to nothing, if 
	successful in both challenges.  Note that to win ties on both Simple tests you must expend 
	two Stamina-related traits.
	\item[5 -- Aegis:]  By spending a permanent Willpower, kindred possessing this level of Fortitude 
	can shrug off even the most damaging blows, becoming truly immortal for the current combat turn.  
	Not only are they immune to further damage, they also erase all damage already taken this turn.  
	To avoid being staked this power must be used before the two Simple tests are thrown.  
	Burning for Aegis in this way removes a permanent Willpower trait from your sheet, first drawing 
	from unspent Willpower if available (e.g. someone starting with 4 permanent Willpower and 2 
	unspent would be reduced to 3 permanent and 1 unspent).
	
	In addition, characters with this level of Fortitude may bid the win-all-ties Stamina trait of 
	\emph{Aegis} in tests to resist damage.  This trait cannot be used offensively, does not prevent 
	touch-based attacks from succeeding, and does not count toward trait totals, but can never be 
	lost due to losing challenges and has no cost to use.  Additionally the trait Aegis can be 
	``expended'' to win ties in testing down damage with Fortitude levels three and four.  
	You may not utilize this use of \emph{Aegis} if you possess no Stamina-related traits.
\end{description}

\subsection{Necromancy}
See the \emph{Crimson Terrors} rules addendum for all aspects of Necromancy and wraiths.

\subsection{Obfuscate}
The power to conceal the wielder from the minds of those nearby, this power is used to great 
effect by the Assamite, Followers of Set, Malkavian, Nosferatu, and Ravnos Clans.  Obfuscate conceals 
the user, their clothes, and small held items and possessions.  The user of Obfuscate is the aggressor 
in all challenges relating to its use, but may add their level of Obfuscate (1-5) as bonus traits in 
such tests.  Since Obfuscate clouds the mind, individuals affected by it will take reasonable precautions 
to avoid bumping into the user, such as walking around them or holding a door open for them to follow on 
their way out.  Obfuscate will always be broken if the user interacts with their environment in a 
meaningful way, including moving furniture, walking through a conversation, splashing in a puddle, bumping 
noisily against something, using a Presence power, moving in Celerity, or issuing a Physical Challenge.  
\emph{Stealth} is the appropriate retest for Obfuscate.

\begin{description}
	\item[1 -- Cloak of Shadows:]  While remaining motionless in a pool of shadow or nearly total concealment 
	behind an object, none will observe your presence.  This concealment is broken if you move, speak, 
	or your hiding place is illuminated.  The hand sign to represent use of Level 1 or 2 Obfuscate is a 
	hand held open in front of the mouth.
	\item[2 -- Unseen Presence:]  You may now walk slowly (1 step per full combat turn) while using 
	Obfuscate, though staying to the shadows is highly encouraged from a role-play perspective.  You may 
	only enter Obfuscate in areas meeting the requirements for \emph{Cloak of Shadows}.
	\item[3 -- Mask of a Thousand Faces:]  While using this power your facial features become bland 
	and unremarkable, making you a generic ``everyman,'' of your gender and of average height and weight.  
	So dramatic is this blurring of features that appearance-based Negative Social traits (such as 
	\emph{Bestial} and \emph{Repugnant}) are suppressed while you are hidden by this power, at Storyteller 
	discretion.
	
	By spending a Mental trait you may assume a specific set of features instead of a generic face, though in 
	both cases neither your clothing nor carried items altered by the illusion.  Through this trait expenditure 
	you can alter your apparent height and weight slightly as well, or even your gender.  Regardless of 
	appearance you continue to rely on your own Social traits.  To convincingly appear as someone who possesses 
	more Social traits than you, you must spend Blood to make up the difference, giving your illusion more 
	validity. This application 	does not actually grant you any additional traits, and you may only spend 
	Blood to emulate traits up to your normal generational cap.  Talk to a Storyteller when attempting to 
	masquerade as another character.

	Mask of a Thousand Faces is also the only Obfuscate  power that can be used while moving in Celerity, 
	and which permits the use of Presence while active.
	\item[4 -- Vanish from the Mind's Eye:]  You may now initiate your use of Obfuscate even when individuals 
	are observing you, leaving them to not worry about where you went.  This power acts as a mob challenge, 
	you versus all onlookers, who may activate Auspex in response to this challenge.  If performed in combat 
	this power activates at the end of the combat turn (see Chapter~\ref{sec:combat}).  
	
	You may now whisper while in Obfuscate and not automatically drop into visibility 
	to all; however those hearing you may test as if you had interacted with your environment.
	\item[5 -- Cloak the Gathering:]  By spending one Mental trait per subject (up to 5) you may use your 
	lower powers of Obfuscate on willing individuals so long as they remain within 10' of you.  Only one 
	power can be imparted at a time; you could not grant both \emph{Mask of a Thousand Faces} and 
	\emph{Unseen Presence} at once.  Nor may an individual be under multiple instances of the same 
	level of Obfuscate at the same time.  Anyone attempting to pierce this effect must test against you; 
	success means they see through the entire illusion.  If a single individual breaks their Obfuscate by 
	interacting with his environment or other action, only they are revealed.  This power may only be used 
	on targets who are both willing and conscious.
\end{description}

\subsection{Obtenebration}
A Discipline steeped in mystery, only the stoic Lasombra have claimed mastery over its dark secrets.  
All challenges for Obtenebration are retested with \emph{Occult}.  A practitioner may see through 
their own Obtenebration shadows without penalty but all others suffer the effects of blindness 
(see Combat, page~\pageref{sec:combat}).  All shadows created with Obtenebration dissipate immediately 
if subjected to true sunlight.

\begin{description}
	\item[1 -- Shadow Play:]  By controlling natural and supernatural shadows, you may help yourself or 
	hinder foes, even in areas of great brightness.  By spending a Blood you may choose one of several 
	effects:  grant yourself +1 Trait in challenges of Stealth, gain +1 Trait in challenges of 
	Intimidation, or inflict a foe with the Negative trait \emph{Clumsy} with a Mental versus Physical 
	challenge, retested with \emph{Occult}.
	\item[2 -- Shroud of Night:]  Spend a Blood to create a globe of inky darkness roughly 10' in diameter 
	anywhere within 50' of you.  Anyone entering the Shroud gains the Negative trait \emph{Clumsy} though 
	you must best a target in a Mental versus Physical challenge to summon it on their location, retested 
	with \emph{Occult}.  The Negative trait Shroud issues is not cumulative with that from Shadow Play.  
	You may move the shroud at a walking pace (1 step per combat turn) so long as you maintain full 
	concentration.
	\item[3 -- Arms of the Abyss:]  You may summon inky tentacles from a nearby shadow by spending a Blood 
	and spending one Social trait per arm, to a maximum of one-third your permanent Social traits, rounded 
	down.  These arms are treated as NPC assistants for the purpose of Mob Combat.  If acting independently 
	they possess three traits and no Abilities or Disciplines.
	\item[4 -- Black Metamorphosis:]  Spending two Blood traits and a Social trait allows you to sprout four 
	Arms of the Abyss from your own body, which grants you the traits \emph{Intimidating x3} and the ability to 
	see through all mundane darkness.  In addition to helping you in Mob Combat, at the conclusion of each 
	combat turn you may make a Static test against an opponent in melee range; success means you may either 
	deal 1 Lethal or grant the victim \emph{Clumsy} as a dark tentacle numbs a limb.  The same target may only 
	receive one negative trait this way.
	\item[5 -- Tenebrous Form:]  Collapse into a puddle of infinite darkness roughly conforming to a humanoid 
	shape by spending three Blood traits and spending three full turns in concentration.  While in this form 
	you may see through all natural darkness without penalty and cannot be harmed by physical attacks, though 
	magic, fire, and sunlight function normally.  You are unaffected by gravity, allowing you to climb up any 
	vertical surface, and may slip through any small crack or hole.  By making a Mental versus Physical 
	challenge you may wrap yourself around someone's face, mirroring the effects of \emph{Shroud of Night}.
	
	While in this form you take an additional level of damage from fire, sunlight, and magical effects, and your 
	capacity to use Disciplines is greatly hindered by your lack of eyes, vocal chords, and usable Blood pool.
\end{description}

\subsection{Potence}
Showing truly inhuman strength Brujah, Giovanni, Lasombra, and Nosferatu kindred can perform truly 
amazing feats, though any use of Intermediate or Advanced Potence is an obvious breach of the Masquerade.  
Potence augments other challenges and as such does not have retests of its own.  Curiously Potence does 
not allow thrown weapons or objects to travel faster or farther than normal.

\begin{description}
	\item[1 -- Prowess:]  Once per evening you may refresh your Strength-based traits at no cost.  In addition 
	your brawl and melee attacks may cause Lethal damage if you choose.
	\item[2 -- Might:]  In tests of Strength you may now use the Might retest, which settles the matter 
	definitively---there are no further retests after Might has been called, though its use can be canceled 
	by an opponent also using this power.
	\item[3 -- Vigor:]  In challenges of Strength you may now throw the bomb, which beats paper and rock but 
	loses to scissors.  You must state before the test is thrown that you are able to throw the bomb. See 
	Chapter~\ref{sec:challenges} for more information.
	\item[4 -- Intensity:]  When entering a test of Strength you may bid the singular trait \emph{Intense}, 
	which can never be lost but does not count toward tie resolution.  If you possess no other Strength traits 
	you may not use Intensity.
	\item[5 -- Puissance:]  In a Strength-based contest you may win all ties and also deal an extra level 
	of Lethal damage, but the use of this power must be declared prior to the challenge being thrown.  If used 
	in a challenge this power prevents you from pulling your punch in any way.  Items not used as designed or 
	possessing the Negative Trait of \emph{Fragile} may very well break with this level of Potence, at 
	Storyteller discretion.
\end{description}

\subsection{Presence}
The seductive lord of the manor with his fancy galas and impeccable smile, wooing the hearts of visitors from 
across the land---this is the result of well-practiced applications of Presence, a force of personality few 
can resist.  Wielded by the Brujah, Followers of Set, Toreador, and Ventrue, it is devastating in the right 
hands.  All Presence tests are retested by the \emph{Leadership} Ability.

\begin{description}
	\item[1 -- Awe:]  By spending a Social trait you draw everyone's attention to you for a moment, catching the 
	eye of everyone in the room.  To forcibly meet someone's gaze in order to use another power on them, such as 
	Dominate, you must best them in a Social challenge.  Awe may be used as a retest in all Social challenges by 
	similarly spending a Social Trait.
	\item[2 -- Dread Gaze:]  By baring your fangs and hissing you may drive away an unwanted guest with a Social 
	challenge.  If successful the subject flees in terror from your presence and will stay away for the scene or 
	hour.  If they are forced to confront you or escape is impossible they must bid an extra trait in all 
	challenges against you due to fear.  This power is an obvious breach of the Masquerade and is an offensive 
	Discipline.
	\item[3 -- Entrancement:]  By besting your subject in a Social challenge they will be favorably disposed 
	toward you and will not insult or attack you for the rest of the scene or hour, barring hostile actions on 
	your part toward them.  They are encouraged to role-play as if you are an ally or friend, or at the least 
	neutral, if they started off as hostile.
	\item[4 -- Summon:]  Your mastery of Presence is such that you can call individuals to you, even across great 
	distances.  With a Social challenge you can force someone to approach, though targets out of the 
	immediate scene require a Willpower expenditure to reach.  Your Status applies to this challenge.  Ask a 
	Storyteller to perform the challenge on your behalf; you are not aware of its failure or success unless the 
	individual arrives.  If successful the subject will make reasonable haste to arrive at your location to be in 
	your presence, though they will not know the supernatural nature of the compulsion to see you.  They will not 
	travel through or to any knowingly dangerous situation, or go to those who they believe wish them harm, and all 
	Summons are broken at daybreak.  If the summoner leaves the immediate area the Summon is likewise broken.
	
	You can only Summon individuals you know, at minimum those with whom you have had several minutes of 
	conversation.  If a person is Summoned by multiple individuals, the most potent Generation wins.  In the case of
	ties the first Summon overrides others until completed or broken.
	
	In the event you have interacted with someone using \emph{Mask of a Thousand Faces} to impersonate the individual 
	you wish to Summon, whomever is closest to you, whether the actual target or the illusion, will suffer its 
	effects.  When in doubt see a Storyteller.
	\item[5 -- Majesty:]  By spending a Willpower you exert your force of personality over everyone in the area 
	for a full scene or hour.  As long as your Majesty is active nobody may insult or act aggressively toward 
	you.  A subject may contest your Majesty with a Social challenge, in which you are the aggressor, but must 
	spend a Willpower to do so.  Failure means the same subject may not re-contest your same Majesty within the 
	same scene.  If you take a hostile or offensive action the aura fades instantly and many onlookers may be 
	outraged with the sudden shift of emotions.  If you enact Majesty near other kindred they receive an 
	automatic test to avoid its effects without the need to spend Willpower.
\end{description}

\subsection{Protean}
Command over the wilds and the rampaging Beast within themselves are the hallmarks of Clan Gangrel, who alone 
boast this transformative power.  All transformations take place at the resolution of the relevant combat turn.  
There are no challenges associated with Protean and as such no specific retests.

\begin{description}
	\item[1 -- Eyes of the Beast:]  A rather glaring breach of the Masquerade, this power causes a kindred's 
	eyes to glow bright red, granting them the ability to see in normal darkness (see Chapter~\ref{sec:combat}) 
	but suffering the Negative Trait \emph{Bestial}.  There is no cost to activate this power.
	\item[2 -- Earth Meld:]  While touching raw soil, spend a Blood trait to meld slowly into the ground, 
	taking your full turn and concentration.  While so submerged your spirit is diffused and almost entirely
	hidden from spirits or curious investigators.  If the soil is disturbed (by someone digging with a shovel, 
	for instance) you explode out of the dirt, unable to act in the first turn.
	\item[3 -- Feral Claws:]  By spending a Blood these claws emerge from your fingers at the end of the turn.  
	Dealing an Aggravated wound if used to attack, they also grant a +1 bonus for tie comparison.  Attacks 
	with claws use the \emph{Brawl} Ability for retests.  You may not hold weapons with this transformative power 
	active.
	\item[4 -- Shape of the Beast:]  By spending a Blood and waiting three full turns, though the process can be 
	expedited one turn per extra Blood spent, your person and small personal possessions are transformed either 
	into a wolf or bat.  While in wolf form you gain the traits \emph{Quick} and \emph{Brutal} and the effects 
	of Feral Claws.  While in bat form you take no penalties for movement and can fly but may only bid three 
	traits when acting offensively.  Other Gangrel animal forms may be allowed by the Storytellers but will 
	share the traits presented here.  \emph{Eyes of the Beast} and \emph{Mist Form} may be used while so transformed.
	\item[5 -- Mist Form:]  By spending a Blood and waiting three full turns, expedited one turn per extra 
	Blood spent, your body turns into an insubstantial mist, able to move at normal pace (1-3 steps) in any direction.  
	You become immune to mundane attacks and take one less level of damage from fire and sunlight.  You are still 
	affected normally by magical attacks and have no blood with which to fuel other powers.  If you fall to any frenzy 
	you leave Mist Form automatically.
\end{description}

\subsection{Quietus}
Practiced solely by the dangerous Assamite assassins, Quietus is a mystical Discipline that leaves their victims 
helpless.  Most powers do not have retests of their own, relying on \emph{Brawl}, \emph{Melee}, and perhaps 
\emph{Athletics} where appropriate.  Any poisons created through vitae become inert at sunrise.

\begin{description}
	\item[1 -- Silence of Death:]  By spending a Blood you create a stationary field of absolute silence in a 
	10' radius around yourself wherein which no sound can be created or escape, though muffled sounds can enter, 
	lasting for a scene or hour.
	\item[2 -- Scorpion's Touch:]  Each Blood trait you fuel into this power is converted into a potent contact 
	poison that causes victims to lose a Physical trait above any lost from a challenge.  You may coat a 
	weapon with a number of doses up to its bonus traits, and/or hold one dose in your mouth to spit at an 
	opponent.  The poison is touch-based and attacks need not deal damage to be successful in delivering its 
	effects.  Each attack inflicts one dose of poison, and each dose of poison takes a full turn to create.
	\item[3 -- Dagon's Call:]  After successfully touching your target you may at any point in the scene declare 
	that you are using this power.  Inform the Storytellers how many Willpower you are spending; for each 
	Willpower spent your target must make a Static Physical challenge against your traits or suffer a lethal 
	wound.
	\item[4 -- Baal's Caress:]  As Scorpion's Touch, save this poison causes an aggravated wound to victims and 
	objects and must enter the target through a successful attack.  No weapon may have more than one magical 
	effect on it at one time.
	\item[5 -- Taste of Death:]  By spending a Blood you can make a single ranged attack against a single target 
	within 10', spewing caustic acid which deals two aggravated wounds.  Retest with \emph{Athletics}.
\end{description}

\subsection{Serpentis}
Only the Egypt-based Followers of Set possess this almost magical Discipline, bringing them closer to their 
sleeping god.  Specific retests are noted below.

\begin{description}
	\item[1 -- Eyes of the Serpent:]  Make a Social challenge against your target, who must have met your 
	gaze.  If successful your target is completely immobilized, barring injury or attack, so long as you hold 
	their attention.  Retest with \emph{Subterfuge}.
	\item[2 -- Tongue of the Asp:]  Turning your tongue into a sensitive weapon you may use it to navigate dark 
	places (see Combat on page~\pageref{sec:combat}) and strike for Aggravated damage from several feet away 
	(retest with \emph{Athletics}).  After a successful strike you may drink using the tongue on successive turns, 
	activating the Kiss.  This is a transformative power.
	\item[3 -- Skin of the Adder:]  Spend a Blood and a Willpower to transform into a large man-snake hybrid at 
	the end of the turn.  Gain the bonus traits \emph{Lithe} and \emph{Wiry}.  You need not grapple first before 
	making a bite attack.  Your bite deals an additional Lethal wound if you win a Simple test after a 
	successful biting attack.  Gain the Negative traits  \emph{Bestial} and \emph{Repugnant} while in this form, 
	though you may activate \emph{Tongue of the Asp}.
	\item[4 -- Form of the Cobra:]  By spending a single blood and concentrating for three full turns you 
	transform to a literal giant cobra with a bite that is lethally poisonous to mortals.  You may fit 
	through small spaces and suffer no penalties related to movement.  Gain a +2 bonus on challenges to 
	initiate, maintain, or escape from a grapple.  You may activate \emph{Tongue of the Asp} while in this form.
	\item[5 -- Heart of Darkness:]  On the evening of the new moon you may remove the heart of a willing 
	subject and place it in a clay urn, rendering them immune to staking and gaining +1 on tests to avoid 
	frenzy.  The subject also loses access to the \emph{Empathy} Ability and also the Social traits \emph{Friendly}, 
	\emph{Empathetic}, or \emph{Genial}.  If the heart is staked the body is also, and if the heart is subject 
	to even one level of damage from the sun or fire, the body explodes into ash.  This ritual takes two hours 
	to perform.
\end{description}

\subsection{Thaumaturgy}
See the \emph{Crimson Terrors} supplement for rules governing Thaumaturgy.

\subsection{Vicissitude}
A terrifying power and science practiced solely by the \emph{Tzimisce}, Vicissitude allows its wielder 
to shape flesh and bone with surgical precision.  Retests are performed with \emph{Crafts: Body Crafts} 
and some tests may require levels in \emph{Medicine} to succeed.

\begin{description}
	\item[1 -- Malleable Visage:]  Able to mold your own face, altering your appearance and voice, this 
	power costs one Blood to use and takes several minutes.  Copying another's appearance requires making 
	a Static Social challenge against their Social traits.  Success allows you to copy their face but not 
	use their Social traits in challenges.  Alternatively you may give yourself \emph{Repugnant} up to 
	three times.
	\item[2 -- Fleshcraft:]  Perform drastic alterations to the soft tissue of your victim, as per 
	\emph{Malleable Visage}.  In addition you may move and shape tissue, permanently removing one of your 
	subject's Physical traits in exchange for an additional Healthy health level, or the reverse, though 
	multiple applications of this effect do not stack.  This power cannot be used in combat and takes time to 
	exercise.  Each alteration costs one Blood trait.
	\item[3 -- Bonecraft:]  Now able to bend and shape bone to your will, you may now alter a subject's 
	height and body structure or create terrifying natural weapons which deal Lethal damage.  Each 
	modification costs a single Blood trait, but if used without \emph{Fleshcraft} your subject suffers 
	one Lethal wound per application.  This power is not usable in combat though the weapons created 
	by it are.
	\item[4 -- Horrid Form:]  Transforming yourself into a terrifying monster with black, rubbery skin 
	and wholly unnatural protrusions and appendages, any who have not seen this power used before must 
	make a Courage check at four traits or suffer R\"{o}tschreck.  Requiring two Blood to activate, 
	you are granted \emph{Bestial}, \emph{Feral}, and \emph{Repugnant} as Negative Social traits but 
	also the Physical traits of \emph{Brawny}, \emph{Ferocious}, \emph{Dexterous}, \emph{Quick}, 
	\emph{Enduring}, and \emph{Stalwart}.  Your unarmed attacks deal Lethal wounds and cause an
	additional level of damage.  This power lasts until sunrise or until dismissed.
	\item[5 -- Bloodform:]  Collapsing individual limbs or your whole body into an animated pool of vitae, 
	this transformative power has no cost and takes effect at the end of the combat turn.  Each limb so 
	transformed becomes one Blood trait with your body making up the rest.  If partially transformed you 
	may return lost limbs either by being in contact with the pool or by spending a like number of Blood 
	traits to regrow the wound, making the separated blood pool inert.  If fully transformed you may not 
	use powers requiring voice, but are immune to all damage save from fire or sunlight.  If all of the 
	Blood is imbibed or destroyed you meet Final Death.
\end{description}

\subsection{Combination Powers}
In the many centuries since each Clan's founding some few talented individuals have mastered their 
innate Disciplines to such a degree that they are able to use two powers in tandem, creating a new 
\emph{Combo Power} that often has effects over and beyond their component Disciplines.  These 
inventions often took decades if not longer to develop, and many are reluctant to give up their 
secrets easily or cheaply.

Knowledge that these powers exist may require \emph{Lore} in a specific Clan, and learning them 
takes both dedicated effort and a great deal of role-play.  No character may ever learn another Clan's 
combo powers except where referenced below.

\subsubsection{Assamite Combo Powers}
\begin{description}
	\item[Forced March]--- Requires Celerity 2, Fortitude 2 --- Costs 6 XP \\
	Often needing to travel long distances through hostile terrain, some Assamites have harnessed the 
	ability to recoup the Blood expended on \emph{Celerity} for such excursions, sharply decreasing 
	their time in the open.  In any downtime scene where your character is moving on foot a 
	considerable distance you may divide the normal travel time by your levels in \emph{Celerity}.  
	Each hour or part thereof you only need spend half your levels of \emph{Celerity} in Blood points 
	to maintain this pace.  You may not use this power more times per night than you have levels in 
	\emph{Fortitude}.
	\item[Shadow Feint]--- Requires Celerity 2, Obfuscate 2 --- Costs 6 XP \\
	Keeping one's enemies off-balance is a time-honored tradition among assassins and the Assamites 
	have incorporated this tactic into their potent arsenal.  When activating \emph{Celerity} for the 
	turn, spend an additional Blood trait and use your Everyman action only for Dodging.  Gain a +2 
	bonus on all ties for physical challenges during the \emph{Swiftness} and \emph{Legerity} follow-up 
	rounds.
\end{description}

\subsubsection{Brujah Combo Powers}
\begin{description}
	\item[Burning Wrath]--- Requires Celerity 3, Potence 3 --- Costs 9 XP \\
	Able to focus their rage into a tangible and terrifying display, Brujah who have mastered this 
	ability can strike their foes with unrelenting force.  By spending a Blood their skin glows 
	red and sometimes even emanate a fine crimson mist but the true benefit of this power becomes 
	evident when they attack:  all successful brawling attacks this turn deal pure aggravated damage.
	\item[Iron Heart]--- Requires Potence 3, Presence 3 --- Costs 9 XP \\
	Able to steel their minds against outside influence, a Brujah who possesses this power may spend a 
	Willpower in order to win all ties when defending against \emph{Dominate}, \emph{Presence}, or 
	mind-altering \emph{Thaumaturgy} in the current Scene.  By spending a second Willpower he may 
	confer this power on another for the next scene or hour, though this individual only receives an 
	additional retest against such powers.
	\item[Pulse of Undeath]--- Requires Auspex 1, Potence 3 --- Costs 3 XP \\
	By focusing their \emph{Heightened Senses} through their supernatural strength, a Brujah having 
	learned this Discipline may determine whether or not a nearby character possesses any of the 
	three physical disciplines---Potence, Celerity, and Fortitude---by besting them in a Static 
	Mental challenge, retested with \emph{Investigation}.  By spending Mental traits he may determine 
	their respective levels, one trait per Discipline.
\end{description}

\subsubsection{Follower of Set Combo Powers}
\begin{description}
	\item[True Love's Face]--- Requires Obfuscate 3, Presence 3 --- Costs 9 XP \\
	Through careful application of \emph{Mask of a Thousand Faces} and a successful \emph{Entrancement} 
	the Setite appears to be his target's closest ally, one with whom they share an emotional bond.  
	The Setite does not instinctively know who they appear to be and must deduce this information 
	from the behavior of his subject.
	\item[Typhonic Beast]--- Requires Potence 3, Serpentis 4 --- Costs 11 XP \\
	A more potent version of \emph{Skin of the Adder} which may represent a truly unflappable 
	devotion to their god, a Settite wielding this power is a terrifying sight to behold.  In 
	addition to the effects of that Discipline, both positive and negative, they gain the 
	traits \emph{Observant x2}, \emph{Wiry}, \emph{Lithe}, and \emph{Enduring} as they become 
	a truly mythic hybrid of man and serpent.  This power costs three Blood and a Willpower to 
	activate.
\end{description}

\subsubsection{Gangrel Combo Powers}
\begin{description}
	\item[Claw Immunity]--- Requires Animalism 2, Fortitude 4 --- Costs 4 XP \\
	At home in the wilds, some Gangrel have developed supernatural defenses against natural 
	predators that may take hostile actions on their travels.  A Gangrel possessing this 
	power spends two Blood and specifies a type of natural animal such as bear or wolf.  When 
	attacked by an animal of that kind, all intermediate \emph{Fortitude} powers are twice as 
	effective, meaning two levels of damage may be tested down per attack instead of just one.  
	This power lasts until sunrise or until reactivation with a different specified animal.
	\item[Flesh Wound]--- Requires Fortitude 2, Obfuscate 3 --- Costs 9 XP \\
	Masters of resilience, Gangrel have a well-earned reputation for near-invincibility.  With 
	the advent of this discipline those rumors have grown exponentially.  Activated by spending 
	a Blood, all damage taken in the next scene or hour fails to leave a visible wound or mark, 
	possibly causing attackers to give up their assault against their ``invincible'' foe.  In 
	reality this power relies on \emph{Obfuscate}, and so may be seen through with \emph{Auspex} per 
	normal.  After appearing to shrug off an attack which deals at least Lethal damage you gain a +2 
	bonus on Social challenges relating to intimidation for the rest of the scene against anyone who 
	is affected by your illusion.
	\item[See the Reflected Form]--- Requires Auspex 4, Protean 4 --- Costs 9 XP \\
	Comfortable with the art of transforming their own bodies, Gangrel who have learned this 
	power have sharpened their senses to be able to detect this ability in others.  Utilizing 
	a specialized \emph{Aura Perception} challenge the Gangrel may see the most powerful 
	transformative power their target possesses (normally the highest-level Discipline).  By spending 
	a Mental trait they may see all shapes the subject regularly uses, and by spending a second Mental 
	trait all forms the individual is capable of may be determined.  This power does not reveal any 
	information related to \emph{Mask of a Thousand Faces}; it applies to true transformative powers only.
\end{description}

\subsubsection{Lasombra Combo Powers}
\begin{description}
	\item[Shroud of Absence]--- Requires Dominate 3, Obtenebration 3 --- Costs 9 XP \\
	Mimicking the effects of \emph{Obfuscate}, the Lasombra who has mastered this power is able 
	to create an area into which no outsider looks unless their attention is specifically called 
	there.  As this is a \emph{Dominate} effect, it only works on equal or higher-generation 
	kindred.  To enact this power the Lasombra spends a Blood and makes a Static Social challenge 
	against seven traits.  Anyone wishing to peer into the area affected, roughly a 10' diameter 
	sphere, must win a Social challenge to do so, wherein the Lasombra is the aggressor.  Retests 
	are made with \emph{Intimidation}.
\end{description}

\subsubsection{Ravnos Combo Powers}
\begin{description}
	\item[Sympathetic Agony]--- Requires Animalism 4, Fortitude 4 --- Costs 11 XP \\
	Used to being on the receiving end of physical punishment, some Ravnos have managed to 
	turn the tide against their assailants, forcing them to feel the sting and hurt they 
	would deliver unto others.  When a Ravnos with this power takes damage they may spend a 
	Blood and spend Social traits up to the number of wounds taken.  Their attacker will, 
	for the rest of the scene, suffer wound penalties as if they have received the same 
	number of wounds, which cannot be healed through Blood.  These effects last for one 
	scene or hour and cannot torpor other kindred.
	\item[Mask of Cathay]--- Requires Animalism 3, Obfuscate 3 --- Costs 7 XP \\
	Spending so long traveling among tight-knit mortal communities, enterprising Ravnos have 
	discovered how to hide their true nature from hunters and those who would wish to find 
	the devils in their midst.  By spending a Willpower and 1 Blood their aura shines as brightly 
	as a mortal's, they gain the Merit \emph{Eat Food}, and they suffer no social penalties for 
	having low Humanity for the rest of the evening.
\end{description}

\subsubsection{Toreador Combo Powers}
\begin{description}
	\item[Bliss]--- Requires Dominate 2, Presence 3 --- Costs 7 XP \\
	Able to turn their Clan disadvantage as a source of strength, enterprising Toreador have 
	learned to keep their Beast at bay with the power of beauty.  Immediately after having 
	been affected by the Toreador Clan disadvantage, you may make a Static Mental challenge 
	against twice the value of your Conscience, Self-Control, or Courage Virtue at Storyteller 
	discretion, with no applicable Ability retest.  If successful you gain an extra temporary 
	trait of that Virtue for the next scene or hour.  Only Toreador on the Morality path of 
	Humanity may use this power.
	\item[Doubletalk]--- Requires Auspex 2, Celerity 1, Obfuscate 1 --- Costs 5 XP \\
	Also available to select Malkavians and Tremere, \emph{Doubletalk} is an ability that 
	has made its way through Toreador courts for centuries if not millennia, and is one of the more 
	``easily''-obtained Combo Powers.  When activated,  requiring a Blood and Static Mental challenge 
	(retest with \emph{Performance}) against six traits, this power's user is able to speak whole 
	sentences between other words for the rest of the scene, sounding to all observers as ``uh,'' 
	``er,'' or ``hmm.''  In this manner two individuals possessing knowledge of this Discipline may 
	have two conversations simultaneously; very popular when discussing private or sensitive topics in 
	public.  Any character possessing this Discipline may make a Static Mental challenge against seven 
	traits to determine that it is being used in their presence, though the Blood expenditure and 
	additional test are required to understand or partake in the message.
	\item[Soul Painting]--- Requires Auspex 4, Presence 2 --- Costs 9 XP \\
	A truly rare gift even among Combo Powers, \emph{Soul Painting} allows a Toreador 
	to capture a subject's very essence in a portrait, able to be perceived by sensitive 
	observers.  When a painting featuring a particular kindred subject is completed, the Toreador 
	makes a Static Social challenge against nine traits.  If successful, the art captures the 
	subject's Nature.  By then spending up to three Social traits they can encourage the following 
	qualities to be present in the work, one per trait:  rough gauge of Humanity (with unflattering 
	results if the subject is on a different Morality path), Willpower, Conscience/Conviction, 
	Self-Control/Instinct, or Courage.  Other kindred with his power immediately see the extra 
	included elements.  Particularly observant or sensitive characters, at Storyteller discretion, 
	may make a Static Mental challenge against twice the number of qualities present to detect 
	the same.
\end{description}

\subsubsection{Tremere Combo Powers}
\begin{description}
	\item[Thaumaturgical Sight]--- Requires Auspex 2, Thaumaturgy 1 --- Costs 3 XP \\
	A specialized use of \emph{Aura Perception}, you become keenly aware of all uses of Blood 
	magic in your line of sight for as long as you maintain your Auspex.  To determine the specifics 
	of an effect you witness, succeed in a Static Mental challenge with a Storyteller (retested with 
	\emph{Occult}) against a difficulty they set, usually based on the level and rarity of the power 
	in question.  You automatically know the identity and function of any powers you yourself possess, 
	but in all cases may only identify a single power per full turn, which requires your full concentration 
	(as per \emph{Aura Perception}).  This power costs one Blood trait per activation and can be maintained 
	for up to one minute.
\end{description}

\subsubsection{Ventrue Combo Powers}
\begin{description}
	\item[Denial of Aphrodite's Favor]--- Requires Dominate 3, Fortitude 3 --- Costs 10 XP \\
	Masters of all they survey, Ventrue who have learned this powerful Discipline are able to 
	protect themselves from \emph{Presence} in much the same way \emph{Dominate} functions.  
	This power costs nothing to activate once learned and allows its user to automatically 
	ignore all uses of \emph{Awe}, \emph{Dread Gaze}, and \emph{Entrancement} used by those 
	of higher generation.
	\item[Lifesong]--- Requires Dominate 1, Presence 1 --- Costs 4 XP \\
	Learning how the mind of one's opponent works is a powerful step to success.  With the 
	development of this power many Ventrue have expedited the process of knowing their 
	enemy.  During a conversation with a Mortal he may make a Contested Mental challenge, 
	retested with \emph{Empathy} and resisted with \emph{Subterfuge}.  If successful he 
	learns the target's Nature.
\end{description}