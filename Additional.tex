\section{Additional Systems}
\label{sec:additional}
Not every aspect of game fits into nice categories.  This chapter focuses 
on those unique or specific rules and mechanics that wouldn't be appropriate 
under previous headings.  In this section other aspects of both the game itself 
and kindred life are explored.

\subsection{Fair Play and Metagaming}
\emph{Sanguine Dreams} is a collaborative, cooperative game wherein which dozens of people come together 
to try and build compelling and amazing stories within the setting known as the World of Darkness.  Not 
every player has the same understanding of the setting, the rules, or the social aspect of live-action 
role-playing, and it is important that we all work together to make sure that we promote a welcoming and 
open environment for all players, whether new to the game or well-versed in its nuances.

Playing fair means making sure that your actions, both in- and out-of-character, fall within the rules as 
presented in this book, as well as within the bounds of good conduct.  Deliberately misinterpreting or 
misquoting rules, taking advantage of another player's lack of familiarity with the setting or system, 
and other examples of poor sportsmanship are severely looked down upon, and anyone who repeatedly abuses 
the collaborative trust of the game will face serious repercussions.  Even though we may portray horrible 
monsters of the night in-character, it is our duty to help each other out-of-character to understand the rules, 
the setting, and how to contribute to a great story---the best way you can help is to lead by example.

``Metagaming'' is a special term that means using out-of-character information to influence your in-character 
decisions or actions.  If a previous character of yours learned that the Toreador Primogen had violated the 
dictates of the Prince, it would be metagaming for your new character to reveal or act on that information 
unless they had verifiable means through which they discovered the same.  Metagaming could also be creating a 
character to avenge your previous character's death, or specifically to bolster another character's endeavors.  
In short, metagaming is cheating, like stealing money from the bank in Monopoly or hiding cards up your sleeve 
in poker, allowing you to benefit from aspects of the game that you have no right to.

It is human nature to talk about things that are exciting, and by and large each of us are excited to play at 
\emph{Sanguine Dreams} and explore how our stories twist and turn each week, but almost invariably the largest 
source of (usually unintentional) metagaming is caused by someone talking about their character or scenes they 
were involved in, and other parties taking that information in-character, perhaps believing that it was relayed 
to their character, or not realizing that it was a purely out-of-character discussion.  The easiest way to prevent 
this type of metagaming is simply to not talk about your character, your character sheet, or current goings-on 
with other players.  That doesn't mean stop talking about Vampire altogether, as there is plenty of interesting 
story to be found in the World of Darkness, but simply that it is a good idea not to discuss secrets, plots you are 
involved in, or upcoming events that are currently relevant with others, unless you do so specifically in-character.

\subsubsection{``For the Good of the Game''}
Vampire is not a game where everyone gets what they want; all too often nobody at all gets to see their plans 
realized, and the eternal struggle against crippling loss is an ongoing theme in \emph{Sanguine Dreams}.  
Some times characters are presented with opportunities to advance their own endeavors or get ahead, but to do so 
they must take actions that could be detrimental the game as a whole, whether it means killing other characters 
or running them out of the Domain.

While every player is entitled to portray their character in a realistic way that is true to their concept, 
it is important to remember that there are more than thirty other players who come together to build a 
collaborative story, and at times ``what the character would do'' is not beneficial to the game.  The 
Storytellers try to discourage the introduction of characters who are merciless killers or hold detrimental grudges 
against other kindred, but the largest onus is on each individual player to recognize that by making another character 
unplayable, whether through death, incapacitation, or exile, they are in all likelihood forever cutting short 
that character's story in a way that is unsatisfying and perhaps hurtful to the other player.

Before you take any action that could remove another character from play, please think carefully about the 
in- and out-of-character effects that such an act would have and work with the other players in the 
scene to come up with a collaborative solution that could provide alternatives to the end of another character's 
story.  Failing that, please at least talk to the Storytellers beforehand.

In short, we encourage all players to portray their characters to the fullest of their ability, within the bounds 
of the setting and the system, but also that everyone keeps an eye toward the collaborative nature of the game and the 
cooperative community we continue to build every week at \emph{Sanguine Dreams}.

\subsection{Time Definitions}
\begin{description}
	\item[Chronicle:]  the scope of the entire game plot-line from its inception until its 
	conclusion.  The current chronicle began in April 2015.
	\item[Story:]  the period of one calendar month, from the first day to the last.  This 
	nomenclature is used to reduce confusion inherent in the use of the word ``month.''
	\item[Scene:]  a continual period of time and place.  A scene could encompass a 
	private conversation in a secluded room or a long, leisurely stroll around the 
	block.  A new scene begins at the cessation of combat turns.
	\item[Turn:]  one full set of combat actions lasting between two and six seconds, 
	having within it several rounds.
	\item[Round:]  one phase of the combat turn (e.g. the Everyman and Swiftness 
	rounds).
	\item[Action:]  any action taken in the combat turn.  The Celerity power of \emph{Alacrity} 
	grants one action outside of normal combat rounds within a given turn.
	\item[``Scene or Hour:''] Many powers describe that their effects last for one scene or hour.  
	This means the effect ends with whichever comes first---either the conclusion of the current 
	scene or one hour within the same scene.  When in doubt if a power is still in effect, speak 
	to a Storyteller for clarification.
\end{description}
	
\subsection{Blood Bonds}
The power of vampiric \emph{vitae} is truly terrifying to behold.  Not only does it 
fuel kindred Disciplines but the merest taste of it is impossibly addicting.  Once 
a mortal imbibes even a single taste, they will want more, going to extreme lengths 
to sate their craving.  Even as powerful as they are, vampires themselves are not 
immune to the bonds of Blood, for if they drink of another kindred they too suffer 
the consequences:  almost no nectar in the world is as erotic, sensual, or delicious, 
far surpassing even the most vibrant taste of mortal Blood brimming with emotion.

Some times enforced as a punishment, at others as a way to pledge allegiance, the bond 
ensures loyalty and obedience from a thrall to her regnant.  After the first drink the 
thrall finds that his thoughts often dwell on their regnant, perhaps even going so far 
as to visit places they hope to run into her.  There are no mechanical effects at this 
stage but the power of the bond should be role-played.

If a second drink is taken on a subsequent night before the bond fades away the thrall 
is considered two steps bound, regarding their regnant as a central figure in his life.  
The thrall may act as he pleases but must win a Simple test to take actions directly 
harmful to his regnant, and suffers a one-trait penalty on all Social challenges with her.

If a third drink is taken on yet another night before the second fades the Blood bond is 
compete, three full steps of obedience and service.  Nothing matches the thrall's 
devotion to his regnant, not family, Clan, allies, or friends---all pales in 
comparison to the one who has so bound him.  The regnant need not even make eye contact 
to \emph{Dominate} her pet, the mere sound of her voice is enough.  In addition to 
suffering a two-trait penalty on all Social and Mental challenges against his regnant, 
the thrall cannot act in any way against her wishes without spending a Willpower---one 
trait lasts a scene for indirect actions or for only a single turn if he tries to 
attack physically.  While a character may be one or two steps bound to multiple characters, 
the devastating power of the full bond prevents him from being thrice bound to anyone else 
at the same time, and the wishes of their three-step regnant override those to whom he may 
bound to a lesser degree.

\subsubsection{Resisting the Bond}
Escaping a blood bond is no easy task, even for the strongest of wills.  A one- or two-step 
bond fades one level in twelve months minus the thrall's permanent Willpower, to a minimum of 
one month, and at Storyteller discretion.  The time may be reduced if the character actively 
resists seeing their regnant, often requiring the expenditure of Willpower as they attempt 
to wean themselves off the ``high'' he feels in her presence.  A character who increases their 
permanent Willpower total, through XP expenditure or other means, similarly decreases the length 
of time the bond holds sway.

A three-step bond however is an entirely different level of enforced dedication.  This bond does 
not fade unless the thrall can completely avoid the presence of their regnant for a period of 
twelve months minus the thrall's permanent Willpower.  Even a single evening spent in their 
regnant's presence may reset this timer, at Storyteller discretion.  Attempting to resist this 
level of connection may require frequent Willpower spends to avoid the compulsion to attend to 
the regnant's needs.

If a vampire is three steps bound to a kindred who dies, they may suddenly feel that 
``something important'' happened, but not know the exact cause of the sensation.  Though 
they continue to be bound three points to the deceased, and likely spending many of their 
nights pining or searching for them, the connection will wane in the same time-frame as a two-point 
bond.

In all cases, particularly abusive or hostile regnants may find their thralls slipping away faster 
than expected while more compassionate ones may see their servants staying loyal for far longer than 
normal.  It is often rumored that a sure-fire way to break a bond is to deal the killing blow to one's 
regnant.  Like all whispered knowledge in kindred society however nothing is certain, particularly 
murderous advice seeping in from the shadows.

\subsection{Ghouls and Retainers}
\label{subsec:ghouls}
The \emph{Retainers} Background provides characters with mortal assistants who are more 
competent and well-rounded than those granted by \emph{Allies} or Influences.  These 
non-player characters are under the full control of the Storytellers but are crafted with 
your desires in mind; if you are looking for a high-powered banker you won't receive a 
street thug, for example.

For every dot in the \emph{Retainers} Background you may pick up an additional base-build 
assistant or add five XP to an existing retainer.  This way they may grow and become more 
useful to you as your needs develop and change.  All characters are limited to a total of 
fifteen points in the Retainers Background, whether spread among many assistants or few, though 
no single Retainer can have more than five dots devoted to it.  A specific retainer may be used 
to increase the amount of Influences a character can control, with more information appearing in 
the \emph{Mortal Manipulations} supplement.

Ghouls are a special kind of retainer and possess many abilities normal mortals do not.  
Regularly invested with your Blood, they possess not only a greater capacity to serve but usually 
an unshakable loyalty enforced by the blood bond.  These creatures, neither fully alive nor dead, 
gain powerful bonuses while the \emph{vitae} stays in their system: \\

\begin{description}
	\item[Disciplines:]  all Ghouls eventually develop one or more Disciplines possessed by their 
	kindred regnant, though almost never more than two or three powers at maximum.  Particularly 
	long-serving Ghouls may learn a first Intermediate power, but these are the exception and not 
	the rule.  All Disciplines learned are at Storyteller discretion and control.  No ghoul may 
	teach Disciplines to others.
	\item[Immortality:]  while kindred blood courses in their veins the aging process stops, 
	rendering them immune to death by natural causes.  In this way some servants have 
	faithfully attended their masters for decades if not longer.
	\item[Healing:]  able to shrug off wounds much in the same way kindred do, ghouls 
	are able to use their stored \emph{vitae} to heal as their vampire masters can.  No Ghoul 
	can heal Aggravated wounds however.
	\item[Increased Traits:]  fueled by supernatural power, a ghoul's trait caps increase by 
	one each to 9, 7, 5 for Primary, Secondary, and Tertiary categories (in the case of a regular 
	human).  They do not automatically gain additional traits, merely the capacity to gain more 
	than the normal, mortal maximum. \\
\end{description}

\noindent The first point of Blood invested to a mortal infuses them with the immortality of a 
ghoul for one month.  All mortals are able to hold the equivalent of two additional traits of 
\emph{vitae} in their system which can be used to heal or invoke Celerity.  If a ghoul does not 
receive more Blood before the end of the month, one of these extra traits is absorbed to prolong 
their immortality for another month.  If a ghoul ever runs out of \emph{vitae} they immediately 
lose access to all associated benefits and powers; their body no longer supernaturally sustained.  
They age one full year per hour until they reach their true age, possibly even dying if they have 
been in service long enough.  A starved ghoul is a terrifying sight to behold as they attack any 
kindred they can identify trying to prolong their own life.

Animal ghouls are treated the same as humans, though they usually possess greatly inferior 
trait caps and Abilities than their human counterparts.  As animals are regularly afraid or 
hostile toward vampires, gaining one's trust well enough to have it feed from a kindred may be 
exceptionally difficult without the \emph{Animal Ken} Ability or the \emph{Animalism} Discipline.

While their unending loyalty makes them the perfect servants, the jealousies inherent in 
such a relationship make maintaining multiple ghouls difficult at best as each tries to outdo 
the other for their regnant's favor, sometimes even going so far as to kill their competition.

For each ghoul, which must be represented by at least one dot of \emph{Retainer}, their master 
is down one Blood trait during the first game of each month, representing the regular upkeep 
required to maintain them.

Often ghouls will begin to take on the appearance or mannerisms attributed to their regnant's 
Clan; Nosferatu ghouls begin to look quite unattractive, Malkavian ghouls become deranged, and 
so forth.  The specific effects of ghoul creation and maintenance are up to the Storytellers.

Typically ghouls are not considered ``supernatural'' creatures for the purpose of Disciplines 
effects or other powers.

\subsection{Downtime Actions}
The characters of \emph{Sanguine Dreams} exist more than only at the Sunday night gather, 
and often many endeavors require mid-week scenes or conversations, such as travel to foreign 
domains or investigations into the activities of others.  These opportunities for role-play 
are called downtime scenes and may take place between players or with the involvement of 
Storytellers.  Downtimes can be run in-person, over the phone, even via our online message 
board.

While we highly encourage players to run scenes with each other outside of game, if your 
scene is likely to impact the world at large or involve numerous or important NPCs it is 
imperative that a Storyteller be involved to ensure a cohesive continuity for the game 
as a whole.

In addition it is important to keep the Storytellers appraised of your character's night 
to night activities as they may be interrupted or preempted by the actions of others.  For 
regular or recurring actions submitting a brief overview is sufficient, giving the 
Storytellers an idea of your usual actions.  Please remember that no downtime scene is set 
in stone until you receive the all-clear from a Storyteller, and keep that fact in mind as 
you interact with others during the week.

\subsection{Feeding and Living Night-to-Night}
It can be safely presumed that any character which is viable for play has passed their 
Accounting (where applicable) or has otherwise figured out how to eke out a manageable 
nightly existence as a vampire.  Barring specific notification to the contrary, Storytellers 
will assume your character is able to keep reasonably fed, as described in ``Starting Blood 
Pool'' in Chapter~\ref{sec:other}, stay out of trouble with mortal or undead authorities, 
and manage their average affairs without incident.

Even with all of the supernatural gifts bestowed on vampires there is however a limit to the 
amount they can accomplish in a given time frame.  Between regularly feeding, managing estates, 
staying in contact with \emph{Allies} and \emph{Retainers}, and checking in on their various 
projects, a kindred may quickly find himself without the resources to take on more.  It is 
important to keep the Storytellers abreast of the projects your character is currently undertaking, 
so they can warn you when the schedule is starting to fill up, and also to give you regular 
progress updates.  Unless a project and the time spent to further it is logged with the Storytellers 
it should not be presumed to happen.

Certain Merits or Flaws notwithstanding, which provide their own opportunities and drawbacks, it 
is presumed that vampire characters are able to feed without drawing undue attention to their 
activities.  This includes carefully selecting prey to make sure they only receive untainted 
blood, feeding in private or in the comforts of their own home, and making sure they aren't taking 
enough to leave lasting damage.  Undue haste, lack of preparation, or an inability to plan ahead 
may all make feeding more complicated.

Traditionally a vampire feeds from a human by piercing their flesh with their fangs, which elongate 
as they strike.  This type of assault normally induces a rush of euphoria in the victim, often called 
``the Kiss,'' and which causes the target to stop resisting, caught in an orgasmic rush of endorphins 
so long as the vampire continues drinking.  Sedated in such a way, the prey will often forget specifics 
of the encounter and the fact that they were attacked at all.  Vampires are able to feel some semblance 
of the emotions running through their prey as well, and so also receive a small measure of the rush, making 
feeding an exciting experience.  Kindred targets, having passed through the unfathomable feelings of death 
and rebirth, may choose to give in to the Kiss or not at their discretion.  

For each full turn of drinking a vampire drains up to three Blood traits, certain Merits or Flaws 
notwithstanding.  If the attacker engages in other actions during Celerity follow-up rounds, the amount 
taken is reduced by a like amount.  Humans and animals being fed from suffer a phantom box of Bashing damage 
for every trait drained, which fades in a few days, the effects of which are at the Storytellers' discretion.  
A human who has been drained four or more blood traits will require swift medical attention or face death.

Feeding from a diseased or sick host may impart penalties to the attacking vampire at Storyteller 
discretion, which only fade after the tainted blood is spent, following a ``last in, last out'' method.

\subsection{Actions During the Day}
Cursed to forever shun the light of day, vampires are not well-suited to participating in 
mid-day activities.  When they are roused, kindred are sluggish and lethargic, unable to 
wield their full undead might against those who would trespass against them.  A vampire 
may not bid more than three-halves their Morality traits (see page~\pageref{sec:morality}), 
rounded down, in Named traits in any challenge while the sun is up.  Bonus traits such as 
those from weapons may be applied normally, but the closer one is to the Beast, the less 
able they are to will their bodies into action.

If a character wishes to wake up during the day, a Static Mental challenge is required, which 
includes the above trait limits.  Certain Merits and Flaws notwithstanding the difficulty 
to rise with provocation, such as being attacked, is four traits.  Without provocation the 
difficulty increases to six traits.  Once risen, the vampire is able to act for the immediate 
scene, but must test to remain awake any longer than necessary, at Storyteller discretion.

Kindred who wish to avoid falling into slumber while the sun is out, either near daybreak or 
after having woken as detailed above, must overcome one of the most potent aspects of their 
curse.  They must succeed in a Static Willpower test against six traits to remain awake for 
one scene or hour, and are subject to the daytime penalties based on their Morality as above.

\subsection{Havens and Holdings}
A vampire's home needs to cover two basic needs:  protection from the sun and ready access to 
food.  By and large kindred live in the cities, near the hustle and bustle of mortal life.  
While this provides them advantages in finding victims from which to feed, they must take 
certain steps to make sure their haven is protected both from the elements and from other 
intruders.  While many assumptions can be made about a character's favorite haunts, elements 
of their character sheet and write-ups submitted to the Storytellers can make a large difference 
in their possessions.

Every regular character in \emph{Sanguine Dreams} is assumed to have made arrangements or 
precautions for their continued good health, presumably a small haven within one of Sonoma 
County's cities.  Unless otherwise specified to the Storytellers, all havens are expected to be 
at least somewhat effective at keeping out local riff-raff, the elements, and provide a small 
modicum of convenience.  It is very recommended that players write up their havens, particularly 
if they want to include particular or more effective security measures or place them in a specific 
location within the world.  Without the Storytellers' permission no special or unique features of 
a haven should be assumed to exist.  Giving the Storytellers this information assures that there is 
no confusion as to where your character stays or the circumstances around their slumber, once they 
have approved it.

Also important are other holdings---many kindred like to host gathers, invite allies or 
enemies over for discussions, and expand their personal command of the mortal world by their 
very presence.  These locations, and the justification for their possession or purchase, should 
be recorded with the Storytellers for like reasons as above.

In all cases the comforts, size, location, and amenities of a character's haven or holdings 
depend greatly on their \emph{Resources} Background, which dictates, among other things, on what 
kind of location they can afford the upkeep.  With higher levels of \emph{Resources} your character 
may be justified in having multiple properties or even an estate, suitable to truly grand receptions.  
A character looking to purchase or acquire new property should both notify the Storytellers and run 
downtime scenes wherein they attempt to acquire or build the location to suit.  In all cases the 
judicial use of \emph{Influences} are likely to greatly expedite the process.

\subsection{OOC Hand Signs}
Many vampires possess powers that are difficult for we as humans to emulate, and so various hand signs 
have been developed which show other players that certain rules or effects are in play.  Here are the most 
common hand-signs used in \emph{Sanguine Dreams}, which should all be respected:

\begin{description}
	\item[Out of Character:] Sometimes players need to step out of the roles of their characters, whether 
	to ask a question of a Storyteller, use a drinking fountain, or to go grab their character sheet.  
	Prominently holding up crossed fingers means that they aren't currently interactable and should be 
	ignored by players in the middle of role-play.  When out of character, please be respectful of those 
	still gaming and try not to interrupt scenes wherever possible.
	\item[Obfuscate:] One hand held over the face represents the use of the first two levels of 
	\emph{Obfuscate}, most commonly to signify that the character using it should not be interacted with or 
	noticed in any way.  They are functionally invisible to most kindred, out of sight and out of mind.
	\item[Heightened Senses:] The first level of \emph{Auspex} allows a character to perceive things 
	from farther away than mortal senses could. By holding one or two fingers up to the eyes, ears, or nose, 
	a player is showing that they have this power active, and that their character may not be physically 
	located where they are standing.
	\item[Language:] To represent that a character is speaking a language other than English, players 
	hold up one hand in the sign language letter ``L'' with the thumb on their chin.
	\item[Psychic Projection:] A closed fist held beneath the chin demonstrates the use of this power, 
	which makes the character invisible, inaudible, and almost impossible to interact with.
	\item[Telepathy:] Occasionally two characters will speak mentally, bypassing any verbal communication. 
	This is represented by a closed fist with the thumb and pinky fingers outstretched held to the temple, 
	similar to a ``hang ten'' or sign language ``Y''.
\end{description}
