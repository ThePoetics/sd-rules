\section{Backgrounds}
\label{sec:backgrounds}
Unlike Traits, which represent who your character is, and Abilities 
which showcase what your character can do, Backgrounds are a metric of 
what your character possesses, both materially and socially.  Normally 
Backgrounds cannot be improved with experience without running scenes 
with the appropriate Storyteller to justify their advancement.

All Backgrounds available for Camarilla kindred are listed below: \\

\begin{description}
	\item[Age:]  Each dot represents the benefit of around 100 years of unlife and grants 
	three free Ability dots chosen from the following list:  \emph{Academics}, 
	\emph{Crafts}, \emph{Etiquette}, \emph{Expression}, \emph{Finance}, 
	\emph{Hunting}, \emph{Law}, \emph{Linguistics}, \emph{Lore}, 
	\emph{Medicine}, \emph{Politics}, \emph{Repair}, \emph{Science} (classics only), 
	\emph{Scrounge}, \emph{Security}, and \emph{Streetwise}.  Remember that being old  
	does not automatically grant you age-related Status (see page~\pageref{sec:status}), 
	and that characters wishing to enter play with more than one dot of this Background 
	will need specific Storyteller approval.
	\item[Allies:]  Representing mortal friends or agents you can call on for favors, 
	\emph{Allies} are halfway between \emph{Contacts} and \emph{Retainers}, in that each 
	dot represents a group with a particular skillset such as locksmiths, bartenders, 
	or nightclub owners.  \emph{Allies} are useful for getting low-level actions done during 
	downtime, though calling on them too often may lead them to ask for favors from you.  
	All \emph{Allies} are under Storyteller control, and their specific archetype or 
	skillset should be recorded when purchased.  Each such Ally can be called on once per 
	story, and no character can possess more than 10 total dots of \emph{Allies}, each dot 
	either making an existing group more useful or adding an additional archetype you can contact.
	\item[Alternate Identity:]  Each dot makes an assumed kindred identity more convincing 
	or may instead be used for a new identity, to a maximum of five total dots.  Make sure to 
	work with a Storyteller and include details of your identity in your background, as these 
	identities can take years to establish. Alternate Mortal identities are normally acquired 
	through the use of \emph{Influences}.
	\item[Contacts:]  See \emph{Influences}.
	\item[Fame:]  Represents popularity in the mortal world in a specific area or societal group.  
	This Background may be required to extend the reach of \emph{Influences} beyond the local city. 
	See the \emph{Mortal Manipulations} packet for more information about using this Background.
	\label{bg:generation}
	\item[Generation:]  The potency of your blood is determined by your Generation, defining how 
	much Blood you may store and use per turn, starting and maximum Willpower, and how 
	susceptible you are to specific powers such as \emph{Dominate}.  So important is this Background 
	that each dot costs two Background points or two Free Traits.  See the following table for the 
	limits your generation imposes on your character sheet.  Without points in this Background 
	characters begin at 13th generation.
\end{description}

\begin{center}
\begin{tabular}{ | l c c c | }
	\hline
	\textbf{Generation} & \textbf{Trait Caps} & \textbf{Blood} & \textbf{Willpower} \\
	\hline
	13th & 10/8/6 & 10/1 & 2/6 \\
	12th & 10/8/6 & 11/1 & 2/8 \\
	11th & 11/9/7 & 12/1 & 4/8 \\
	10th & 12/10/8 & 13/1 & 4/10 \\
	9th & 13/11/9 & 14/2 & 6/10 \\
	\hline	% The below line needs to be edited based on final text size.  6.67cm if smaller, 8cm if larger
	\multicolumn{4}{| p{8cm} |}{Blood represents your maximum blood pool and maximum 
	per-turn expenditure.  Willpower represents your starting and maximum permanent 
	Willpower traits.} \\
	\hline
\end{tabular}
\end{center}

\begin{description}
	\item[Herd:]  You are in frequent contact with those who willingly let you feed from them, 
	even if they don't realize your true nature.  This Background refreshes once per week and may 
	be used to gain easy access to vitae, one point per dot.
	\item[Influences:]  Influences are described fully in the separate 	\emph{Mortal Manipulations} 
	packet and represent subtle control over mortal circles such as \emph{Media} or \emph{High Society}.  
	See the Storytellers for more information on the proper use of this Background.
	\item[Malkavian Time:] \emph{Malkavians only.} Representing the unconscious and unpredictable bursts 
	of inspiration some Malkavians exhibit, this Background allows a character to, either during or 
	immediately after a scene, ask a Storyteller for information relating to a the scene in question or 
	a particular ongoing plot, which will be given a difficulty of one to five traits.  The character's 
	dots in this Background act as their traits for the Static test which determines whether or not they 
	receive the information they seek.  No character may benefit from this Background more than once per 
	story.
	\item[Mentor:]  A more powerful kindred will grant you aid when asked.  
	Speak with a Storyteller and make a simple test.  Success means they help 
	you without asking for a favor, a tie means they ask for payment after assisting you, 
	and failure forces you to help them before they help.  A Mentor may only 
	be called upon once per story and their abilities increase with your dots 
	in this Background, from answering esoteric questions through teaching you 
	Disciplines, though they will never teach Clan-specific or Advanced powers, and 
	will only ever teach as many levels as dots you possess in this Background.  No character may 
	have multiple \emph{Mentors}, and their particulars should be worked out with the Storytellers.
	\item[Nosferatu Information Network:] \emph{Nosferatu only.} Representing the connections many 
	Nosferatu have with their kin in other cities, this Background allows a character to, at the start 
	of a session, ask a Storyteller for information relating to a particular character or remote Domain, 
	which will be given a difficulty of one to five traits.  The character's dots in this Background act 
	as their traits for the Static test which determines whether or not they receive the information they 
	seek. No character may benefit from this Background more than once per story.
	\item[Resources:]  Representing both the scale and luxury of your material holdings, \emph{Resources} 
	details how much ``spending money'' you have available each story and the general comfort level of your 
	haven.  How many locations you possess, their size, and general opulence, are all  limited by your dots 
	in this Background; work with a Storyteller to detail your holdings.  When in doubt use modern prices 
	for goods and services, and be aware that money gained through Resources cannot be hoarded without 
	Storyteller approval.
\end{description}

\begin{center}
\begin{tabular}{ | l l r | }
	\hline
	\multicolumn{3}{| c |}{\textbf{Resources}} \\
	\hline
	0 Traits & Poverty. & \$200 \\
	1 Trait & Small savings and a small apartment. & \$500 \\
	2 Traits & Modest savings and a home. & \$1,000 \\
	3 Traits & Significant savings and a large home. & \$3,000 \\
	4 Traits & Large savings and an estate. & \$10,000 \\
	5 Traits & Rich savings and estates. & \$30,000 \\
	\hline
\end{tabular}
\end{center}

\begin{description}
	\item[Retainer:]  A notable employee or ghoul who will do your bidding.  Each 
	dot represents 5xp that goes into its sheet or an additional helper.  All 
	retainers are created and controlled by Storytellers.  Each ghoul requires 
	a blood investiture of one trait before the first game of each month.  There is a 
	limit of fifteen dots of this Background per character, however dispersed.  For 
	more information see \emph{Ghouls and Retainers} on page~\pageref{subsec:ghouls}.
\end{description}

