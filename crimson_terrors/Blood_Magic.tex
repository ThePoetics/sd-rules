\section{General Rules of Blood Magic}
\label{sec:rules}
The powers of Thaumaturgy and Necromancy are wide and varied, with effects that have the capacity 
to alter the game or characters' interactions far more than any other Disciplines in the game.  
Rules specific to each power are presented in their own sections, but the following rules apply to 
every use of blood sorcery, whether it is a ritual or regular Discipline.

\begin{itemize}
	\item This volume contains every power available to player characters and most powers available to 
	non-player characters.  It is not possible within the scope of a chronicle to develop a new path 
	or ritual for either Thaumaturgy or Necromancy.
	\item Nearly all powers require a challenge to be thrown, usually retested with the \emph{Occult} 
	Ability.  Each such challenge should be thrown with a Storyteller present, due to the complexity 
	of interactions with other powers or characters.
	\item Unless otherwise specified, the casting of Thaumaturgy or Necromancy is an obvious and evident 
	process, requiring both hand gestures and vocal incantations.  If used in combat, the use of these 
	Disciplines takes the entire combat turn and can not be sped up with Celerity.  All Thaumaturgy and 
	Necromancy powers resolve at the end of the combat turn unless otherwise specified.  Even with 
	Celerity the casting character may not perform additional actions save for movement.
	\item Some powers allow for the creation or alteration of items.  No such item may have more than 
	one magical affect upon it at a time; at Storyteller discretion either the more powerful effect 
	takes precedence or the newest effect supplants the old.  This prohibition also applies to the 
	Quietus powers of \emph{Scorpion's Touch} and \emph{Baal's Caress}, among others.
	\item Many powers require a Mental versus Physical Challenge.  Typically these challenges require 
	the defender to bid a Dexterity trait to avoid the power's effects.  If a victory condition is solely 
	damage, Stamina may also be used.  In all cases Storytellers are the final arbiter of which traits may 
	be used in a given challenge.
	\item Ranged powers require a ``line of effect'' when determining valid targets for a Discipline.  
	For a ranged subject to be a viable target there must be no solid barrier between you and, depending on 
	Storyteller fiat, the path must be relatively clear of obstruction.  Even transparent materials such as 
	glass will block ranged Disciplines.
\end{itemize}