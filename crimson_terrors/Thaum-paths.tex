\subsection{Thaumaturgy Paths}
\label{sec:tpaths}
The Discipline of \emph{Thaumaturgy} is divided into many paths, each with a different 
focus or theme.  All uses of the Disciplines listed in this section require a Blood 
expenditure, as well as any other listed costs.

Each Thaumaturge has a Primary Path of study which is determined by their Clan, and represents an 
affinity or focus that Clan has for a particular set of powers.  While learning other paths is 
possible, and often encouraged, certain conditions must be met before Storytellers will approve 
the purchase of additional dots in additional paths.

Before learning either of the Basic levels of a Secondary Path, the magus must have at least an 
Intermediate level of understanding in their Primary.  Similarly before gaining the Intermediate 
powers of a Secondary Path, they must have mastered the Advanced power of their Primary.  A 
similar restriction follows all future paths under study, called ``Tertiary Paths,'' and which 
are limited by one's comprehension of their Secondary Path.  Once a character has mastered both 
their Primary and Secondary paths, there is no longer a restriction on what the character can 
learn, from a mechanical perspective.

Almost all users of Thaumaturgy however must ask permission from their superiors before embarking 
on new studies or advancing existing Paths.  This serves as a fantastic opportunity for role-play 
as the student pleads their case, also ensuring the player knows what their character will be 
learning before diving in blindly.  In all cases, requests to advance one's knowledge of 
Thaumaturgy requires specific Storyteller approval.

As with all other Disciplines levels one and two are considered Basic, levels three and four 
Intermediate, and the final dot is considered the Advanced level.

\subsubsection{Alchemy (Chrysopoeia)}
Overlooked by most twentieth century thaumaturges, elder Tremere still practice this Path as a 
reminder of the principles modern blood magic is based on.  Dealing with Hermetic ideals of reality, 
and less practical applications thereof, this path transmutes one substance into another, normally 
without changing the form or function of the raw materials at hand.  Anyone hoping to utilize this 
Path must possess either two levels of \emph{Science: Chemistry} or four \emph{Academics} and must 
be working in a controlled environment, much akin a modern-day laboratory or clean room.   Each 
application of \emph{Alchemy} takes one hour per level being used, and the caster must have exacting 
knowledge of the chemical makeup of the material they are working with.  All levels require a 
Static Mental Challenge to succeed, with a difficulty set by the Storyteller, retested with 
the appropriate \emph{Science}.  \emph{(Availability: A4.  Source: ST p.47)}

\begin{description}
	\item[1 -- Commuta:]  This power allows the thaumaturge to change the state of a single, uniform 
	material; solid to gas, gas to liquid, liquid to solid, or the reverse.
	\item[2 -- Conforma:]  With this power the caster can separate out simple elements, creating loose 
	clouds of hydrogen and oxygen from water, or forcing liquid into a specific, simple solid shape.
	\item[3 -- Scinde:]  At this level complex form changes can be made, such as changing water into 
	breathable oxygen and free hydrogen, compounds into piles of separate elements, and so forth.
	\item[4 -- Mutate:]  Minor shifts in composition are now possible, such as shifting a single element's 
	atomic number by up to five on the Periodic Table of Elements.
	\item[5 -- Transmutate:]  The thaumaturge may now force radical shifts in composition, functionally 
	allowing them to transform any starting element into any other element.  Any use of this power will 
	be carefully overseen by the Storytellers to ensure proper game balance.
\end{description}

\subsubsection{Biothaumaturgy (Bioconmutatus Arcanum)}
A truly macabre path which deals with flesh in all states, is possibly the inspiration for such works 
of fiction as ``Dr. Frankenstein,'' and those who practice it are regarded with curiosity and revulsion, 
even within the Sabbat which developed it.  This Path does not require any Blood traits to be spent in 
its use, but individual costs are listed under each power.  Anyone wishing to utilize this power must 
possess at least three levels of \emph{Science: Biology} (or similar) and one dot of \emph{Medicine}.  
All retests are performed with the relevant \emph{Science}, and uses of this power must be performed in 
a workshop or laboratory prepared for the purpose.  \emph{(Availability: Sabbat.  Source: ST p.48)}

\begin{description}
	\item[1 -- Thaumaturgical Forensics:]  After a week's worth of study on a plant, animal, or 
	supernatural tissue sample and succeeding on a Static Mental Challenge, difficulty set by the 
	Storytellers, this power allows for a wealth of information to be obtained, far beyond what 
	``normal'' forensics or genetic testing would yield.  The information will always be physical 
	in nature, and it is up to the Storyteller how much is gleaned under a given examination.  
	This study destroys the sample and requires the expenditure of one Willpower at the start of the 
	week.
	\item[2 -- Thaumaturgical Surgery:]  Further studies into this Path yield amazing recuperative 
	abilities for a caster's patients.  By spending up to three Mental traits and a like number of 
	Blood, neither of which can be replenished by any means until the next evening, the caster can 
	force a subject to heal.  For each Mental trait spent one wound of the caster's choice downgrades 
	from Aggravated to Lethal, or Lethal to Bashing, or Bashing to nothing, which requires ten minutes 
	of dedicated ``surgery'' time per wound.  This power may only be used once per evening, regardless 
	of the number of available patients.  The subject must spend the majority of the rest of the night 
	in the laboratory, under the caster's care, for this power's effects to hold.
	\item[3 -- Lesser Animation:]  Imbuing an intact, dead being with a part of their own undead life 
	essence, a biothaumaturge can create small golems, no larger than dogs, to fulfill simple 
	instructions.  These animated corpses have reduced Physical traits and wound levels from what they 
	had in life, and zero Mental or Social traits, though they are immune to \emph{Dominate} and 
	Social-based powers as a result.  Immediately crumbling to dust if exposed to sunlight or fire, 
	these creatures can nevertheless follow a single basic, one-sentence command issued by the caster, 
	which it will follow to the best of its ability for the rest of its pitiful life.  Animating a 
	corpse requires at least seven nights of preparation and experimentation, culminating in a 
	Static Mental Challenge with a difficulty set by the Storytellers.  The use of this power requires 
	the permanent sacrifice of a Blood trait which remains unavailable until the creature is destroyed.
	\item[4 -- Greater Animation:]  As \emph{Lesser Animation} save the caster may now reanimate 
	creatures of almost unlimited size, or graft additional body parts onto existing animated corpses.  
	To awaken larger creatures requires two weeks of study while making modifications requires only a 
	single week of experimentation.  In both cases the biothaumaturge must succeed in a Static Mental 
	Challenge, with a difficulty set by the Storyteller.  Truly large beings may require the sacrifice 
	of multiple Blood points.
	\item[5 -- Cognizant Construction:]  With three weeks of work, the caster may imbue their creation 
	with some semblance of mental acuity, though always less than it had in life, its intellect turned 
	to sinister purpose.  While it now loses its immunity to \emph{Dominate}, the caster may now change 
	the orders given to such a creation at any time, unlike with lesser beings.  All creatures so 
	created require a difficult Static Mental Challenge.
\end{description}

\subsubsection{The Path of Blood (Rego Vitae)}
The Primary Path for all Tremere characters, this path represents a near-total mastery of the 
vitae which fuels their night to night existence, both within themselves and, eventually, within 
others.  \emph{(Availability: any Tremere.  Source: LotN p.177)}

\begin{description}
	\item[1 -- A Taste for Blood:]  By imbibing a trace amount of another's vitae (which can form 
	or advance a Blood Bond), you may learn the following information about the individual:  
	how much Blood is currently in their system, how recently they have fed, if and to what degree 
	they are wounded, a vampire's Generation, and whether or not the vampire is a Diablerist.
	\item[2 -- Blood Rage:]  Requiring light contact, which may necessitate a Physical Challenge, this 
	power allows you to force another kindred or ghoul to spend a single Blood trait in any fashion 
	you desire, even over the limits of their Generational ability to do so in a single turn.  This 
	power is sometimes used to bring kindred out of torpor, forcing their bodies to heal.
	\item[3 -- Blood Potency:]  This amazing power allows the thaumaturge to temporarily lower their 
	Generation by spending two Mental Traits for each Generation, to a maximum of six Traits.  This 
	power lasts for a scene or hour and while so affected you may claim your temporary Generation for 
	the efficacy of Dominate, total blood pool, and trait maximums.  If you are diablerized or attempt 
	to sire childer while in this state however, use your actual Generation.
	\item[4 -- Theft of Vitae:]  By engaging in a Mental versus Physical Challenge with a target within 
	fifty feet, you are able to effectively feed at range.  This power requires the expenditure of up 
	to three Mental Traits before throwing the challenge, which determines how many Blood traits you 
	can steal from another, one per.  Blood gathered this way flies directly from your target into your 
	hands and is not purified in any way -- using this power on another vampire creates a Blood Bond, if 
	the blood is tainted you will become sick, et cetera.  This power is an enormous breach of the Masquerade.
	\item[5 -- Cauldron of Blood:]  After establishing a firm grip on a target, a thaumaturge can boil 
	the literal blood out of them.  By spending up to three Mental Traits and then engaging in a Mental 
	versus Physical Challenge (Stamina traits are a valid defense), you may cause a like number of Blood 
	traits to boil out of their body, causing one Aggravated wound apiece.  Mortals are almost instantly 
	killed by this attack and you may not successfully boil more Blood than they currently possess.
\end{description}

\subsubsection{Path of Curses (Crea Maledicta)}
Rumored to exist for as long as vampires have been studying blood sorcery, this Path provides powerful 
hexes and curses upon its victims.  While powerful this Path is also dangerous, for each incantation 
must be said directly to the target.  The thaumaturge wishing to use this power upon another must have 
a symbolic link to fuel the magic -- a bit of blood, a treasured object, or the like, and then best 
the target in a Social challenge, retested with \emph{Intimidation}.  A new symbolic link is required 
for each casting, even against the same target.  All powers last until sunrise unless otherwise specified, 
or until lifted by the caster, whichever is first.  \emph{(Availability: Non-Camarilla.  Source: ST p.56)}

\begin{description}
	\item[1 -- Stigma:]  This power marks the target with an aura of revulsion, making others uncomfortable 
	to be around them.  The subject of this curse must bid an extra trait in all Social challenges, and loses 
	all ties in such challenges.  This power lasts until the following sunset.
	\item[2 -- Malady:]  If successfully invoked this power afflicts the subject with a terrible sickness, 
	causing them to suffer a two trait penalty on all Physical challenges due to discomfort.  The illness 
	is much like the flu, with similar symptoms, which the victim should role-play appropriately.
	\item[3 -- Pariah:]  Twisting the perceptions of a victim's peers, this power forces all who look upon the 
	subject to see their most hated foe, though their voice is unaffected.  This power works as an offensive 
	\emph{Mask of a Thousand Faces}, save that the person under its effects cannot break it themselves.  \emph{Auspex} 
	can test to see through this power normally.
	\item[4 -- Corrupt Body:]  By denouncing the physical form of the accursed, the caster forces the victim's 
	body to warp into a grotesque parody of what it once was.  This outward caricaturing brings with it mind-wracking 
	pain, and subjects who have undergone this curse are often scarred both physically and mentally.  This power 
	reduces the target's maximum Physical traits to half, confers two Negative Physical traits chosen by the 
	Storyteller, and renders all Appearance-based Social traits unavailable for the duration of the curse.  The 
	target must also spend one full turn doing nothing but writhing in agony as the curse takes hold.
	\item[5 -- Fall from Grace:]  If successful in the application of this power, the thaumaturge lays a truly 
	disastrous curse upon the victim, ensuring the odds are always stacked against them.  The subject loses all 
	ties in all challenges and cannot use any retests, including those given by Merits or Disciplines.  Any use 
	of this power will be carefully overseen by the Storytellers to ensure proper game balance.
\end{description}

\subsubsection{The Path of Conjuring (Crea Materia)}
Creating objects from nothingness is the practice of Conjurers who draw on their own knowledge and experience 
to craft items and materials from empty space itself.  Objects created with this path are generic, without 
distinguishing marks or ornamentation, and are the same very time.  It is impossible for a caster to conjure 
anything larger or heavier than themselves, and they must have a working knowledge of the object to be brought 
forth, which may require specific Abilities.  Any object that is created with this power that is broken apart 
or otherwise severely damaged will melt into unusable sludge as the spell fades away.  All items created through 
this Path must be represented by item cards signed by a Storyteller which includes the amount of Mental Traits the 
caster had at the time of casting.  \emph{(Availability: Tremere A5.  Source: LotN p.182)}

\begin{description}
	\item[1 -- Summon the Simple Form:]  You are able to create basic objects made of only one material, little more 
	than chunks of matter.  Lacking any complex or moving parts, sample objects include a rod of metal, a club, 
	wooden stake, a rock, or lump of coal.  All objects created with this power require a Mental Trait to be 
	sacrificed at the start of each turn, or roughly three seconds, or the object dissolves.
	\item[2 -- Permanency:]  After an object is created with \emph{Summon the Simple Form}, spend three additional 
	Blood Traits to invest the creation with the gift of continued existence, no longer requiring additional upkeep.
	\item[3 -- Magic of the Smith:]  Objects that contain mixed materials, moving parts, and are complicated in 
	form or design are now within the conjurer's ability to create.  As long as they are familiar with an object's 
	workings, represented by dots in a relevant Ability, they can create a relatively plain copy of it out of thin 
	air.  Creating an object with this power costs five Blood traits instead of the usual one, and includes the 
	effects of \emph{Permanency} at no additional cost.
	\item[4 -- Reverse Conjuration:]  By spending a full turn in concentration a conjurer may unweave any item created 
	by this Path, banishing it wholly into the nothing from which it emerged.  A caster's own objects may be dismissed 
	without a challenge, but objects created by others requires a Static Mental Challenge versus the number of traits 
	listed on the item card.  Without the use of other powers there it is not usually possible to determine whether 
	an object is real or conjured.
	\item[5 -- Power Over Life:]  Though you cannot create true life, through this power you can create a creature that 
	has at least the semblance of it.  Using this power requires the expenditure of ten full Blood traits, and the creation 
	lasts for one week before dissolving.  These simulacra have no will of their own and follow your commands mindlessly, 
	doing their best to follow simple directions.  At all times the actions and statistics of a creature conjured with this 
	power are under the control of the Storytellers.
\end{description}

\subsubsection{The Focused Mind (Patet Mentis)}
A truly rare Path, students of its teachings require serenity and dedication to train in, aspects often lacking in the 
modern, chaotic nights.  Its powers provide a thaumaturge with powerful cognitive abilities, perhaps only recently 
paralleled with the rise of computers.  Any wishing to learn this Path must have at least seven permanent Mental traits.  
\emph{(Availability: A6.  Source: ST p.50)}

\begin{description}
	\item[1 -- Readiness:]  By succeeding a Static Mental Challenge against five traits the thaumaturge becomes more alert 
	and aware of their own memory.  For the next scene or hour the caster is up three traits Mental Challenges related to 
	leaps of logic, investigation, memory, or reasoning, at Storyteller discretion.
	\item[2 -- Centering:]  Speaking soothing words as a mantra, this power's effects may be used by the caster or upon 
	another willing subject.  By sacrificing up to three Mental traits and then succeeding on a Static Mental Challenge 
	against six traits, the subject is unaffected by wound penalties---Bruised, Wounded, or Incapacitated---and gains one 
	trait to resist persuasion per Mental trait sacrificed.  These bonus traits may be used in challenges to resist or 
	remain in frenzy, and challenges against mood-altering Disciplines.
	\item[3 -- One-Tracked Mind:]  Narrowing another's focus to a single desire, with a successful Contested Mental 
	Challenge and one full turn of conversation, the thaumaturge ensures that the subject will continue to fixate upon a 
	single thought or action that was undertaken immediately before or during the casting of this power.  This power lasts 
	a single scene or hour, until the individual is the subject of violence, or until their task is completed.  While so 
	affected this power's target receives a one trait bonus on all challenges related to their focused task.
	\item[4 -- Dual Thought:]  By spending a Willpower the caster may now take a Mental action during the Swiftness 
	combat round in the following combat turns, excluding other uses of \emph{Thaumaturgy}.  Once cast the thaumaturge 
	could \emph{Dominate} two different people per turn or work through logic problems with blinding speed.  This power 
	does not reduce or remove any cost associated with such actions, and lasts for either a single combat scene or ten minutes.  
	This power cannot be combined with \emph{Celerity} to receive two different actions in the Swiftness round.
	\item[5 -- Perfect Clarity:]  Steeling their resolve against all aggressions, the thaumaturge who has reached this level 
	of mental focus can shrug off external influences with a moment's concentration.  By spending a full turn in whispered 
	meditation and spending a Willpower, the caster becomes temporarily immune to all sources of frenzy, gains a 
	two-trait bonus to all defensive challenges, and all opponents attempting to influence or control the thaumaturge in any way, 
	including through \emph{Dominate} or \emph{Presence}, lose all ties to do so.  This power lasts for one combat scene 
	or ten minutes.
\end{description}

\subsubsection{Hands of Destruction (Perdo Materia)}
Magicians within the Sabbat have turned their focus toward violent and brutal applications of Thaumaturgy, and so have 
wrought this Path, which is rumored to have come from pacts with infernal demons.  As with all powers, effects which require 
touch first require a Contested Physical Challenge before the casting of the Discipline in question.  \emph{(Availability: 
Sabbat.  Source: LotN p.183)}

\begin{description}
	\item[1 -- Decay:]  Inanimate matter crumbles rapidly under your touch, aging a full year for each turn you 
	maintain contact, which can reduce wood or organic matter to rotted morass quickly, and even weaken metal or plastic 
	with sufficient time.  Undead flesh is susceptible to use of this power, though it only discolors and provides no 
	pentalties to the victim.
	\item[2 -- Gnarl Wood:]  Up to fifty pounds of wood, whether a door or a pile of split lumber, can be forced 
	to swell, contract, or twist into strange shapes with a glance.  Affecting wood held by someone, such a stake, requires 
	a Mental versus Physical Challenge.
	\item[3 -- Acidic Touch:]  Able to turn your own blood into a viscous and caustic acid, with a single touch you 
	can mar or burn most surfaces with a touch.  You may excrete the black ooze from any flesh on your body, which 
	does not harm you but deals a single Aggravated wound to any person or surface it touches.  This power may take 
	multiple applications to burn through solid, large, or particularly dense objects, at Storyteller discretion, and 
	cannot be flung or spit as a ranged attack.
	\item[4 -- Atrophy:]  With a single touch you are able to wither the limbs of your opponents, turning them into 
	useless, fragile appendages.  A limb struck in such a way shrivels, granting the victim \emph{Clumsy} and 
	\emph{Lame}, which can be stacked up to four times, once each per limb successfully struck.  A victim without 
	arms cannot grapple or wield weapons, a victim without functional legs cannot move.  While this effect is permanent 
	on mortals, vampires can heal the effects as if they were a single Aggravated wound per limb.
	\item[5 -- Turn to Dust:]  After maintaining a firm grip on your victim, engage them in a Mental versus Physical 
	Challenge.  Each Mental trait you expend before this challenge will age the target ten years if successful, causing 
	aged mortals to crumble into dust and vampires to become \emph{Repugnant} for the rest of the night.  This power 
	is only effective against living or undead targets; \emph{Decay} is for inanimate subjects.
\end{description}

\subsubsection{Hearth Path (Arcem Praesidio)}
Originally a collection of rituals, this Path deals with the security of one's haven.  All uses of this power 
may be ended at any time.  This Path only functions on locations important to the caster.  All powers last 
until the next sunset after casting.  \emph{(Availability: ???.  Source: ST p.51)}

\begin{description}
	\item[1 -- Guest's Herald:]  By placing small blood sigils on the inside of a door or other entryway, this 
	power creates a small auditory or visual effect that will alert the thaumaturge that someone has breached 
	the opening.  The signal can be anything the caster desires, so long as the effect is subtle.  The caster 
	must be within the haven for this power to have any effect.
	\item[2 -- Master's Order:]  Reaching a new level of connectedness with one's haven, through the use of this 
	power a thaumaturge instantly knows where each one of their inanimate possessions are within their haven, so 
	long as caster is inside.
	\item[3 -- Rhyme of Discord:]  This power befuddles the minds of intruders, causing them to become lost, 
	no matter how small or plain the haven actually is.  Any intruders must succeed in a Mental Challenge against 
	the caster, retested with \emph{Occult}, or else be trapped inside the haven until the thaumaturge breaks 
	the spell or it wears off naturally.  All individuals so affected will be unable to remember the details of 
	the protected haven even after they have left, unless at a future point they return when the haven is not 
	protected by this power.  All intruders may still fight and defend themselves normally, even if they could 
	not effectively pursue the thaumaturge.  This power does not require the thaumaturge to be home.
	\item[4 -- Temportal:]  A master of his haven, the caster may cause doorways inside to connect however he 
	chooses to link them, such as making shortcuts between floors or to a secured basement.  Anyone other than the 
	caster using or passing through the doors is unaffected by this power, often leading to a great deal of 
	confusion as the caster seems to disappear into thin air.
	\item[5 -- The Caultron's Rede:]  By imparting a small amount of awareness into the haven itself, a 
	master of the Hearth Path can ask any or all of the household items, or the building itself, what transpired in 
	or near the haven during the time this power was active, though it will not remember anything from previous 
	castings.  In cases of extreme danger, such as fire or an impending ambush, the voices of the haven will 
	actively scream out to the kindred, regardless of location.
\end{description}

\subsubsection{The Lure of Flames (Creo Ignem)}
One of the most feared powers in Thaumaturgy, this path allows the caster to control fire, that eternal 
enemy of kindred.  A practitioner of this path does not risk R\"{o}tschreck while in control of the flames, 
but secondary or unintentional fires pose a danger as normal.  The first three levels of this Discipline 
only require a single action of casting, not a full combat turn.  \emph{(Availability: any Tremere.  Source: 
LotN p.178)}

\begin{description}
	\item[1 -- Hand of Flame:]  Wreathe one or both hands in a flickering firelight, sufficient to light 
	a small room and cause other kindred to recoil.  This power lasts one scene or hour, or until you 
	decide to extinguish the flames.  All successful \emph{Brawl} attacks deal a single Aggravated wound, which 
	can be combined with powers such as \emph{Celerity} or \emph{Potence}.
	\item[2 -- Flame Bolt:]  By pointing at your target you may manifest a streak of flame which flies toward 
	your target.  With a successful Mental versus Physical Challenge you cause one level of Aggravated damage 
	to a victim within fifty feet.  If striking a readily flammable object such as a stack of papers, a 
	secondary fire may start.
	\item[3 -- Wall of Fire:]  Causing a column of flame to erupt from the ground, you can extend the fire to a 
	roaring curtain of flame.  The fire you cause reaches six feet high and can either form a straight line 
	barrier of ten feet or a circle with a six foot diameter.  If you attempt to conjure the wall beneath someone, 
	engage them in a Mental versus Physical Challenge.  If you win they take a single Aggravated wound as flames 
	erupt around them.  Anyone attempting to pass through or standing inside the barrier once created suffers a 
	level of Aggravated damage per turn.  The wall lasts for one scene or hour, or until you are knocked 
	unconscious, move fifty feet away, or dismiss it.
	\item[4 -- Engulf:]  By concentrating on a subject for a full combat turn you may cause them to burst into 
	flames, remaining alight until they are put out.  By succeeding in a Mental versus Physical Challenge the 
	victim suffers two Aggravated wounds as they are engulfed, perhaps causing other secondary fires to start.  
	At the end of each successive turn they suffer an additional Aggravated wound until they or a friendly 
	onlooker take a full combat turn to put themselves out.  Using this power on someone multiple times does 
	not increase the end of round damage, but does cause the initial wounds as normal.
	\item[5 -- Firestorm:]  Creating a swirling vortex of flames, any area you can see within fifty feet can 
	become the epicenter of a truly terrifying blaze.  Everything within a twenty foot diameter is subject to 
	roaring sheets of flame, which continue to strike until you dismiss them, move farther than fifty feet away, 
	or are knocked unconscious.  Make a mass Mental versus Physical Challenge with all in the area of effect -- 
	any who fail are struck in the initial blaze, suffering one Aggravated wound.  Everyone still inside the 
	area of effect at the end of successive combat turns suffer an additional Aggravated wound each turn.
\end{description}

\subsubsection{Mastery of the Mortal Shell (Dominum Pupa)}
This Path provides control over the physical workings of a body.  While devastatingly effective against living 
creatures, the undead are able to shrug off its influence by spending a Willpower at any time. This power 
cannot be used to re-animated corposes.   Unless otherwise stated effects last for a number of combat turns equal 
to the number of Mental traits spent in the casting, to a maximum of 3, and each power requires two challenges:  
one to establish a Physical touch, and a Mental versus Physical challenge to enact the power.  As normal these 
two challenges may be performed in the same action.  \emph{(Availability: A5.  Source: ST p.53)}

\begin{description}
	\item[1 -- Vertigo:]  After successfully touching a victim and besting them in a Mental challenge, the caster 
	can force their subject to suffer a wave of dizziness and disorientation, resulting in a one trait penalty to 
	all Challenges for the remainder of the scene, and may cause some effects such as acrophobia at Storyteller 
	discretion.
	\item[2 -- Contortion:]  With a successful touch to an arm or leg and follow-up Mental challenge, this power 
	forces the target to lose control of the affected limb, suffering the Negative Trait \emph{Lame}, and other 
	related penalties as determined by a Storyteller.  Additionally the caster may cast this upon themselves to 
	tighten their muscles, granting themselves two bonus traits for use when maintaining a grapple.
	\item[3 -- Seizure:]  Causing uncontrollable convulsions on their targets, this power forces the victim to 
	writhe on the ground, suffering one level of Bashing damage per turn, and making it impossible for them to 
	initiate Physical challenges, able only to defend with Stamina traits when applicable.  This power requires 
	the full concentration of the caster.
	\item[4 -- Body Failure:]  This power will all but instantly kill mortals as their autonomic nervous system 
	simply shuts down, and is extremely uncomfortable and painful to kindred.  By spending a Willpower the 
	caster causes the victim to suffer total system shock, causing three Lethal damage immediately and one more 
	each additional turn they are affected by this power, to a maximum of three.  Kindred affected drop to the 
	floor and may not initiate Physical challenges, only defending with Stamina traits where applicable.  Kindred 
	may overcome the effects of this power for one turn by spending a Willpower.
	\item[5 -- Marionette:]  After a successful Physical challenge to establish a grip, at any point later in the 
	same scene the thaumaturge may engage the victim in a Mental versus Physical challenge to establish dominance 
	over their entire body, forcing them to perform any physical actions desired, including speech.  The subject's 
	movements are not jerky or suspicious, acting naturally and ably as the body under control.  Because of the 
	intense concentration required for the use of this power, the thaumaturge cannot initiate challenges while 
	controlling a victim in this way.  The victim knows full well that their body has been usurped by an outside 
	force, but can do nothing to prevent acting in a way the caster desires for the duration.
\end{description}

\subsubsection{Movement of the Mind (Rego Motus)}
Through this power objects and even people can be manipulated from afar, though this control does not provide 
any tactile response.  Any subject within visual range can be a target for this power.  \emph{(Availability: any 
Tremere.  Source: LotN p.180)}

\begin{description}
	\item[1 -- Force Bolt:]  By winning a Mental versus Physical Challenge with your target you may knock them 
	off their feet until they spend one action to get up, or knock an item out of their hands.  This power can 
	also be used against unattended objects weighing less than one-hundred pounds, which may get shoved several 
	feet at Storyteller discretion.  This power requires only one action to use.
	\item[2 -- Manipulate:]  Through force of will you can exert fine manipulation over something at range.  
	With this power you may attempt to use any object you could normally wield with one hand, such as picking 
	something up, pushing a button, or firing a gun.  Using an object remotely requires your full concentration 
	and the difficulty of doing so requires that you bid an additional trait in all challenges related to the 
	item.  Control over the object continues so long as you maintain your concentration.
	\item[3 -- Flight:]  With this power you are able to lift a whole person, bodily lifting them from the ground.  
	Other large objects are also able to be lifted, but without fine control.  When invoking \emph{Flight} you are 
	able to lift any object of up to three-hundred pounds at a brisk walking speed (2 steps per combat turn).  
	This power lasts for as long as you are able to fully concentrate and see the target.  Lifting an unwilling 
	individual requires the success of a Mental versus Physical Challenge.
	\item[4 -- Repulse:]  Through the use of this power a physical shockwave blasts into your surroundings, knocking 
	objects and even people away from you.  You can affect any number of objects within your line of sight, and such 
	items fly up to twenty feet away.  Targeting individuals require a Mental versus Physical Challenge.  To strike 
	someone with a \emph{Repulsed} object requires a similar challenge, and likely deals one Lethal damage, at 
	Storyteller discretion.
	\item[5 -- Control:]  Proving mastery over telekineses, this power allows you to completely direct an object's 
	movements through the air, obeying your magical phrases and gestures.  Anything up to one ton in weight can be 
	the target of this Discipline, and you are able to manipulate it with the same dexterity as if you were using 
	both hands.  With this power you can use the item as a weapon, with damage depending on the object wielded, or 
	even control a melee weapon remotely, though this latter action requires you to bid two extra Traits for any 
	associated challenges.  People grabbed with \emph{Control} can be rendered nearly paralyzed or can be flung about 
	at your whim.  Initiating this power requires a Mental versus Physical Challenge, repeated every time you wish 
	to move the victim.  Exercising this level of manipulation requires your total concentration, and the power ends 
	if you become distracted, take another action, or lose sight of the subject.
\end{description}

\subsubsection{Neptune's Might (Rego Aquam)}
Though vampires often find swimming difficult, the sea has been a place of wonder and power for civilizations 
across the ages, and kindred kind is no different.  Once a thaumaturge reaches the Intermediate level of this 
path they may choose to specialize in either salt or fresh water, if desired.  This choice, or lack thereof, 
cannot be later unmade or changed.  If made, the caster receives a +2 bonus to using this power on the water 
they chose, but must bid two additional traits to affect the type of water they did not.  Blood is considered 
to be neither salty nor fresh for this purpose, and difficulties to manipulate it are unaffected.  
\emph{(Availability:  Tremere A4.  Source:  Camarilla Guide, p.81)}

\begin{description}
	\item[1 -- Eyes of the Sea:]  The thaumaturge may look deep into a body of water and see events that have 
	transpired in or around it, from the water's perspective.  Through this power the caster may see up to 
	twenty-four hours in the past.  By passing successive Simple Tests additional time may be observed as shown 
	below, though spending a Willpower counts as an automatic success.  This power can only be used on standing 
	water; lakes and puddles will do while oceans and rivers or wineglasses will not.
\end{description}

\begin{center}
\begin{tabular}{ | l l |}
	\hline
	\multicolumn{2}{| c |}{\textbf{Eyes of the Sea}} \\
	\hline
	1 Test & One additional week \\
	2 Test & One full month \\
	3 Test & One full year \\
	4 Test & Ten full years \\
	\hline
\end{tabular}
\end{center}

\begin{description}
	\item[2 -- Prison of Water:]  Exerting control over a body of water, you can force it to engulf and imprison 
	a victim.  A significant amount of liquid is required, though even a few gallons are enough to form binding 
	chains.  If insufficient water is present you must bid two additional traits in the initial Mental versus 
	Physical challenge to snare your target.  Mortals who are subject to this power may drown if you are not careful.  
	Spend additional Blood traits into creating this effect -- for every one you spend the prison gains two Physical traits.  
	Your level of \emph{Occult} is added as bonus Traits for any challenge the prison makes.  To break free, the 
	subject must defeat the prison in a Physical Challenge.  If desired you can order the prison to crush the 
	victim once per turn, initiating a Physical Challenge between the prison's traits and the subject's (Stamina 
	only, retested with \emph{Survival}).  Each successful challenge deals one Lethal wound.  The prison lasts 
	for one scene or hour, or until you move away from the area.  You must maintain line of sight to both the 
	prison and the source of water for the power to remain.
	\item[3 -- Blood to Water:]  Often used as a devastating assault against kindred and mortals alike, this power 
	allows the thaumaturge to transmute other liquids into water, even liquids inside a living body.  After establishing 
	a grip on the victim, requiring a Physical challenge, sacrifice up to three Mental traits and make a Mental versus 
	Physical challenge against the target.  If you succeed that many Blood traits are replaced with water, causing 
	excruciating pain and often death in mortals.  Vampires affected by this power have their current Blood pool reduced 
	and suffer penalties as if they had taken that many wounds, which automatically at a rate of one per hour and cannot 
	be healed by spending Blood.  Those possessing \emph{Fortitude} do not suffer the additional penalties, but still 
	lose vitae.
	\item[4 -- Flowing Wall:]  Commanding a powerful flowing wall of water to emerge from a sufficiently-sized body of 
	standing water is now merely a matter of effort for you.  Touch the surface of the water and spend two Willpower 
	traits.  For each Mental trait you then sacrifice you cause ten feet of watery barrier to appear in a single dimension, 
	either width or height, no more than one foot in thickness.  The wall can be placed anywhere in your line of sight and must 
	be formed in a straight line.  It remains in place until sunrise and cannot be climbed, though those attempting to pass 
	through it must make three Static Physical Challenges against your full permanent Mental traits (retested with 
	\emph{Athletics}).  Unless they succeed on all three Challenges they cannot pass.  This barrier also prevents travel by 
	psychic or spirit beings, who use their Mental traits instead (retest with \emph{Occult}).
	\item[5 -- Dehydrate:]  This level of mastery allows a thaumaturge to directly attack both mortal and supernatural 
	targets by stripping the water from their bodies at range.  Victims who die from this assault leave behind mummified 
	corpses.  Make a Mental versus Physical Challenge against a target within fifty feet.  If successful you inflict three 
	Lethal damage (if mortal) or strip three Blood traits (if kindred).  If a kindred is out of Blood they will suffer wounds 
	as if they were mortal.  Victim suffering damage must make a \emph{Courage} test against the amount of health levels 
	lost times two or collapse, overcome with agony for one turn.  Additionally this power may be used to dry wet clothing 
	or evaporate puddles.  
\end{description}


\subsubsection{Oneiromancy (Oneiromance)}
The unconscious thoughts and visions which swim through the slumbering mortal, and often immortal, mind have long 
been an interest to those wishing to gain a glimpse of the future that is to come, true diviners and paranoid 
Tremere alike.  Any use of this Discipline requires at least five minutes' worth of a dream-like trance during 
which time the practitioner interacts with the slumbering world.  \emph{(Availability:  Tremere A4.  Source:  ST's 
Guide, p.54)}

\begin{description}
	\item[1 -- Portents:]  By piecing together fragments of her own dreams, the magus may perform a reading of their 
	immediate future.  Casting this spell immediately after waking from a ten-minute dream trance and winning a 
	Static Mental Challenge against six traits provides for symbols and allegories directly related to an upcoming 
	event.  Multiple uses of this power will show the same vision until a new event becomes more pressing.
	\item[2 -- Foresee:]  This power allows the oneiromancer to use \emph{Portents} upon another sleeping individual, 
	so long as they are in the person's presence.  This power requires the caster to spend a Willpower if used on 
	a kindred.
	\item[3 -- Dreamspeak:]  At this level the thaumaturge may send messages, warnings, or even threats to a target 
	through dreams.  This power may be used upon anyone the magus has met, though the target must be asleep at the time 
	of casting.  Scenes lasting no longer than five minutes will repeat themselves through the victim's dreams, creating 
	a clear picture of the message upon waking.
	\item[4 -- Augury:]  Diving deeper into the collective unconscious, by spending half an hour in the dream trance and 
	winning a Static Mental Challenge against nine traits, the caster may receive a much clearer and deeper set of imagery 
	than what is found from \emph{Portents}.  The onieromancer may also use this power on anyone who has been the target 
	of a successful \emph{Foresee} or \emph{Dreamspeak} in the currentstory.  Remember that no matter the clarity, divination 
	is an imprecise and sometimes error-prone art.
	\item[5 -- Reveal the Heart's Dreams:]  By observing those around her, the thaumaturge gains incredible insight into 
	a subject's innermost fears and desires.  After spending at least five minutes in the company of the victim and winning 
	a Contested Mental Challenge (defended against with \emph{Subterfuge}), the caster may enter the dream trance later that 
	evening for fifteen minutes and receives the truth about either the subject's deepest fear or heart's greatest desire.  
	The answer to the question should be a driving force for the subject, not a passing interest or fear.  To use this power 
	on a mortal requires the expenditure of one Willpower at the time of the Challenge.  To use this power on a kindred 
	requires the expenditure of two Willpower.
\end{description}

\subsubsection{Path of Transmutation (Iter Transmutationibus)}
Regarded by elder Tremere as a poor man's \emph{Alchemy}, this path addresses the physical properties of matter within 
the thaumaturge's direct line of sight.  Its powers are very versitile, but can bring great destruction if used 
improperly.  All powers last for one scene or hour, or until dispelled by the caster.  \emph{(Availability:  Tremere A5.  
Source:  ST's Guide, p.60)}

\begin{description}
	\item[1 -- Fortify the Solid Form:]  By strengthening the bonds inside an object, the transmuter temporarily hardens 
	an object against damage.  An item with the Negative trait \emph{Fragile} loses it for the duration, armour becomes 
	more durable and able to soak an additional health level, and other uses have been employed by creative thaumaturges.  
	Enacting this power costs a Mental trait.
	\item[2 -- Crystalize the Liquid Form:]  This power will transform up to two pints of liquid into solid form, without 
	changing the substance's temperature or other physical properties such as its acidity or weight.  The solid form is 
	\emph{Fragile} and prone to breaking if not handled delicately, and all such transformed matter reverts to its natural 
	state at the end of the scene or hour.  This power costs a Mental trait, and only liquid that is open to the air can be 
	transmuted.  This power does not affect creatures who are already in a liquid form.
	\item[3 -- Liquefy the Solid Form:]  The reverse of the previous power, the thaumaturge must spend a number of Mental 
	traits relevant to the size of the object affected -- a hand-held object costs one, a door five, a barn eight or more.  
	At the end of the scene the puddle will revert to its natural form unless it is scattered.  This power does not work on 
	living or unliving beings.  Enacting this power requires a Static Mental Challenge against the number of traits 
	sacrificed.  Targets which may be affected by this power, such as by the floor becoming liquid, may receive a Dexterity-based 
	Challenge to jump away, at Storyteller discretion.
	\item[4 -- Goal:]  This power gives the transmuter control over the very air, solidifying it into an opaque, 
	indestructible and immobile barrier, even to kindred with \emph{Potence}.  \emph{Goal} has two different uses 
	which adds to its versatility:  an object or person may be encased in matter, or a wall created that is all but 
	impassible.  If creating a wall the caster must sacrifice a number of Mental traits relevant to the size of the wall, 
	as above.  If encasing a living being the caster sacrifices a like number of traits, usually five for a human-sized 
	creature, and engages in a Mental versus Physical Challenge, with the defender usually bidding a Dexterity-based trait.  
	The solid air is technically breathable, but mortals take a level of Lethal damage after being released as they cough 
	up the harsh material.  Creating a wall with this power requires a Static Mental Challenge against the number of traits 
	sacrificed.
	\item[5 -- Ghost Wall:]  At this level, thaumaturges find that even solid objects offer no resistance.  With 
	a full turn of concentration any inert material is rendered vaporous and may be passed through easily.  Floors underneath 
	foes may suddenly disappear, parachutes suddenly cease to work, and individuals can even suddenly find themselves falling 
	out of a speeding car that no longer supports their weight.  The caster spends a number of Mental traits relevant to 
	the size of the object being made gaseous, which returns to its former location after the power concludes.  Anyone standing 
	within the area of a recombining object suffers Aggravated damage, at Storyteller discretion.  Casting this level requires 
	a Static Mental Challenge against the number of traits sacrificed.  Any use of this power will be carefully overseen by the 
	Storytellers to ensure proper game balance.
\end{description}

\subsubsection{Vine of Dionysus (Dionysius de Vitis)}
Almost more a religion than a true thaumaturgical study, practitioners of this art are looked down upon by other 
Tremere for their lavish and hedonistic revelries.  Upon learning the first dot of this Path the thaumaturge automatically 
loses one permanent Willpower as they become ``addicted'' in its practices.  \emph{(Availability:  A5.  Source:  ST's Guide, p.61)}

\begin{description}
	\item[1 -- Methyskein:]  By making physical contact with their target and besting them in a Contested Mental Challenge, 
	the caster can make them feel drunk, suffering a one-trait penalty on all Dexterity- or Intelligence-related challenges, 
	forcing them to behave with the slurred speech and clouded judgment of intoxication.  If used on a mortal or ghoul three 
	nights in a row they can become addicted to this euphoria and may suffer ill effects per the \emph{Addiction} Flaw.
	\item[2 -- Omophagy:]  This power overwhelms the target with feelings of insatiable hunger, requiring only eye contact 
	and a Contested Mental Challenge to enact.  Mortals eat until they become sick and even then continue to want more, gorging 
	further after they vomit.  Kindred affected will not only drain a vessel completely of blood, but can sometimes even cause 
	them to feed upon other kindred, and even attempt to devour the victim's flesh, though without the merit \emph{Eat Food} this 
	behavior leads to predictably messy results.  A vampire affected by this power seeks out the easiest prey readily available, 
	at Storyteller discretion, and all victims lose awareness of their surroundings, in some cases causing likely breaches to the 
	Masquerade.  In no case will this power force a kindred to commit diablerie.  By spending a Willpower a victim can resist 
	falling to their urges for a few minutes, hopefully enough to exit the scene.  This effect lasts one scene or hour on vampires 
	and for the rest of the night on mortals.
	\item[3 -- Hamartia:]  With physical contact and a successful Contested Mental Challenge, the thaumaturge induces a severe 
	intoxication in the victim, to the point of perversity, delirium, and possibly torpid stupor.  A victim becomes two traits down 
	on all challenges for the remainder of the scene, and acts in a manner according to their Negative Traits and Flaws, and otherwise 
	as determined by the Storyteller.  If this power is used upon another practitioner of this Path however, in addition to the 
	penalties listed above they also gain a two trait bonus on all challenges of Strength.
	\item[4 -- Enthousiasm\'{o}s:]  Exuding powerful pheromones, the caster may make a Contested Mental Challenge with all 
	targets inside a ten foot radius.  Any victim of this power falls into a narcotic stupor as pleasant, dreamlike images dance 
	in their minds, causing them to giggle, dance, or just sit and sigh happily.  Targets under this power's effects gain 
	\emph{Submissive x2}, \emph{Lethargic}, and \emph{Witless} as Negative Traits for the remainder of the scene.  Kindred may 
	ward off these effects for one turn by spending a Willpower.
	\item[5 -- Oinos Aimatos:]  By invoking the essence of the wine god into themselves, the caster transforms his blood into a 
	powerful narcotic elixir.  For one scene, anyone imbibing even the tiniest drop of the thaumaturge's blood will be affected 
	as per \emph{Enthousiasm\'{o}s}.  This power costs an additional Blood and one Willpower to cast.  If Blood is placed into a 
	communal punchbowl or the like, its presence can only be detected by \emph{Heightened Senses} or other similar powers.
\end{description}