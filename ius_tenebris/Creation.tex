\section{Character Creation}
\label{sec:creation}
Every player character (PC) in \emph{Sanguine Dreams} is a vampire with a history, goals, 
motivations, and a character sheet.  While in many games the character sheet comes first, 
our focus is on the collective story, and how each character can enhance and add to it.

Instead of starting out by creating a character sheet think of a living, breathing 
person to portray and then turn them into a vampire.  Having an idea of your character's 
motivations and personality goes a long way to making them fun to portray, far more 
than having the ``perfect'' character sheet or being the most mechanically sound to 
perform one specific task.  Much more important than what skills they'll have access to 
or what Disciplines may come naturally to them, think of the story of the character and 
you are sure to have a more rewarding experience.

Often great concepts can fit into multiple Clans, and so coming up with the circumstances 
of their embrace can be a great exercise.  Answering our Character Questionnaire 
(available at game) will go a long way to helping you flesh out your character and 
background, and as an added bonus will even net you a few XP for filling it out.  

Think about short-, medium-, and long-term goals.  Each goal should help further 
the character's advancement toward the next.  For example, with the long-term goal of 
becoming Prince your character will need allies, Status, a number of favors, and perhaps 
even help from outside the Domain.  Your goals should be an encouragement to role-play 
and a way of ensuring that you always have something to do while at game.  When someone 
approaches the Storytellers complaining of boredom, almost universally the first question 
asked is ``what are your goals?''  If you've thought of some great goals, boredom is 
exceedingly unlikely to set in.

Most likely the new character will not have started off in Sonoma County.  Think of where 
your character was raised, embraced, and then spent their accounting---very likely in 
different places.  Work with the Storytellers to solidify the feeling and mood of former 
Domains, and perhaps to flesh out some of the more important NPCs in the character's 
history.  Not all Domains are created equal, and many have unique variety you should be 
familiar with, particularly if other characters have traveled through the same cities.

Once you have a good handle on where the character has been, who they are, and where they 
are going, it's time to build the character sheet.  While it may look intimidating at first, 
following the following directions and using the information presented in subsequent chapters 
will make putting your character to paper a very straightforward process.

\subsection{The Mechanics of Creating a Character}
\begin{enumerate}
	\item Choose your character's Clan and record the associated advantages and disadvantages 
	from Chapter~\ref{sec:clans} .  Make a note of your character's in-Clan Disciplines.  
	This is a Vampire game and as such ghouls, mortals, and ``others'' are not allowed as PCs.  
	No special bloodlines or offshoots of the primary Clans are allowed for play.
	\item Decide on your character's primary, secondary, and tertiary trait categories 
	(Physical, Social, and Mental).  Fill in seven dots total in your primary category, 
	five in your secondary, and three in your tertiary.  See Chapter~\ref{sec:traits} .
	\item Assign five Ability dots.  See Chapter~\ref{sec:abilities} .  Note that no Ability 
	may be purchased to 4 or 5 at character creation.
	\item Assign five Background dots.  \emph{Generation} is unique in that it costs 
	2 points to raise, and you cannot have more than 2 dots in it without Storyteller 
	permission.  See Chapter~\ref{sec:backgrounds} .
	\item Assign three dots of Disciplines, to be spent in your Basic (levels one and 
	two) in-Clan Disciplines only.  See Chapter~\ref{sec:disciplines} .
	\item Choose your Nature and Demeanor.  See Chapter~\ref{sec:other} .
	\item Assign Morality path and Virtue traits.  All characters begin on the 
	Humanity morality path unless they spend 3 starting XP on purchasing a separate 
	path, with Storyteller permission.  Characters with Conscience and/or Self-Control 
	receive a free dot in each, and all characters have a free dot of Courage.  Assign 
	seven more dots.  Your total Morality rating is equal to your first two Virtues 
	(Conscience or Conviction + Self-Control or Instinct).  See Chapter~\ref{sec:morality} .
	\item Choose Merits and Flaws, if desired.  You may have no more than 7 points of Merits 
	and only the first 7 points of Flaws count toward Free Traits.  See Chapter~\ref{sec:merits}.
	\item Choose Negative Traits, if desired.  You may choose up to five, no more than 
	three in any one category.  See Chapter~\ref{sec:traits}.
	\item Count Free Traits:  each character begins with 5.  Every point of Flaws (up to 7) adds 
	an additional Free Trait, while every point in Merits costs one.  You may gain up to an 
	additional five by doing some or all of the following:  taking Negative Traits (1 each, maximum 
	of 5), lowering your starting Morality (1 point per dot, maximum of 2), or taking a derangement 
	(once for 2 points.  The Malkavian disadvantage does not grant points).  In no situation can 
	a character ever have more than 17 Free Traits (5 starting, 7 for Flaws, 5 for the rest).
	\item Spend Free Traits. These are spent the same as XP on page~\pageref{sec:xp}, save 
	that you may purchase the \emph{Generation} Background for two Traits each, and Physical, Social, 
	and Mental traits only cost one apiece.
	\item Calculate Willpower and Blood based on your Generation; see page~\pageref{bg:generation}.
	\item Spend 20 starting XP as described in Chapter~\ref{sec:xp} , including that which is 
	required for playing an uncommon Clan or for starting on an alternate Morality path as 
	listed below (note that either requires an extensive background to be approved):
\end{enumerate}

\begin{center}
\begin{tabular}{ | l c | }
	\hline
	\textbf{Clan} & \textbf{XP Cost} \\
	\hline
	Camarilla Seven & 0 xp \\
	Caitiff, Giovanni & 3 xp \\
	Follower of Set, Ravnos & 6 xp \\
	Assamite, Lasombra Anti-Tribu & 9 xp \\ \cline{2-2}
	\hline
	Alternate Morality Path & 3 xp \\
	\hline
\end{tabular}
\end{center}

\subsection{Special Notes}
\begin{itemize}
	\item \emph{Sanguine Dreams} uses descending trait caps:  based on the Generation Background 
	(see page~\pageref{bg:generation}), your secondary trait category maximum is two less than your 
	primary, with your tertiary reduced by two more.
	\item Remaining at 13th Generation is now a Flaw worth 2 points (see~Chapter~\ref{sec:merits}).
	\item No character may begin play with level 4 or 5 Abilities, Disciplines, or Influences
	\item All characters require Storyteller approval before play.  The existing game environment or 
	other circumstances may make some concepts or Clans unsuitable for new characters.
	\item Filling out a full background and character questionnaire nets you 10 bonus XP
	\item No PC Venerates or Elders will be allowed for play, and the ability to become an Ancilla 
	during game is an approval item. Only exceptionally rarely will characters be allowed to enter play 
	as Ancilla.
	\item Please allow for at least one week after submitting your character for the Storytellers to 
	review it, particularly if the character sheet includes unusual or rare elements, some examples 
	of which follow.  In the event that your character sheet or background includes any of these 
	items, a character background (not just a brief time-line) and questionnaire is required before 
	approval for play.
\end{itemize}

\subsubsection{Sample Approval Items}
While not all-inclusive, this list shows examples of items that require specific Storyteller approval before 
being allowed for play.  If a character sheet, concept, or background includes one or more of these elements 
a full and complete background and questionnaire are required before the Storytellers will consider the request:

\begin{itemize}
	\item Playing an unusual Clan or possessing a Morality path other than Humanity
	\item An out-of-Clan Discipline
	\item Interaction with truly noteworthy NPCs (Mithras, Villon, et al.)
	\item Merits or Flaws from other Sect books
	\item The ability to become Ancilla during play
	\item Over 150 years of unlife and/or more than 1 dot of the \emph{Age} Background
	\item Unusual or high levels of the \emph{Lore} Background
	\item Select Merits and Flaws such as \emph{Enemy}, \emph{Debt of Gratitude}, or \emph{Status}
\end{itemize}

\subsection{Character Sheets}
After submission to and approval by the Storytelling staff, they will provide a printed copy of your 
character sheet.  Until such time as your sheet has been logged by the Storytellers, we encourage you 
to ``soft-play'' your character, avoiding challenges where possible.  This allows you to role-play your 
concept and try it out, though you should not assume any particulars of your sheet, especially strange 
ones, until it has been approved by the Storytellers.

Similarly, after XP expenditures are recorded (see Chapter~\ref{sec:xp}) a new sheet will be issued, 
at which time you may start utilizing your new purchases.  It is very important to keep a current copy 
of your character sheet on you while portraying your character, either at game or during a downtime scene, 
remembering that only those elements which are present on a Storyteller-printed sheet are valid for play.

During a game your character may lose or gain Traits, Blood, Willpower, or any number of other statistics.  
It is very important to keep an accurate count of the current status of your sheet to prevent mistakes 
in the use of your character's capabilities.  It is encouraged that you keep a pencil with your sheet and 
mark off expenditures as they occur, to ensure there is no question about the state of your character.  
This is particularly important in high-stress situations such as combat.  The Storytellers will also 
endeavor to keep a current version of all character sheets available on the website, but it is every 
player's responsibility to track their temporary expenditures and losses.

If at any time you believe a player is mistaken about what their character can do or what powers they possess 
you may ask for a Storyteller to verify their sheet and that it is being used appropriately.  The scene should 
stop until such a verification can be made, to either correct a misinterpretation or confirm that play 
may continue.  Our collective aim is to foster a positive and encouraging role-playing environment, one that 
is fair to all players.

See the section \emph{Fair Play and Metagaming} in Chapter~\ref{sec:additional} for more information.