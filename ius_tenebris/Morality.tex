\section{Morality and Virtues}
\label{sec:morality}
\subsection{What is Morality}
Within every vampire is the Beast, a destructive and terrifying entity 
entirely concerned with two things:  feeding and survival.  The Beast 
is only kept at bay by a kindred's strict adherence to a set of 
principles akin to a code of ethics:  their Morality.

We as players recognize that stealing from others is bad, hurting them 
is worse, and murder is a terrible crime.  Born of humanity, all kindred 
begin with this same set of principles, called the Morality path of 
Humanity.  Humanity's dictates of right and wrong are easily understandable, 
since they mirror our own.  As a kindred begins to accept their predatory 
place in the world however, their Humanity begins to slip, causing them 
to find increasingly malicious and violent actions justifiable or 
even acceptable.  The lower one falls on any Morality path, the closer they 
are to falling to the Beast, forever lost in an eternal frenzy that from 
which there is no return, but conversely the better equipped they are to 
handle the dark and sometimes violent acts which may be necessary for their 
continued survival.

While Humanity is by far the most common Morality path, there are 
others which kindred have developed over the centuries to keep the Beast 
at bay and still further their individual aims.  These paths represent a 
truly alien mindset from our own, and can be very difficult to role-play.  
On some paths for example murder may be perfectly acceptable but failing 
to research long-forgotten lore is as deplorable as mass slaughter would 
be to us.  It is only through a strict adherence to a Morality path that 
a kindred is able to maintain their balance against the overwhelming 
fury of the Beast.

The different values of Humanity and what constitutes a sin at each level 
is presented below, and the sins and tenets of other paths can be found 
by asking the Storytellers.  Please note that most kindred do not know that 
other schools of thought exist, and the process of translating Humanity's 
moral compass to something so alien is not a journey lightly or easily taken.  
In fact, trying to change one's morality can only ever result in one of 
three outcomes:  the character becomes horrified at the prospect 
of losing everything they are and abandons the journey, the character 
fails and loses to the Beast, or the character succeeds and becomes a 
fledgling follower of a new path of Morality.  New characters may enter 
play on an alternate Morality path by spending 3 XP and providing the 
Storytellers with a very compelling background and history.  Existing 
characters must find a willing teacher currently on the desired path.

A character's Morality rating is relevant to their nightly existence 
in several ways---not only do they rise earlier and stay awake longer 
each evening with a higher path rating, but if they fall into torpor 
the minimum amount of time that passes before they wake is directly 
related to their Morality rating.  Actions during the day are also 
more difficult the closer a kindred is to the beast:  one may never bid 
more than three-halves of their Morality in named traits, rounded down.  
Characters with low (or no) Humanity have difficulty interacting peacefully 
with mortals.  On non-intimidating Social challenges with mortals, vampires 
suffer a penalty of seven minus their current Humanity rating; those 
on alternate paths have an effective Humanity of zero for such tests.

This distance from the trappings of humanity present itself in different 
ways:  as a character's Humanity slips they may not blink or fidget as much 
and will stop breathing save to speak.  They stare at others like a lion 
watches a gazelle; an experience that is most unsettling to mortals. Their 
skin grows ashen and papery, becoming unsightly.  The reality is that the 
Beast has neither need nor desire to fit into mortal society and so little by 
little these little automatic actions that help with fitting in fall away.  
Because the effects of low Humanity are behavioral as well as visual, even Obfuscate 
cannot hide the fact that the character is becoming something truly monstrous.

This change of behavior and outlook is so noticeable that kindred are easily able 
to determine the relative connection others have to their humanity.  With a Static 
Social challenge against another vampire (retested with \emph{Empathy}) a kindred 
can tell whether they are at high Humanity (8-10), medium Humanity (5-7), low 
Humanity (2-4), or inhuman (1 or on a different Morality path).  Kindred with high or 
middle Humanity are often rightly disturbed by the casual or nonplussed way in which 
lower-Humanity kindred discuss their night to night endeavors, while those lower in 
Humanity may consider others to be ``soft'' or in denial of the realities of life as a 
predator.

Your starting Morality rating is found by adding your Conscience/Conviction 
and Self-Control/Instinct traits together.

{\footnotesize
\begin{center}
\begin{tabular}{| c | l | l |}
   \hline
   \multicolumn{3}{| c |}{\textbf{The Sins of Humanity \& Length of Torpor}} \\
   \hline
   10 & Selfish thoughts & Two Days\\
    9 & Minor selfish actions & One Week\\
    8 & Injury to others (Accidental or otherwise) & One Month\\
    \hline
    7 & Theft & Six Months \\
    6 & Accidental violation (Drinking too much from a vessel) & One Year \\
    5 & Unreasoning destruction (People or property) & Five Years \\
    \hline
    4 & Impassioned violation (Manslaughter, killing while in frenzy) & Ten Years \\
    3 & Planned violation (Murder, savored exsanguination) & Fifty Years \\
    2 & Casual violation (Thoughtless killing, feeding past satiation) & 100 Years \\
	\hline
    1 & Utter perversion & 500 Years \\
    \hline
\end{tabular} 
\end{center}
}

\subsection{Virtues}
Each Morality path has three Virtues which represent your character's ability to 
adhere to the path and control their Beast.  Humanity utilizes the traits 
\emph{Conscience}, \emph{Self-Control}, and \emph{Courage}, but other paths may 
use \emph{Conviction} or \emph{Instinct} instead.

\begin{description}
	\item[Conscience] represents how sorry your character feels when they commit 
	a sin, based on the chart above.  If you truly repent your actions, represented 
	by winning a Virtue test, you will not lose a dot of Morality.
	\item[Self-Control] is the ability for you to avoid flying into an anger-fueled 
	frenzy when attacked, threatened, or in danger.  Someone with a high 
	\emph{Self-Control} is unlikely to fly off the handle easily.
	\item[Courage] dictates how easily your character responds to frightening or 
	potentially lethal situations.  The R\"{o}tschreck is a blind terror frenzy 
	that high levels of \emph{Courage} can prevent.
	\item[Conviction] represents the character's ability to steel their resolve and 
	swear that they will rise above their temporary failings.  It is a cold and 
	dispassionate trait that demands strict adherence to an internal moral compass.
	\item[Instinct] is very different from \emph{Self-Control} in that those with 
	\emph{Instinct} are in harmony with the Beast and often welcome its rampages.  
	Such characters cannot normally test to avoid frenzy, fully embracing the destruction.
\end{description}

\subsubsection{Mortals and Morality}
Vampires are unique in the World of Darkness in that they have to actively fight against 
a powerful internal force trying to drive them to depravity.  Humans, no matter how violent, 
do not have Virtue or Morality traits and are never at risk for frenzy or slipping further 
into the Beast's clutches.

Humans and most non-vampire supernaturals are never required to make Morality or Conscience 
tests.  Lacking the capacity to frenzy as kindred do, whenever they are faced with great stress, 
as would otherwise be represented by a Self-Control or Courage check, they are likely to react 
either with mortal rage or great fear, without the benefit of a challenge.

All mortals and their reactions to stimuli are under Storyteller control.

\subsection{Virtue Tests}
In certain situations your character's resolve may be tested, either when attempting 
to restrain the Beast or when you suffer a lapse of ethics and violate the tenets of 
your Morality path.  In all such cases a static Virtue test will determine the 
repercussions and may even change the direction of your character.

Normally only two retests are available for any Virtue test:  you may spend a Willpower 
for one retest and you may also spend a corresponding temporary Virtue trait itself to 
gain another.  If you fail the challenge after spending a Virtue trait for a retest, not 
only do you lose the trait permanently but you suffer additional penalties as described 
below.  If however falling to frenzy or R\"{o}tschreck would violate a character's Nature 
they may receive an additional retest once per night as their core ideals steel their 
resolve against the Beast.

You may never bid more traits in a Self-Control/Instinct or Courage test than you have current 
Blood in your system.  If your character is particularly low on Blood, represented by having 
two traits or fewer, the difficulty for these Virtue tests is increased by one as the need for 
sustenance whips the Beast into action.

\subsubsection{Conscience/Conviction}
When you commit a sin equivalent to those listed at or below your current Morality rating 
a Conscience/Conviction test determines whether or not you lose a permanent dot of Morality.  
The difficulty of this Static test is half the rating of the sin, rounded down to a minimum 
of one trait.  If you spend a Conscience or Conviction trait to retest and still lose, you 
additionally gain a permanent Negative trait associated with your sin, as chosen by a 
Storyteller.  Typically only one Morality test is made per scene, barring exceptional 
circumstances.

\subsubsection{Self-Control/Instinct}
When incited to frenzy a character may make a Self-Control test to restrain the Beast, 
with a difficulty determined by the type and severity of the stimulus.  Failing this 
challenge immediately places you into an anger frenzy as described on 
page~\pageref{subsec:frenzy}.  If you succeed you need not make further tests against 
that same stimulus for the rest of the scene, unless it intensifies in difficulty.

Those with Instinct are unable to resist frenzy unless they possess twice the stimulus 
difficulty in Instinct traits.  Instead they may make tests to guide or control their 
frenzy after it has begun, but their control only lasts for one turn per successful test.
Even if successful they cannot stop the destructive nature of the Beast, but they can choose 
targets for their rage, including inanimate objects.  This control also allows them to 
exert some measure of tactics and higher-thinking for the specific turn, such as not using 
Masquerade-breaking levels of Celerity.  All other benefits and drawbacks of frenzy apply.

Losing a Self-Control/Instinct test when you have spent a Virtue trait for a retest causes 
you to also gain the permanent Negative trait \emph{Callous} or \emph{Feral}, which may 
be bought off normally.

\begin{center}
\begin{tabular}{ | l l | }
	\hline
	\multicolumn{2}{ | c | }{\textbf{Self-Control/Instinct Difficulties}} \\
%	\textbf{Traits} & \textbf{Stimulus} \\
	\hline
	1 Trait & Smell of blood when hungry \\
	2 Traits & Sight of blood when hungry, life-threatening situation \\
	3 Traits & Physical provocation or attack, taste of blood when hungry \\
	4 Traits & Loved one in danger, humiliated \\
	5 Traits & Outright humiliation, mortal insults \\
	\hline
\end{tabular}
\end{center}

\subsubsection{Courage}
All characters regardless of Morality path possess Courage, the ability to stay the Beast 
in its efforts to flee a potentially dangerous situation such as fire or the sun.  While 
you may not be required to test against fire that is under your control, someone 
brandishing fire at you will almost always require a test.  Failure means you fall to 
R\"{o}tschreck, a blind panic described on page~\pageref{subsec:frenzy}.  Success allows 
you to withstand the stimulus for the next ten minutes unless the difficulty increases, which 
necessitates a new challenge.

Spending a Courage for a retest and still losing also grants you the permanent Negative 
trait \emph{Cowardly} or \emph{Submissive} which can be bought off with experience.

\begin{center}
\begin{tabular}{ | l l | }
	\hline
	\multicolumn{2}{ | c | }{\textbf{Courage Difficulties}} \\
%	\textbf{Traits} & \textbf{Stimulus} \\
	\hline
	1 Trait & Being bullied, a flicked lighter, sunrise \\
	2 Traits & Brandished torch, obscured sunlight \\
	3 Traits & Bonfire, uncovered window during daylight \\
	4 Traits & House fire, being burned \\
	5 Traits & Trapped in a burning building, direct sunlight \\
	\hline
\end{tabular}
\end{center}
	