\section{Health and Healing}
\label{sec:health}
Vampires do not require food, water, or air, and will never die of old age.  The two feelings 
that drive every kindred are the desire to feed, and the desire to survive.  Inflicted damage 
notwithstanding, a kindred will return to state in which they were embraced every evening, 
including hair 	length and color, scars and tattoos, even perhaps wounds suffered before they 
died.  That said, though vampires are difficult to kill they aren't immune to damage and cannot 
heal naturally as other creatures may -- only through the expenditure of blood will undead flesh 
knit itself back together.

Each character begins play with nine empty health levels which are 
crossed off as they take damage.  If every box is filled the character 
is in very real danger of meeting the Final Death from which there is 
no return.  These health levels are divided into categories, with 
increasing penalties for being damaged as detailed below.  Barring 
specific Disciplines, Merits, or Flaws, all vampire characters have 
the same number of health levels when they begin play.

Each damage category is cumulative; \emph{Wounded} characters also suffer 
the penalties for being \emph{Bruised} and so forth.

\begin{description}
	\item[Healthy:] 2 boxes.  Suffering no penalties, if you are 
	damaged at all it is likely just cosmetic, mainly bumps and bruises.
	\item[Bruised:] 3 boxes.  Starting to feel the pain with broken bones 
	and other injuries, you are down one trait on all tie comparisons.
	\item[Wounded:] 2 boxes.  Likely suffering from broken bones 
	and other mass trauma, engaging in any challenge is a strenuous 
	affair requiring that you bid an additional trait.  You also automatically 
	lose all tie comparisons.  If you are using a power that allows you to win 
	all ties, compare them as normal instead.
	\item[Incapacitated:] 1 box.  Knocked near to unconsciousness, 
	you are completely helpless and unmoving for ten minutes.  After 
	that time you remain almost entirely immobilized save from the 
	occasional hoarse whisper.  Mortals falling to Incapacitation 
	require swift medical attention or face certain death.
	\item[Torpor:] 1 box.  A death-like sleep, you are both unable 
	to perceive the world around you or act in any way.  The length 
	of your torpor is directly related to your Morality rating 
	(see page~\pageref{sec:health}).  Mortals do not possess this 
	wound level and are instead killed immediately.
\end{description}

\begin{center}
\begin{tabular}{ | c | c | c | c | c | }
	\hline
	\multicolumn{5}{ | c | }{\textbf{The Standard Vampire Health Track}} \\
	\hline
	Healthy & Bruised & Wounded & Incap. & Torpor \\
	\large{\ding{111}\ding{111}} & \large{\ding{111}\ding{111}\ding{111}} & 
	\large{\ding{111}\ding{111}} & \large{\ding{111}} & \large{\ding{111}} \\
	\hline
\end{tabular}
\end{center}

\subsection{Types of Damage}
\begin{description}
	\item[Bashing Damage] is the least likely to seriously injure a 
	kindred, normally inflicted by punches, kicks, and like attacks.  
	So rugged are they that vampires halve all Bashing damage 
	received, rounding up, to a minimum of one box.
	\item[Lethal Damage] represents wounds that are more serious or 
	life-threatening, such as from bladed weapons or blows fueled 
	with supernatural powers like \emph{Potence}.  Lethal wounds 
	are likely to cripple mortals in short order.
	\item[Aggravated Damage] is terrifying even to the undead for 
	these wounds take a great deal of effort and Blood to heal, 
	and very likely can result in their Final Death.  A vampire 
	suffers Aggravated wounds from fire, sunlight, the fangs of 
	another kindred, and some supernatural powers such as 
	\emph{Wolf's Claws}. \\
\end{description}

\noindent If your character suffers injury, make a note on your character sheet 
for the wound(s) suffered---typically a `B' for Bashing, `L' for Lethal, 
and an `X' for Aggravated damage.  If a character receives multiple types of 
damage before healing, the more severe level goes above the rest, 
``pushing'' existing wounds down the health track.  For example if a 
character suffered two Bashing and then took a Lethal, their wounds 
would be recorded as \framebox{L}~\framebox{B}~\framebox{B}, filling their 
Healthy levels and the first box of Bruised.  If they are unfortunate enough 
to also take an Aggravated wound their record would show 
\framebox{X}~\framebox{L}~\framebox{B}~\framebox{B}.

All weapon item cards (see Chapter~\ref{sec:items}) include the type of damage they deal, 
typically Bashing or Lethal.  Disciplines and other damaging effects are described in their 
respective chapters.  Atmospheric damage, such as from a collapsing building or falling from 
a great height, is determined by an on-scene Storyteller, but certain guidelines are as follows:  \\

\begin{itemize}
	\item Characters on fire will suffer between 1 and 3 Aggravated wounds per combat turn
	\item Vampires falling less than 50 feet suffer 1 Bashing wound per 10', which may be reduced 
	with an \emph{Athletics} test at Storyteller discretion.
	\item Those falling greater than 50 feet suffer 1 Lethal wound per 10'
	\item Daylight, even through thick curtains, will deal 1 Aggravated wound per turn, while 
	direct exposure can yield 3 or even more.
	\item Lost limbs may be regrown as if they were Aggravated wounds, at Storyteller discretion.
\end{itemize}

\subsection{Torpor and Death}
A character having reached torpor is in great risk of dying, but the kindred 
body is hardy and resilient.  Bashing wounds taken while in torpor upgrade 
existing Bashing wounds to Lethal.  Once the health track is filled 
with Aggravated or Lethal wounds, Bashing damage has no more effect.  
If a character declares a ``killing blow'' and strikes a torpored 
kindred with Lethal damage, the victim will die.  Any Aggravated wounds 
suffered while a character has been beaten into torpor automatically kill 
the victim.

Torpor isn't always the result of a fight gone awry however.  A kindred desiring 
to sleep away the years may voluntarily enter torpor, falling to the same 
lifeless sleep as if they had been grievously wounded.  However unlike those 
having been beaten into torpor a kindred who entered this state intentionally 
only needs to spend half the time required by their Morality rating, as listed on 
page~\pageref{sec:health}.  After the requisite time has passed the character 
may spend a blood and make a Static Mental challenge against four traits to wake, 
which may be repeated once an evening until successful.  If a character is wholly 
out of blood they cannot rise from torpor until someone provides \emph{vitate}.

Torpid characters are largely similar to inanimate objects; they have no aura, 
cannot engage in challenges, and appear to all the world as a corpse.  Their bodies 
are not empty however and cannot be possessed by ghosts or other spectral entities.  
Time spent in torpor does not contribute toward the \emph{Age} Background.

A kindred may also wake from torpor if they are fed Blood from another 
vampire that is at least three Generations more potent than their own; 
the body automatically heals the Torpor wound level.  There is also 
rumored to exist rare Disciplines which can force a torpid kindred to 
wake, but these may be just that---rumors.

When a kindred dies, its body is no longer sustained by the Beast and 
quickly reverts to its natural state as if it had never been granted 
extra time---truly aged vampires may turn to fine ash while younger 
kindred decay rapidly, softening and often liquefying in a disgusting 
display of natural processes.  Should occasion arise where hair, limbs, 
or even \emph{vitae} is removed from an active kindred the same process 
occurs, usually within the same scene or hour.

\subsection{Healing Damage}
As Blood maintains a kindred's night to night existence so too does it 
have the capacity to heal any wounds suffered.  For each Blood trait 
spent two Bashing wounds may be healed, or one Lethal.  Damage is healed 
from right to left, meaning Bashing is always healed before Lethal and 
Lethal before Aggravated.

Aggravated wounds however are not so easily cured, requiring a full 
day's unbroken rest, three Blood, and a Willpower for each wound.  
There is no way to hasten recovery from such grievous injuries, and most 
kindred wisely avoid situations in which they could be so seriously 
damaged.

Humans are much more fragile than kindred and can only recover naturally and 
without assistance from the most topical of wounds.  If a human ever becomes 
Bruised with Lethal damage or Wounded with Bashing damage they will require 
hospitalization to heal.  Any human who suffers Lethal damage in their 
Wounded health boxes or enough Bashing to Incapacitate them requires immediate 
hospitalization or faces death.  Humans who receive any amount of Aggravated damage 
must start receiving care within the same scene or risk severe and permanent problems 
at Storyteller discretion.  Ghouls are considered mortals for the purpose of damage, 
though of course they can heal Lethal and Bashing wounds as vampires can, provided 
they have enough \emph{vitae} in their system (see Chapter~\ref{sec:additional}).

\subsection{Being Staked}
An innate aspect of the kindred condition, the reasons for which have been debated 
from time immemorial, being staked is terrifying and debilitating experience 
for any kindred regardless of age or experience.  If a vampire's heart is 
pierced by a shaft of wood (see the specific rules in Chapter~\ref{sec:combat}, 
Combat) they instantly lose any and all control they have over their bodies.  
Appearing to any observer to be in torpor, unable even to move an eyelid, the 
true terror of being staked is that the kindred is fully conscious of everything 
going on around them yet remains powerless to do affect it.

While staked a kindred can see, hear, and feel like normal, and can later recall 
the events surrounding their staking, but are completely incapable of speech, 
Discipline use (including purely mental powers such as \emph{Auspex}), spending 
Blood or Willpower, or movement of any kind.  Their mind may be awake and fully 
conscious but they have no ability to interact with the world.  Also, unlike torpor, 
a staked kindred must still spend Blood every evening when they would normally wake.  
A staked kindred may however enter into voluntary torpor as described above.

In short, a character who is staked is in very real trouble.  While they are able to 
bid Stamina-based traits to avoid further damage, and resist Mental or Social powers 
as normal, they are unable to activate any Disciplines or use Physical or Social 
Abilities to help them do so; they are at the absolute mercy of their captors.

The only way to recover from being staked is for another character or environmental 
circumstance to remove the wood from the heart.
