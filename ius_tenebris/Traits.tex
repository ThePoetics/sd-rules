\section{Traits}
\label{sec:traits}
Also called ``attributes'' Traits are divided into three 
categories---Physical, Mental and Social---and are adjectives that serve two 
purposes:  helping you role-play your character and providing a mechanical 
benefit when it comes to throwing challenges.

Each primary category is split into three sub-groups of related traits.  Care 
should be taken when throwing a challenge to use the appropriate trait; 
it is unlikely that someone could be \emph{Wiry} enough to fire a gun---such 
would be the province of a Dexterity-based trait instead.

Use these traits not only when building your character to shape how you 
want them to act, think, and appear, but also while playing the 
character as a reminder of how you should present yourself.  Someone who 
is very \emph{Patient} isn't likely to fly off the handle often, much as 
someone who is \emph{Graceful} won't be tripping over their own feet.

If a trait is lost as the result of a challenge or other circumstance it is 
unavailable for claiming in future challenges that evening, unless it is 
refreshed through methods addressed in other sections of this guide.

The maximum number of traits in your primary category is determined by the 
Generation Background (see page~\pageref{bg:generation}), with the secondary 
and tertiary category caps descending by two each (e.g. a 13th generation 
kindred's primary cap would be 10, secondary 8, and tertiary 6).

Only the following listed positive traits will be permitted for Camarilla 
kindred, and each are rated on the character sheet from zero to five.

\subsection{Positive Traits}
\subsubsection{Physical Traits}
Representing how hale and hearty your character is, physical traits 
measure a character's ability to deal, avoid, and absorb damage and 
physical punishment where required.  Of all trait categories, the 
greatest care must be chosen when entering into a Physical challenge 
to bid the correct type of trait.

Strength traits are used when relying on raw physical prowess, usually 
to deal damage to an opponent.  Challenges with strength are most often 
retested with the \emph{Melee} or \emph{Brawl} Abilities, and can be 
enhanced through use of the Potence Discipline.

Dexterity traits can be used for offensive purposes but are usually used 
to avoid damage, such as when evading incoming attacks.  The \emph{Dodge} 
Ability is most often used as a retest when used in such a manner.  
Offensively Dexterity traits are often used when firing a gun or striking 
with speed instead of power.

Stamina represents the unyielding and immortal aspect of kindred nature.  
Used almost exclusively to withstand physical damage, challenges of stamina are 
retested using the \emph{Survival} Ability.  And, unlike the above, can often 
be used as a defense even when caught unawares or surprised.

\begin{description}
	\item[Strength-Based:]  Athletic, Brawny, Brutal, Ferocious, Fierce, 
	Vigorous, Wiry
	\item[Dexterity-Based:]  Agile, Dexterous, Graceful, Lithe, Nimble, 
	Quick, Steady
	\item[Stamina-Based:]  Enduring, Energetic, Resilient, Rugged, 
	Stalwart, Tenacious, Tireless
\end{description}

\subsubsection{Social Traits}
The Camarilla places a high regard on those who can move through its social 
waters with grace and dignity, touting itself as a polite and genteel 
organization.  Social traits represent your character's ability to 
outplay others at the social games so popular in court.  If a character fails a 
Social challenge they initiated (where both aggressor and defender are bidding 
Social traits), they cannot issue the same challenge to the same character for 
a period of five minutes.

Charisma traits detail your pleasant conversational attributes, and how 
likely you are to be enjoyable company.  A society taking for its ideals 
a strict social order, these traits show that you can socialize with the 
best of them.  Often tests involving Charisma are retested with the 
\emph{Leadership} Ability.

Manipulation traits represent your ability to get your way, often without 
others realizing it.  Tricking, coercing, and sometimes demanding that 
your needs be tended to are valuable tools in any successful kindred's 
toolkit.  Challenges of Manipulation are usually retested with \emph{Subterfuge}.

Appearance traits reflect the physical or social attractiveness your 
character may exude.  Perhaps they are beautiful or just have an aura 
about them that enchants others.  Appearance-based traits are often used 
in seduction-type challenges where you endeavor to appeal to a target's more 
base nature.  \emph{Subterfuge} or \emph{Leadership} are also often the 
appropriate Abilities with which to retest these challenges.

\begin{description}
	\item[Charisma-Based:]  Charismatic, Charming, Compassionate, 
	Expressive, Friendly, Genial, Witty
	\item[Manipulation-Based:]  Beguiling, Commanding, Diplomatic, Eloquent, 
	Empathetic, Ingratiating, Persuasive
	\item[Appearance-Based:]  Alluring, Dignified, Elegant, Gorgeous, 
	Intimidating, Magnetic, Seductive
\end{description}

\subsubsection{Mental Traits}
Mental acuity is essential for a kindred's long-term survival.  The 
ability to see danger well before its arrival is a skill too few 
cultivate before it is too late.  These traits provide a look into the 
mindset and methods of your character.  If a character fails a Mental 
challenge they initiated (where both aggressor and defender are bidding 
Mental traits), they cannot issue the same challenge to the same character 
for a period of five minutes.

Perception traits detail how aware a character is of their surroundings; 
can they piece together those subtle clues that tell the story of what 
hangs in the air?  These traits are most often used with the 
\emph{Investigation} Ability.

Intelligence traits provide a look into the ability for a character to 
process information and recall facts.  A kindred with many Intelligence 
traits is unlikely to be caught ill-equipped for a situation requiring 
brainpower.  Retests include the \emph{Academics}, \emph{Science}, and 
\emph{Lore} Abilities.

Wits traits represent the on-your-feet quick decision-making ability 
common to the sharp-eyed among kindred society.  Knowing the right moment 
to strike is often more important than the strike itself, and characters 
with a great deal of wits will know the moment and the method both.  
Retests often include the \emph{Awareness} ability.

\begin{description}
	\item[Perception-Based:]  Alert, Attentive, Dedicated, Discerning, 
	Insightful, Observant, Vigilant
	\item[Intelligence-Based:]  Creative, Disciplined, Intuitive, 
	Knowledgeable, Rational, Reflective, Wise
	\item[Wits-Based:]  Calm, Clever, Cunning, Determined, Patient, Shrewd, 
	Wily
\end{description}

\subsection{Negative Traits}
Just as positive traits represent desirable qualities of your character, 
Negative traits show those undesirable, antisocial, and unseemly quirks 
kindred hope nobody else notices.  Just as you endeavor to role-play your 
positive traits, make sure to show your Negative traits as well, for they are 
just as much a part of your character and can often lead to great story 
when displayed or exploited.  Unlike Positive traits which can be lost or spent during play, 
the only way to remove Negative traits is by spending XP (see Chapter~\ref{sec:xp}).

If taken during character creation each Negative trait, with some limitations, 
grants one Free Trait.  Each Negative Trait is normally rated from zero to three.  Some 
Disciplines or situations may grant a character Negative Traits.  Normally the same power 
cannot be used on a single character multiple times for greater effect, though the effects of 
different powers do stack.  When in doubt see a Storyteller.

Special rules for Negative traits in Challenges can be found in Chapter~\ref{sec:challenges}.  

\begin{description}
	\item[Physical Negatives:]  Clumsy, Cowardly, Decrepit, Delicate, Docile, 
	Flabby, Lame, Lethargic, Puny, Sickly
	\item[Social Negatives:]  Bestial, Callous, Condescending, Dull, Feral, 
	Naive, Obnoxious, Repugnant, Shy, Tactless, Untrustworthy
	\item[Mental Negatives:]  Forgetful, Gullible, Ignorant, Impatient, 
	Oblivious, Predictable, Shortsighted, Submissive, Violent, Witless
\end{description}