\section{Combat}
\label{sec:combat}
Sometimes kindred don't get along, and political or social remedies don't seem sufficient.  
\emph{Sanguine Dreams} is a low-combat game, emphatically encouraging situations and 
disagreements to be role-played through instead of devolving into challenges, but combat is 
a very real and important part of the system.

Challenges tend to make emotions run high, Physical challenges more than most; it is always 
suggested that a Storyteller be on-hand, with this rule book, to mediate and keep tempers in 
check.  As mentioned in both the Character Creation and Challenges chapters (\ref{sec:creation} 
and \ref{sec:challenges} respectively), all characters should have an up-to-date and 
Storyteller-signed copy of their character sheet on-hand when engaging in combat, to ensure 
there is no dispute about what a character can and cannot do during the combat turn.  Similarly 
no player should move away from their character's current position to make sure there is no confusion 
as to where each character is standing.

Since combat often takes much more time to resolve than it would take in-game, it is encouraged that 
before the first combat turn begins all involved players wait thirty seconds for others to walk through 
or otherwise interrupt the scene; if a character does not enter the scene in that time, it is exceedingly 
unlikely that they would have encountered the fray in time to affect it.

\subsection{The Combat Turn}
The combat scene is divided into turns, each representing a span of time between three and six seconds 
long.  Each turn can be further divided into several rounds, if characters have the ability to act multiple 
times per turn, such as by activating Celerity.  

Each character is limited to one single action per round, such as drawing a weapon, using that weapon, 
opening a door, or issuing a challenge to another player.  Each subsection of this chapter addresses 
additional rules or opportunities available for combatants, and should be read in full for a greater 
understanding of this complex system.

Typically combat begins with the issuance of a Physical challenge or a visible threat, and each turn 
progresses as follows: \\

\HRule

\begin{description}
	\item[\emph{Pre-Rounds Actions}] \hfill \\
	Reflexive actions such as spending Blood or turning on \emph{Heightened Senses} can be activated. 
	These actions take no time and may be done at any time, but customarily they are at the start of 
	the combat turn, before formal rounds begin. \\
	\item[Everyman Round] \hfill \\
	All characters may issue Physical, Social, or Mental challenges during this round or interact with 
	their environment, barring injury or other unique situations.  Short verbal phrases may also be spoken 
	or uttered.
	\item[Swiftness Round] \hfill \\
	Only available to characters who have activated the Celerity power of \emph{Swiftness}, characters 
	acting in this round may issue Physical challenges or interact with their environment, but the use 
	of speech or either Mental or Social powers is prohibited.  Characters unable to act in this round 
	may only defend against Physical challenges using Stamina traits (and thus the \emph{Survival} retest), 
	as they cannot move quickly enough to dodge incoming attacks.  Speech issued during this round is 
	unintelligible, even to others moving in Celerity.
	\item[Legerity Round] \hfill \\
	Only available to characters who have activated the Celerity power of \emph{Legerity}, this round 
	is otherwise indistinguishable from the Swiftness round. \\
	\item[\emph{Clean-up Phase}] \hfill \\
	At this phase transformative powers and uses of \emph{Vanish from the Mind's Eye} activate, and 
	the on-scene Storyteller should confirm whether or not combat is continuing.  If not, the combat 
	scene ends and a new scene begins.
\end{description}
	
\HRule \\

\noindent As the individual turns and rounds are called out by the Storyteller it is customary to show your intention 
to act against another character by pointing at them until the Storyteller addresses the relevant challenges.  
They will make sure everyone has declared their actions before any challenges are thrown so as to not 
miss anyone.  If no characters are acting within a particular follow-up round it can be skipped.

A character who, during the Everyman round only, issues a Social or Mental challenge is not normally denied 
his chance to defend against incoming Physical challenges; they may however only attempt to resist damage 
by way of Stamina traits, and thus the \emph{Survival} retest.  Similarly, the target of this Mental or 
Social challenge does not give up any desired attack in order to defend.

The Celerity power of \emph{Alacrity} allows a character to take one action outside of normal combat 
rounds; this action cannot be used to attack another character, as described in its writeup in 
Chapter~\ref{sec:disciplines}, but does permit one to interact with the environment.  This action may 
be taken before or after any of the above rounds, but it must be declared and its use completed before 
characters announce their intentions for the round.

\subsection{Initiative}	
In most cases actions resolve simultaneously, particularly with Physical challenges where 
two or more characters' actions are to attack one another.  Occasionally an order of actions 
must be determined however, such as when someone is trying to \emph{Dread Gaze} an assailant 
who is intent on staking them; very rare are the situations where a Social or Mental challenge 
is retested with a Physical Ability, and vice versa.

When initiative must be determined, each challenge participant declares how many Named 
Traits they claim in the category relevant to their action, which may be less than their 
actual total.  Named Traits are adjectives such as \emph{Wiry} or \emph{Knowledgeable} 
and do not include Bonus Traits from weapons, Status (in Social challenges), or any other 
addition not indicated by a proper name.  All applicable penalties however do count 
against that total, such as from Flaws (Negative Status does not apply to Initiative).  Traits 
you have expended or lost due to other challenges or situations also do not contribute to this 
total---rely on your current traits only.

The character with a higher Named Trait total has their action occur first, with all 
proper challenges resolved as normal.  In the above example should the Social character go 
first, he would have an opportunity to \emph{Dread Gaze} his opponent before they had a 
chance to stake him, which would be a normal Contested Social challenge.  If unsuccessful, 
the assailant would then immediately take the action of trying to stake him, which would be 
resisted physically as normal.  It is suggested that all physical actions be resolved at once 
to prevent a back-and-forth series of attack and defend challenges.

Initiative need not be used if the combatants desire a more speedy conflict resolution.  
In that case all challenges resolve simultaneously, regardless of challenge type; this could 
well result in a \emph{Dominate} challenge being cross-aggressed by someone throwing a punch.  
For each combatant the trait bid would determine the type of challenge in which they 
were involved---for one a Mental challenge, the other Physical, both considered the aggressor.

\subsection{Movement}
While characters outside of the combat turn may move at their regular pace, specific Flaws 
or Negative Traits notwithstanding, movement within the combat turn is more restrictive.  During 
each round all characters who are eligible to act may move, incurring the following penalties if 
they do so.  All movement needs to be declared before any challenges are thrown:

\begin{description}
	\item[1 Step]--- regular walking pace, there is no penalty for actions taken while 
	moving at this safe speed.
	\item[2 Steps]--- considered a jog, suffer a -1 trait penalty on ties for all 
	challenges.  This speed breaks the Obfuscate power \emph{Unseen Presence}.
	\item[3 Steps]--- running quickly, any actions taken suffer a -2 trait penalty.
	\item[4-6 Steps]--- a dead sprint precludes any other action this round.  The 
	character automatically relents to all challengers whose victory conditions are damage.  
	Challenges made to restrain or knock down the sprinting character may be resisted, but 
	so intense is their headlong rush that the defender may only claim half their normal traits, 
	rounded down.
\end{description}

\noindent Remember that it is impossible for a normal human to move more than 6 steps in a single 
combat turn, and then only at a full sprint.  Characters who are seen moving more quickly than this 
risk breaking the Masquerade if they do not take action to correct the situation.

\subsection{Special Attacks}
Not every Physical challenge is as simple as ``I hit you with my fist;'' some attacks result in a 
victim being tripped or grappled, staked or merely touched on the arm.  This section covers all 
special attacks where additional rules may be required.

\subsubsection{Victory conditions}
Damage isn't always the desired outcome for Physical challenges.  Perhaps a character wants to 
scale a wall, knock someone down, or creep past a guard unnoticed.  These are all examples of 
``Victory Conditions,'' and each challenger in a test can only have one.  You cannot both do 
damage and pin the victim to the ground, for example; you must choose one or the other for each 
Physical challenge.

If your desired outcome is not damage, you must declare what your victory condition is before the 
challenge is thrown so everyone is clear of your intentions and what retests will be applicable.

\subsubsection{Staking}
Often the most efficient way to immobilize a kindred, melee weapons with the ``Staking'' quality 
(see Chapter~\ref{sec:items}) may be used to pierce a character's heart.  Small shafts of wood like 
pencils or thin dowels are not sufficient to stake a vampire, though a broken chair leg may be, at 
Storyteller discretion.

If a melee attack with an applicable weapon succeeds in doing damage to the target, not reduced to 
zero by armour or other protective measures, damage is applied as normal and the attacker then engages 
in two Simple Tests.  If both tests succeed with either a win or tie, the victim is staked, unable to 
move or act in any way, including activating Disciplines or spending Blood.  More information about 
the condition of being staked can be found in Chapter~\ref{sec:health}.

Should the attacker fail either of the follow-up Simple Tests, combat proceeds as normal, with a 
new challenge (and two Simple Tests) required to stake the victim again; it is generally assumed that 
the attacker will retain the weapon in-hand.

Successfully staking a mortal will result in their instant and gruesome death.

\subsubsection{Grappling}
At times a character wishes merely to restrain another character instead of dealing damage.  Engaging 
in a Physical challenge with the victory condition of ``grappling'' can accomplish this, always retested 
with the \emph{Brawl} Ability.  If successful, the actions of both parties are greatly limited---no longer 
can they take steps, and physical attacks are limited to biting or brawling strikes against their opponent.  
Barring exceptional circumstances, individuals may only grapple or be grappled by one other person at a time.  

During each round the victim of a grapple may endeavor to break free with either a Strength- or Dexterity-based 
Physical challenge as you try to maintain the grapple.  If they are able to act in rounds that you are not, 
such as due to higher levels of Celerity, you may still resist their attempts to escape with a Strength-based 
challenge as you hold them tightly, in a singular exemption to the rule that you cannot be the aggressor 
in combat rounds in which you cannot normally act.

\subsubsection{Biting/Feeding}
Once engaged in a grapple, a vampire may endeavor to bite their opponent in a subsequent combat round, either 
for damage or to drink their blood, both of which require a new Physical challenge.  Most often a vampire's 
fangs elongate at the moment of the strike, whether the intent is to drink or deal damage.  When used to deal 
damage in this way a vampire's supernatural fangs cause an Aggravated wound.  See Chapter~\ref{sec:additional} 
about the effects of feeding on victim and attacker alike.

\subsubsection{Touch-Based Powers}
No power may affect another character without a challenge.  In the case of touch-based powers such 
as \emph{Quell the Beast} or \emph{Dagon's Call}, a firm grip or contact must be made for the power 
to succeed.  Such is accomplished with a Strength- or Dexterity-based Physical challenge.  The 
defender's only recourse to avoid the touch is a Dexterity-based dodge, if they are not already 
engaged in an aggressive Physical action.  The Dexterity test represents them twisting or moving so as 
to prevent the necessary full contact, even if they are unaware of the reason for the grip.  Attempting 
to resist the contact with a Stamina trait can not succeed because it does not prevent the physical 
connection.

Even if a handshake or other willing contact is offered, it is not sufficient for a power's use; a 
challenge must be initiated, though the handshake is a great way to masquerade one's true intentions.

\subsubsection{Two-Weapon Combat}
While it is possible to wield weapons in both hands (normally suffering a -2 penalty for challenges 
using the off-hand item), attacking with two weapons at once is not possible.  At the beginning of a 
challenge you must declare which weapon, if any, you are using for the attack.  You may only apply 
the rules governing that one weapon to the challenge; from a mechanical standpoint the other weapon 
does not exist.

\subsubsection{Called Shots}
By and large ``called shots'' directed at particular parts of an enemy's anatomy are not allowed in this 
system.  If someone has an item card for armour, that protection is assumed to protect against all 
relevant attacks.  Victory conditions such as ``disarm'' or ``knockdown'' are absolutely valid, but 
declaring that you are shooting your opponent in the face in order to ignore armour or inflict two 
victory conditions is not.

A Storyteller may, at their discretion, impose a trait bonus (to the defender) or penalty (to the 
attacker) in cases where environmental factors such as cover or concealment make getting a clear shot 
difficult.

\subsection{Fair Escape}
If a character is able to escape a dangerous situation without being impeded, they may declare 
``Fair Escape'' and retire from the scene, unable to return or interact further with it.  Fair 
Escape may only be declared if the character is not under immediate threat and has a clear means 
of escape; those in the middle of martial combat cannot simply walk away unscathed.  Often used 
by bystanders who want to leave the scene without the burden of combat's specific movement rules, 
those who have successfully evaded their pursuers may also claim it, at Storyteller discretion.  
Some conditions which may mitigate the ability to declare Fair Escape include:

\begin{itemize}
	\item \textbf{Ranged Weapons:}  A character may still be affected by damaging ranged attacks 
	even if there are no assailants in melee range.  In this case the Storyteller may rule that 
	the escaping character may not flee the scene until they are no longer under threat of being 
	shot.
	\item \textbf{Lack of Exits:}  Being sealed or caught inside a room with no unblocked exits 
	will preclude any character from escaping the scene unless they are allowed to leave.
	\item \textbf{Pursuit:}  A character wishing to escape someone chasing them must either 
	out-run or find a way to hide before they can claim Fair Escape.  Barring specific Flaws or 
	Negative Traits all characters have the same movement options during combat, and so chases 
	should be arbitrated by Storytellers.  In the case of Celerity use, it is generally assumed that 
	a character with higher levels of Celerity, and both a willingness and the Blood to power it, 
	will either overtake or escape someone with less Celerity.
\end{itemize}

\subsection{Special Circumstances}
Not every physical altercation is created equal.  Sometimes one or more characters will have the 
drop on their victim, or terrain or other situational modifiers may make attacking or defending 
more difficult.  The Storyteller on scene is encouraged to give reasonable penalties or benefits 
to characters who are fighting in uneven circumstances.  For example, a combatant who declares that 
they are not looking into an opponent's eyes (to avoid being subject to Dominate or \emph{Quell the 
Beast}, for instance) may find themselves at a -2 penalty as they have a harder time predicting their 
opponents' actions.

\subsubsection{Mob Combat}
\label{subsec:mobcombat}
Any time multiple challengers desire to affect one target, or when circumstances allow one aggressor 
to affect multiple defenders, challenges should be resolved as a single massive challenge, collectively 
called ``mob combat.''  While the rules for mob combat follow the regular challenge progression in most 
respects, there are specific differences which are presented here.  This section describes a single 
defender against multiple attackers, but the same rules apply for one aggressor versus multiple defenders 
(such as in the case of Obfuscate or weapons with the \emph{Spray} quality).

A single defender may be subjected to a limited number of hostile challenges per round.  During 
the Everyman each character need only defend against the first hostile Social and Mental tests 
levied against them, ignoring future challenges of the same type.  In the case of multiple like 
challenges, use the rules for Initiative to determine which challenge takes precedence.  Physical 
challenges, available in any round, are limited to a maximum of five, provided the subject is fully 
surrounded.  Storytellers may rule that less than optimal circumstances may reduce the number of 
viable attacks against them.

When attacking as part of a mob challenge, use the rules for a standard challenge as normal; 
bidding traits, retesting or canceling if desired, and either succeeding or failing as usual.  
Mob combat differs from regular contests in the mechanics used by the defender.  The defender 
chooses one target for their action, whether offensive or defensive, and only loses traits if 
they lose against that one target, regardless of other assailant's victories, and only after the 
whole challenge has been completed.  A defender wishing to retest against his attackers need only 
spend one Ability relative to their chosen action in order to retest against everyone.  If an 
attacker cancels the retest it is canceled against them only, and the challenge may proceed as 
usual against all others.  If the defender wishes to cancel he must do so individually, expending 
one Ability for every canceled challenge.

In a mob combat challenge all tests may be thrown at once or individually as determined by the 
defender.  If a defender won the challenge against his chosen foe he would not lose any traits, 
regardless of how many other attackers bested him, though each of their victory conditions would 
take effect.

Storytellers may rule that ranged attacks must be resisted by a defensive challenge instead of 
contributing to a mob, to prevent the occasion where the defender retests a Firearms test with 
Brawl or similar.

NPC assistants involved in mob combat, including those created with \emph{Splinter Servant} or 
\emph{Arms of the Abyss}, do not have individual victory conditions or engage in independent 
challenges and instead grant their controller bonus traits, up to +4, respecting the limits of 
five attackers for physical challenges in mob combat.

\subsubsection{Surprise}
Not every character can perfectly predict when and where they will be attacked.  Sometimes a 
close ally will turn unexpectedly, a nearby kindred may fall to Frenzy, or they walk into a 
well-prepared ambush.  In such cases they are considered surprised, which gives their attackers 
a distinct advantage.

When a would-be aggressor proclaims a Physical challenge (e.g. ``I am \emph{Brawny} enough to 
punch you'') and either the victim's player is too surprised to respond within a reasonable time-frame 
or when they decide that their character is suitably surprised, the following rules take effect.  
\emph{Sanguine Dreams} expects its players to role-play their characters true to form, and sometimes 
that means being the victim and being caught unawares.  There is no situation which guarantees a 
victim is surprised, including attacking from behind or immediately from Obfuscate.  Any question as 
to whether or not a character is surprised should be directed toward a Storyteller who will assess the 
situation.  

The surprising character's benefit is that they may take one action before the start of normal 
combat turns.  Their (usually) Physical challenge is issued, to which the defender may only resist 
the damage through Stamina traits and the \emph{Survival} Ability---\emph{Dodge} is not applicable 
since the victim is unaware of the attack.  In this way surprise all but guarantees that touch-based 
attacks succeed.

Immediately following this single action, the regular combat turn will begin as outlined above, if 
necessary.  Surprise does not grant any special retests, bonus traits, or any unique ability to 
shape an encounter---the singular benefit is a preemptive strike before their opponent can react.

\subsubsection{Fighting in Darkness}
Very few kindred are well-equipped to handle areas of true darkness, rare in these modern nights.  
Normally found underground or in secluded rooms, all challenges taken within an area of severe to 
absolute darkness suffer the following penalties:  -2 traits on tie resolution, the automatic 
failure of all powers requiring sight, and any successful challenge may be retested due to the 
situational retest \emph{Darkness}.  Typically when all contestants are equally disadvantaged 
all trait and retest penalties are waived to ease gameplay.

There are however powers and Abilities that may offset the situational penalties---\emph{Heightened 
Senses}, \emph{Tongue of the Asp}, and \emph{Eyes of the Beast} may all be used to lower the trait 
penalty by one; if a character activates several of these powers, they can reduce the penalty to zero 
but never gain a trait bonus.  A character who has reduced their trait penalties to zero may be 
considered to have line of sight on others when required for Discipline use or ranged weapons.

The only way to counter the Darkness situational retest is the Sabbat-only ability \emph{Blindfighting} 
which not only cancels the automatic retest but may also be used in mob combat, a single expenditure 
canceling all such retests. 

\subsection{Diablerie}
The worst sin in the Camarilla, diablerie is an act even more disgusting than the murder of 
another kindred.  Through the process of diablerie a vampire not only destroys the body of 
their victim but drains their soul as well, empowering their own abilities at the expense of 
the spirit's eternal rest.  What's worse, the rush of power committing such a deplorable act 
brings is as far different from the taste of kindred \emph{vitae} as \emph{vitae} is from 
animal blood, making it unquestionably addicting.

Rarely does diablerie happen by accident---even when in the midst of the most violent frenzy 
kindred will stop far before absorbing the soul of another.  Diablerie is a willing act that 
scars a kindred's psyche with its evil for all to see.

\subsubsection{Engaging in the Amaranth}
The process of diablerie starts by incapacitating the target, either by staking or beating 
them until they fall torpid.  Once immobilized the would-be diablerist begins draining the 
victim of all their blood, typically by drinking.  At any point so far the attacker can 
decide to stop the process, through drinking the last point of blood, without challenge or 
drawback.

Once all the blood has been drawn however, the diablerist continues drinking, absorbing 
the very essence of their victim's health.  The attacker engages in a Physical challenge 
with his target for each of the victim's remaining health levels, if they aren't already in 
torpor.  The victim can only bid Stamina traits and the attacker Strength traits.  Neither 
side may retest, but the challenge may be repeated for a given health level until one side 
has run out of traits and cannot continue.

Once the victim is bloodless and in torpor, either from trauma inflicted while being 
immobilized or through his attacker's feeding, the final test begins.  One last set of 
Physical challenges decides whether the diablerie succeeds, using the same rules as for 
draining health levels, save that the defender is up three traits.  The defender need not bid 
a trait to resist this challenge as his soul strives to avoid being devoured.  If at any point 
the attacker stops drinking or cannot continue the challenge due to trait loss, the victim's 
soul departs normally.  In some rare cases even failure at this state may brand someone with the 
marks of a diablerist.

If the attacker is ultimately successful, the body crumbles to ash and the diablerist 
immediately makes a Self-Control/Instinct test against five traits; they have absorbed 
the very soul and essence of another kindred into themselves and the Beast surges through 
their veins with wild abandon.

\subsubsection{Changing your Mind}
Before health levels are being drained, the attacker may decide without test or penalty to 
stop the process of diablerie.  During this time they can react to outside stimulus as 
normal, the true thrill of the act has not yet blinded them to the world.

Once health levels have started draining however changing your mind is incredibly difficult.  
The attacker must spend a Willpower trait and succeed a Static Mental challenge against 
six traits to willfully pull away.  At this point in time the attacker is completely immune 
to Mental and Social Disciplines, so caught up are they in the Beast's desire for the finest 
meal ever conceived.  At the same time however they are unable to defend themselves, resisting 
attacks only with Stamina traits and with the \emph{Survival} ability.

\subsubsection{Benefits and Drawbacks of Diablerie}
Once the diablerie succeeds the attacker is overwhelmed with a euphoria unmatched by any drug 
ever harvested or invented.  All rewards of diablerie are at Storyteller discretion, but often 
characters discover that they have absorbed Abilities, Traits, and perhaps even Disciplines or 
Merits of their victim.  The real prize however is if the victim was of more potent blood than 
the diablerist---if so, the diablerist immediately gains one dot of the Generation Background, 
and all of the increased privileges that Generation may afford them.

While all detriments of diablerie are at Storyteller discretion, almost unerringly the 
diablerist's aura is stained with dark, inky veins that are unmistakable in nature, visible 
to any who have the power of \emph{Aura Perception} and even to some powers of 
\emph{Thaumaturgy}.  What's more, the attacker may have absorbed more than just the positive 
aspects of their victim; Flaws, Negative Traits, and even Derangements are all likely outcomes 
of this gruesome and deplorable act.

Diablerie, the worst crime in the whole of the Camarilla, is usually punished by an immediate and 
unflinching \emph{Blood Hunt}, marking the transgressor as an enemy of all the Ivory Tower 
stands for.  See Chapter~\ref{sec:camarilla} for more information on this harshest of penalties.

\subsection{Frenzy and R\"{o}tschreck}
\label{subsec:frenzy}
Inside each kindred lies the Beast, a primordial power that wants only for two things---to kill 
and to survive.  Even the most rational and reflective kindred can feel its desire resonating in their 
chest, and as kindred slip down the rungs of Morality their actions are more and more likely to be 
ruled by those two simple motivations, the Beast drawing closer and closer to the surface of their night 
to night existence.

When a vampire loses control, normally due to some external stimulus as detailed in 
Chapter~\ref{sec:morality}, their frenzy is always either driven by rage or fear.  Both types of 
frenzies have the same mechanical effects, but are very differently role-played, owing to the 
powerful instincts driving them.  When in frenzy the kindred's rational mind is wholly gone, 
subsumed by the raw emotions of the Beast itself.  A frenzied character is immune to all wound 
penalties, through \emph{Incapacitated}, and need not risk an initial trait to resist Mental 
or Social Disciplines.  As a drawback however, they themselves cannot use any Mental or Social powers, 
relying only on those physical or transformative powers which strengthen their bodies.  In addition 
the Beast has no care or concern for the Masquerade, Status, or etiquette---it wants what it wants and 
it will do everything short of self-sacrifice to get it.  Two-way communication with a vampire in 
frenzy is almost always impossible.  The Beast will always use all applicable Ability and Discipline 
retests in pursuit of its goals, but it is up to the player whether or not to use Willpower to retest 
a given challenge.

Normally just called `frenzy,' the fiery temper and hate of the Beast is almost unmatched by anything 
that has walked the Earth.  While enraged the kindred lashes out at the target of their frenzy, or 
first people and objects blocking their way, attempting to utterly destroy the cause of their hate.  
The Beast in such a state will use the most efficient means of destruction available, often burning 
Blood into Celerity or transformative powers such as \emph{Wolf's Claws} to augment their 
preternatural lethality.  All frenzies caused by lack of blood are of this type, but the desire is to 
feed as much as possible, with the victim's death little more than an afterthought.  This type of frenzy 
can be avoided or controlled with a Self-Control/Instinct challenge.

A kindred driven to Frenzy through lack of \emph{vitae} in their system is much more likely to feed 
from targets instead of destroy them, unless their feeding is interrupted or hampered in some way.  
Kindred who fill their blood pool to more than two thirds while suffering a hunger-induced Frenzy 
may test to regain control.  If a starving kindred completely fills their blood pool they automatically 
return to their senses, the Beast satisfied.

Fear frenzy, also called R\"{o}tschreck, is the blind panic the Beast exerts when it believes its 
existence is in real danger.  A kindred in such a terror will do most anything to escape the source 
of their fear, particularly burning Blood for Celerity.

One frenzy may turn into the other if the situation changes, at Storyteller discretion.  With no 
conscious mind to rein in its desires, someone in frenzy is usually assumed to fail any Virtue test 
related to anger or fear.

\subsubsection{Controlling and Stopping the Beast}
If walking into a situation where one expects that they will have to make a Virtue test to avoid 
falling to the Beast, a character may spend a Willpower trait to steel their resolve against the 
forthcoming threat, but may do so only before a Virtue test is called for.  Specific rules governing 
Self-Control, Instinct, and Courage are found in Chapter~\ref{sec:morality}.

Once in frenzy a character has little option but to let the Beast run its course.  A character whose 
Morality path uses Instinct instead of Self-Control may endeavor to ride the frenzy as described 
previously, but this option is not available to those with Self-Control.  In either case a character 
in frenzy may spend a Willpower to gain control over their actions for one combat turn (roughly three 
seconds), though they still must obey the mechanical benefits and drawbacks of being in frenzy---they 
may speak and guide their actions, but may not use Social or Mental Disciplines, do not suffer wound 
penalties, and the like.

There are only three methods of stopping either form of frenzy:

\begin{enumerate}
	\item \textbf{Separation:}  If the object of one's frenzy has been removed from the scene, or the 
	frenzying character likewise removed, the Storyteller may either allow the frenzy to cease or 
	call for another test to calm down.  This rule may also apply if the object of one's anger frenzy 
	is slain or incapacitated.
	\item \textbf{Grave Injury:}  Only through great injury will the Beast be brought down.  If a 
	frenzying character is torpored or killed the frenzy stops, but this case may present its own 
	problems for all involved.
	\item \textbf{Being Talked Down:} All characters may attempt to calm or talk down a character in frenzy, 
	though in doing so they only provide an opportunity for the frenzied kindred to regain control, and 
	nothing is guaranteed.  This attempt takes a full combat turn, with the test being performed only at the end 
	of the turn.  With a successful Static Social challenge using only your Named Social traits, 
	e.g. no bonus traits or Status, against a difficulty of ten minus half the target's Morality rating 
	(round down), you allow them to test again to control their beast, even for characters with Instinct.  
	The retest for this challenge is \emph{Animal Ken}.  This means the difficulty to calm down a character 
	with 3 Morality is 9 traits, for example.  Failing to talk someone out of an anger frenzy will very likely 
	result in you becoming the object of their hate.
	
	The Animalism power \emph{Quell the Beast} is unique in that it utilizes a normal contested Social 
	challenge, including all relevant bonuses and modifiers, and pulls the target completely out of frenzy 
	if successful.
\end{enumerate}