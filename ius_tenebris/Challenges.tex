\section{Challenges}
\label{sec:challenges}
Vampire: the Masquerade is a social game, where the political machinations of each character are 
what drives both the night-to-night and overarching stories.  Even with the focus on interpersonal 
interaction, some direct conflict is unavoidable.  While role-playing through conflict is optimal, 
to help facilitate the resolution of what happens in the event that role-play isn't sufficient, there 
is a simple method to determine the outcome:  rock, paper, scissors.

While the process of throwing a challenge has a great deal of nuance and multiple options for players 
wanting to come out on top, the actual mechanics for throwing a challenge are very simple:  two players 
declare their intention and throw rock, paper, or scissors to determine who comes out on top.  Rock 
defeats Scissors which defeats Paper which in turn defeats Rock.  This section dives into the many 
complications that can arise in the throwing of challenges, but as far the overall mechanic goes, 
it is no more difficult than this childhood game.

Care should be shown to make sure that challenges are as non-disruptive to other ongoing scenes as 
possible, and that emotions don't run too high; contests of any kind are exciting, but in the collaborative 
environment of \emph{Sanguine Dreams}, we don't want tempers to get out of hand.

\subsection{The Challenge Process}
By and large all challenges will follow the same generic formula, outlined here.  

\begin{enumerate}
	\item \textbf{Initial Trait Bid:} each contestant declares their desired outcome and risks a 
	relevant Physical, Social, or Mental trait from their sheet.  Any Merits, Flaws, or other effects 
	relevant to the challenge are announced at this time.
	\item \textbf{Rock-Paper-Scissors:} all involved characters throw rock-paper-scissors to determine 
	the outcome.  If two or more challengers throw the same symbol, each declare the number of 
	relevant traits they have, with the higher number winning.  If one player is the defender in a 
	challenge, ties on trait totals default to them.
	\item \textbf{Retests:}  if a player isn't satisfied with the outcome of the first test, 
	they may endeavor to throw one or more re-tests, described more fully below.
	\item \textbf{Outcome:}  after all retests are thrown, the player who won the final test 
	has won the challenge, and so their declared intent occurs.  The penalty for losing a challenge 
	is the loss of any traits bid during the initial phase, which remain unavailable for the rest of 
	the evening.
\end{enumerate}

\noindent To initiate a challenge, each involved player must declare what they hope to accomplish and what trait 
they are risking in order to do so, e.g. ``I am \emph{Charming} enough to use the power \emph{Entrancement} 
on you!'' and ``I am far too \emph{Intimidating} for your charms to work on me!''  This ensures that all 
players are clear on exactly what the challenge is for, what retests are available (if any), and what 
happens depending on the victor.

If both players throw the same symbol, either on the initial throw or during any of the retests, players 
must compare trait totals to determine the outcome.  For example, a player involved in a Physical challenge 
(whether of Strength, Dexterity, or Stamina) adds together all of their current Physical traits and any 
bonuses from weapons or other modifiers.  Once both players have their totals calculated, the aggressor must 
declare their total first.  In the event that there is more than one aggressor, they claim at the same time. 
No player is required to reveal their full total, and may under-claim if they want to hide their true 
prowess, but whomever claims the higher total wins.  In the event of a tie, the defender is considered to be 
the victor; an attacker must overcome the defender in order to win.

If someone issues a challenge and you do not want to contest them, you may relent at any time before the 
challenge is thrown---this means that you do not risk any traits, but the challenger's victory condition, 
such as affecting you with a power or dealing damage, occurs without resistance.  Relenting is common when 
an effect would be beneficial or to reduce a challenge's interruption of an ongoing scene.

If a character issues a Mental or Social challenge and fails, they cannot issue the same type of Mental or 
Social challenge against the same character for five minutes of gameplay---you cannot continually try to 
\emph{Dominate} someone, for example.  Storytellers will be the final arbiter of whether a new challenge is 
too similar to a failed test.

\subsubsection{Special Hand Signs and Abilities}
Some Disciplines, most notably Potence and Celerity, grant special advantages when used in relevant 
challenges.  At times these Disciplines may grant a player the option to throw ``the bomb'' in addition 
to rock, paper, or scissors.  The symbol for the bomb is the same as rock with your thumb sticking up, 
and this sign defeats both rock and paper, but loses to scissors.  A player who has the ability to throw 
the bomb must declare such before the challenge is thrown.

Three advanced Discipline levels, notably \emph{Fleetness}, \emph{Aegis}, and \emph{Puissance}, grant the 
unique ability to ``win all ties'' when utilized in specific circumstances.  If a player uses one of 
these powers and a challenge results in a tie, they automatically win, as if they had claimed more traits.  
In the event that both players have a win all ties power active, compare traits normally.  If a condition 
forces you to lose all ties, such as through injury, the use of a win-all-ties power allows ties to be 
compared normally instead.  As with the bomb the ability to win all ties must be declared before a challenge 
is thrown.

\subsection{Types of Challenges}
Most often when engaging in a challenge you are contesting another character, the environment, 
or seeing what results fate has in store for you.  By and large there are only three types of 
challenges, which are used to represent these different situations:

\begin{description}
	\item[Contested:]  The most common type of test, Contested challenges are between two characters, 
	where both bid traits and describe victory conditions.  In the event of ties traits are compared, 
	with equal trait totals considered a win for the defender, if there is one.  Usually all combatants 
	may retest, depending on the specifics of the challenge being thrown.
	\item[Static:]  When a test is required but there is no active defender, the Storyteller 
	may call for a Static test, setting the difficulty at a specific number of traits.  After 
	bidding a trait, proceed as with a Contested challenge, but usually the defender, if there 
	is one, cannot retest or cancel you.  To succeed you must either win outright or claim more 
	traits than the set difficulty in the event of a tie.
	\item[Simple:]  Unlike the others, these tests do not require an initial trait bid and 
	have no applicable retests.  Some require that you win or tie to succeed, while others 
	demand an outright win.  Storytellers often use these as a ``good, bad, or worse'' fate 
	or luck challenge.
\end{description}

\subsection{The Initial Trait Bid}
In most all Contested and Static tests, an aggressor must begin a challenge by ``bidding'' one of 
her traits to initiate a challenge.  Defenders must also bid a trait in Contested tests.  At the same time 
all contestants must declare what kind of challenge they are entering into---one character may be \emph{Wiry} 
enough to punch another, while their target is too \emph{Agile} to dodge, for example.  The initial trait bid 
is important because it determines the type of challenge for that contender; most Potence powers can only be 
used in tests of strength, as an example.

There are the rare occasions, most notably with the Disciplines of Necromancy and Thaumaturgy, or with 
the \emph{Firearms} Ability, where one contestant will use a different trait category than another.  In 
this instance one individual may be engaging in a contest of Wits or Charisma while another a contest of 
Stamina or Dexterity.  The type of trait each challenger claims will determine the appropriate retests and 
bonus traits that can be applied to a challenge.

Note that characters may only bid a trait they currently possess and is relevant to the challenge at hand, 
and not ones lost from previous challenges or expenditures.  If a challenger does not have an appropriate 
trait to bid, or a willingness to bid it, they are forced to relent to the challenge.  Just as a Stamina 
trait would be improper to bid when attempting to throw a punch, an Appearance trait would not be applicable 
to initiate a remote challenge like \emph{Summon}.

It is during this phase that you must declare any Flaws you possess that are relevant to the challenge, and which 
Merit or Ability specialization you are using, if any, either to the opponent or to an ST if desired.  Remember that 
you may only use one Merit or Ability Specialization in the same challenge.

\subsubsection{Negative Traits}
Just as positive traits may be used to initiate a challenge, there exist Negative traits which, if used 
correctly, can make engaging in a challenge more risky for an opponent.  Common sources of Negative traits 
are those chosen during character creation (see Chapter~\ref{sec:creation}), the use of weapons, or Flaws 
such as \emph{Flesh of the Corpse}.  If you believe an opponent possesses Negative traits that would apply 
to the ensuing challenge, you may call one or more of them out before the challenge is thrown.  For each 
Negative trait you guess correctly, the opponent must bid an additional trait to continue the challenge; 
if they lose the test, all risked traits are similarly lost.

If however you incorrectly guess a Negative trait, you yourself must bid an additional trait in order to 
continue the challenge.  Negative traits are never lost and do not run out; they apply to every relevant 
(Mental, Social, or Physical) challenge until they are bought off with XP (see Chapter~\ref{sec:xp}) or the 
effect that grants them has worn off.

Only Negative Traits that are relevant to the current challenge are applicable to call; appearance-based 
Negatives such as \emph{Bestial} or \emph{Repugnant} are only applicable in local challenges where their true 
form is apparent (e.g. not using \emph{Obfuscate}). Similarly a Storyteller may rule that someone firing a 
gun is not hindered by their being \emph{Lame} if standing still. When in doubt, ask a Storyteller.

Similarly to positive traits, Negative traits should be well role-played as they give as much, if not more, 
personality to your character than their more admirable qualities.

Falling to the ``Wounded'' wound level also requires characters, whether aggressive or defensive, to bid an 
additional trait to continue all challenges.

\subsubsection{Not having enough traits}
If a character is forced to bid more named traits than they possess at the time of the challenge, 
whether from wound penalties, Negative traits, or other factors, they cannot continue the 
challenge and so must relent.  Remember that bonus traits such as from weapons, Status, or spending 
Blood do not count as named traits and so are not biddable in a challenge---they are used for tie 
comparison only.

Any kindred who finds himself low on traits may want to refrain from engaging in challenges to ensure 
they aren't caught completely defenseless should the occasion arise.

\subsubsection{Ties with No Defender}
In situations where both contestants are attacking or are considered aggressors, the normal rule of 
``ties go to the defender'' does not apply, since there is no specific defender.  If such a challenge 
results in a tie and both claim equal traits, both characters lose any traits bid as if they failed the 
challenge outright.  In this case there is simply no winner and no victory conditions occur.

Alternatively if all involved players agree, the combatants may throw another test, with the winner 
decided by one throw.  In the case of yet another tie, this last test is repeated until a clear victor 
emerges.

\subsubsection{Sacrificing Traits}
Some powers, including \emph{Possession}, \emph{Telepathy}, \emph{Subsume the Spirit}, and some uses of both 
Necromancy and Thaumaturgy, require a character to sacrifice traits in order to prolong, enhance, or create 
an effect.  Unless specified in the writeup for that power, all such traits must be expended prior to the 
challenge being thrown.

\subsubsection{Challenges with No Valid Defender}
Occasionally a character will wish to throw a challenge against someone who is not a valid 
target---investigating someone for using \emph{Mask of a Thousand Faces} when they aren't, or trying 
to \emph{Command} someone of lower Generation, for example.  While this situation will usually be handled 
by the first Golden Rule, a popular option is for the character, who does not know their target isn't 
subject to the challenge, to risk traits and perhaps spend retests as applicable in the attempt, even 
though there is no hope for success.  When in doubt please talk to a Storyteller.

\subsection{Retests and Canceling}
In the event that a Contested or Static challenge doesn't go your way, you may be able to ``retest'' and 
gain a second (or even third) bite at success.  Each type of retest may only be used when applicable, and 
only once per challenge per contender.  When all retests have been thrown, the challenger who won the last 
test has won the challenge.  Potential retests include:

\begin{description}
	\item[Ability:] if a character possesses a relevant Ability for the challenge, they may expend one 
	dot for the evening in order to gain a retest.  This is the most common use of Abilities during a 
	game session.  This retest can be canceled by a relevant Ability.
	\item[Discipline:] though rare, some Discipline powers such as \emph{Awe} or \emph{Might} allow for 
	a retest in specific challenges, as detailed in their writeups.  This retest can be canceled by the 
	same Discipline.
	\item[Merits:] some Merits such as \emph{Luck} may grant a retest, if the Merit's use is declared 
	before the challenge is thrown.  Merit retests cannot be canceled.
	\item[Willpower:] available only to the defender in Mental or Social challenges, by spending a 
	temporary Willpower trait they may gain an additional retest.  This retest cannot be canceled.
	\item[Situation:] in rare cases the Storyteller may dictate that the environment will grant one side 
	or both a retest, such as fighting in darkness.  Situational retests may be canceled or not by 
	Storyteller decision, and would usually require a relevant Ability.  If all combatants are subject 
	to the same situational retest it is often omitted.
	\item[Overbid:] a unique retest, attempting an Overbid represents the belief that your character is 
	simply too powerful to be so easily stopped.  By voluntarily losing your initial bid trait(s) and risking 
	a new one you may declare your intent to Overbid.  If your current named trait total is at least twice 
	the defender's, you gain an additional retest, though your initial trait remains lost, even if you 
	ultimately win the challenge.  Remember that named traits do not include bonus traits from weapons, 
	Status, or other miscellaneous sources.  Overbid retests cannot be canceled.
\end{description}

\noindent If one character endeavors to retest a challenge by using an Ability or Discipline and their opponent wants 
to stop them, they may ``cancel'' that retest by using a like Ability or Discipline, as detailed above.  If you 
cancel a given retest however you may not then use the same retest---canceling someone's Ability retest with your 
own Ability prevents you from calling for an Ability retest in the same challenge.

The opportunities to retest and cancel are optional, and need not be taken if not desired.  Please be careful to 
note all expenditures made from retesting or canceling, as they are spent whether or not the challenge is 
ultimately successful in your favor.

\addtocontents{toc}{\protect\setcounter{tocdepth}{1}}
\subsection{Remote Challenges}
\addtocontents{toc}{\protect\setcounter{tocdepth}{2}}
Not all challenges are conducted strictly by the players involved.  Challenges where one character 
may not instantly know the outcome, such as \emph{Summoning} someone from a great distance, should 
be performed by a Storyteller.  Inform the Storyteller of the challenge you wish to throw, your 
number of relevant traits, and any applicable retests you are willing to use.  The Storyteller will 
conduct the challenge and report the result to you afterward.  Until they do, presume you have used 
all stated retests and lost the challenge; this will ensure you do not accidentally double-spend 
Abilities or other traits while waiting for the response.

You should also ask a Storyteller to throw any challenges for you when you do not know the location 
of your target; this will ensure your scene stays continuous while the Storyteller finds your target 
instead of you leaving or putting the scene on hold in order to track them down.

Some powers exist which grant benefits but only to the local scene; these bonus traits or retests 
are not available in remote challenges.  Any such limitation, if any, is described in each particular 
power.