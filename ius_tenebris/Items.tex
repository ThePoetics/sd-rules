\section{Items}
\label{sec:items}
Often characters will come into possession of useful or important items such 
as a particularly expensive car, weapons, or ornate disguises.  Such items 
are represented by item cards, small notecards signed by the Storytellers that 
detail the item's abilities, statistics, and perhaps origins.  While small 
or normal items such as pens, a briefcase, or other normal household objects 
do not require item cards, anything used as a weapon or that is out of the 
ordinary must be.  Players should have on-hand any items their characters have 
on them, and keep all other item cards in the out-of-character room or otherwise 
out of the scene.  In any scene a Storyteller may grant or revoke item cards as 
they see fit, both for game balance and to improve role-play.  

Sometimes particularly enterprising kindred will attempt to create an item, 
usually requiring levels in the \emph{Crafts}, \emph{Expression}, or 
\emph{Repair} Abilities.  Crafting items takes time, determined by the 
Storytellers.  True masterpieces and wondrous pieces, the likes of which 
awe mortals and transfix Toreador, may take many weeks or months of effort, 
with regular actions being logged toward that end.

Most often item cards are used to represent weapons and armour.  More importantly 
than with any other card, characters wishing to make use of these items must have 
the cards on-hand at the start of combat.  Storytellers will actively refuse an 
item's use if a valid card is not present.

\subsection{Weapons}

Often relying on more than their preternatural gifts to dispatch their enemies, kindred 
of all Clans have embraced humanity's arms race to create better and deadlier tools with 
which to dispatch their enemies.  Barring exceptional circumstances no weapon will be 
allowed that is not detailed in this chapter.  All weapons require a valid item card 
signed by a Storyteller prior to their use, and may require running scenes or using Influences 
to acquire.

No actual weapon should ever be brought to game, and toy facsimiles must be approved by both 
the Storytellers and CAG-SSU president before their inclusion in a costume.  For the safety and 
security both of our players and of the game site itself, following this rule is essential.

\subsubsection{Weapon Traits}
All weapons grant bonus traits in the event of ties.  As mentioned in 
Chapter~\ref{sec:combat}, Combat, these traits do not count for Initiative or 
overbidding. All weapons also have Negative Traits which may be called against any 
character wielding them.  These function the same way as Negative Traits on a character 
sheet, and may be called as such.  All weapons also have a concealability rating, dictating 
how well the weapon can be hidden.  If a weapon is not properly concealed its presence 
should be announced to anyone interacting with your character.  Storytellers will not 
sign off on item cards having statistics that differ from those presented in this 
chapter without extremely extenuating circumstances.

Possession of several levels of a relevant Ability, such as \emph{Firearms}, \emph{Melee}, or 
\emph{Athletics} may, at Storyteller discretion, allow you to innately know the attributes and 
traits, both positive and Negative, of a given weapon presented to or used against you.

Unless specifically noted on the item card all weapons take one action to draw, and can 
only be used once drawn.  Some weapons have additional abilities, noted on in their 
description and detailed below:

\begin{description}
	\item[Fully Automatic:]  Firearms with this ability are able to shoot all of 
	their ammunition in one burst, dealing an automatic extra level of damage 
	on a successful challenge, but requiring the weapon to be reloaded before it 
	can be fired again.
	\item[High-Caliber:]  Some select firearms cause crippling wounds in their 
	victims.  After a successful challenge win or tie a Simple test to deal an 
	extra level of damage.
	\item[Speed:]  These quick weapons provide an extra +1 trait bonus when used 
	against someone wielding a weapon with the \emph{Heavy}, \emph{Clumsy}, or 
	\emph{Slow} Negative traits, if called against them.
	\item[Spray:]  Able to strike up to three adjacent targets who are at least 10' 
	away from the shooter, use the rules for Mob Combat (see page~\pageref{subsec:mobcombat}) 
	to resolve such challenges.
	\item[Staking:]  Melee weapons made of wood have a chance to immobilize kindred 
	targets.  After a successful damaging attack, make two Simple tests.  If you win 
	or tie both, the subject is immobilized, per ``Being Staked'' in Chapter~\ref{sec:health}.
	\item[Two-Hand Requirement:]  These large weapons are impossible to wield one-handed.
\end{description}

\subsubsection{Melee Weapons}
All melee weapons use the \emph{Melee} Ability for retests and are generally available, with some 
effort required to track down more exotic items.  All ax damage bypasses shields and actually 
destroy them after three successful hits.  \\

{\footnotesize
\begin{tabular}{ | l l l c l | }
	\hline
	\textbf{Weapons} & \textbf{Traits} & \textbf{Conceal.} & 
	\textbf{Dmg} & \textbf{Special} \\
	\hline
	Club & +1, \emph{Clumsy} & Jacket & 1B & \\
	Small Ax & +2, \emph{Clumsy} & Jacket & 1L & \\
	Knife & +2, \emph{Short} & Pocket & 1L & \\
	Stake & +2, \emph{Clumsy} & Jacket & 1L & Staking \\
	Short-sword & +2, \emph{Short} & Jacket & 1L & \\
	Large Ax & +3, \emph{Clumsy, Heavy} & Trenchcoat & 2L & Two-Hand \\
	Rapier & +3, \emph{Fragile} & Trenchcoat & 1L & Speed \\
	Staff & +3, \emph{Heavy} & N/A & 2B & Staking, Two-Hand \\
	Broadsword & +3, \emph{Heavy} & Trenchcoat & 2L & \\
	Greatsword & +4, \emph{Clumsy}, \emph{Heavy} & N/A & 2L & Two-Hand \\
	\hline
\end{tabular}
}

\subsubsection{Thrown and Projectile Weapons}
All thrown and projectile weapons use the \emph{Athletics} Ability for retests.  Acquiring 
grenades requires the use of \emph{Police} Influence 5 or \emph{Underworld} Influence 4. 
Crossbows cannot be used to attack more than once a turn, and a quiver is not concealable.  \\

{\footnotesize
\begin{tabular}{ | l l l c l | }
	\hline
	\textbf{Weapons} & \textbf{Traits} & \textbf{Conceal.} & 
	\textbf{Dmg} & \textbf{Special} \\
	\hline
	Dagger & +1, \emph{Clumsy} & Pocket & 1L & \\
	Molotov Cocktail & +2, \emph{Fragile} & Jacket/NA & 1A & Spray, Two-Hand \\
	Grenade & +2, \emph{Clumsy} & Jacket & 2L & Spray, Two-Hand \\
	Crossbow & +2, \emph{Heavy}, \emph{Slow} & N/A & 2L & Two-Hand \\
	Spear & +3, \emph{Clumsy}, \emph{Heavy} & Trenchcoat & 1L & Two-Hand \\
	Longbow & +3, \emph{Fragile}, \emph{Heavy} & N/A & 2L & Two-Hand \\
	\hline
\end{tabular}
}

\subsubsection{Firearms}
Utilizing the \emph{Firearms} retest, all firearms are assumed to have the capacity to 
fire 10 bullets without reloading, unless specified on the item card.  Please carry item 
cards representing any additional ammunition.  Reloading a firearm takes one action.  
Shotguns may only be fired twice in a turn, regardless of the level of Celerity possessed.  
Sniper Rifles and Assault Rifles are not available to the general public, requiring the 
use of \emph{Police} or \emph{Underworld} Influence 5 to obtain.  Other firearms are 
available if legally registered, otherwise requiring \emph{Police} or \emph{Underworld} 3. 
Vampires only take Bashing damage from firearms of any caliber.  \\

{\footnotesize
\begin{tabular}{|l l l c l|}
	\hline
	\textbf{Weapons} & \textbf{Traits} & \textbf{Conceal.} & 
	\textbf{Dmg} & \textbf{Special} \\
	\hline
	Pistol & +2, \emph{Loud} & Pocket & 2L & \\
	Heavy Pistol & +2, \emph{Loud} & Jacket & 2L & High-Caliber \\
	SMG & +2, \emph{Loud} & Jacket & 2L & Fully Automatic, Spray \\
	Rifle & +3, \emph{Loud} & N/A & 2L & Two-Hand \\
	Shotgun & +3, \emph{Loud} & Trenchcoat & 3L & Spray, Two-Hand \\
	Sniper Rifle & +3, \emph{Loud} & N/A & 2L & High-Caliber, Two-Hand \\
	Assault Rifle & +3, \emph{Loud} & N/A & 2L & Fully Auto, Spray, Two-Hand \\
	\hline
\end{tabular}
}

\subsection{Armour}
Many kindred who anticipate violence in their future decide to augment their supernatural 
toughness with more mundane methods of protection, namely armour.  All armour provides 
additional Healthy health levels (see page~\pageref{sec:health}) against specific types 
of damage suffered while worn; most do not protect against bullets except where specifically 
noted below.  Ballistic Vests specifically do not provide any protection against melee weapons 
while other armours do.  To obtain a Reinforced Vest or Riot Suit a character must 
utilize \emph{Police} or \emph{Underworld} Influence 4, but all other armours are readily available, 
with some time investment to acquire rare or uncommon items. 

If damaged, armour can be repaired by someone with the requisite \emph{Crafts} Ability and an 
amount of time determined by the Storytellers, usually at least one full day per health level 
repaired. \\

{\footnotesize
\begin{tabular}{|l c l c l|}
	\hline
	\textbf{Armour} & \textbf{Health} & \textbf{Negative} & \textbf{Conceal.} & \textbf{Special} \\
	\hline
	Leather & 1 & & N/A & \\
	Chain Mail & 2 & \emph{Heavy} & Trenchcoat & \\
	Ballistic Vest & 2 & \emph{Heavy} & Jacket & Ballistic \\
	Reinforced Vest & 2 & \emph{Heavy} & Jacket & Ballistic \\
	Plate Mail & 3 & \emph{Clumsy}, \emph{Heavy} & N/A & \\
	Riot Suit & 3 & \emph{Clumsy}, \emph{Heavy} & N/A & Ballistic \\
	\hline
\end{tabular}
}

\subsubsection{Shields}
Rare in the modern age, hand-held shields are still useful to some when protecting 
against incoming attacks.  All shields grant bonus traits when defending against melee 
or unarmed attacks, but not when attacking or against ranged or ballistic attacks save 
where designated.  Riot Shields are only available through use of \emph{Police} or 
\emph{Underworld} Influence 4.  No shield is concealable.  \\

{\footnotesize
\begin{tabular}{|l l l|}
	\hline
	\textbf{Shields} & \textbf{Traits} & \textbf{Special} \\
	\hline
	Impromptu/Environmental & +1, \emph{Clumsy} & \\
	Small Shield & +1 & \\
	Large Shield & +2, \emph{Heavy} & Ranged \\
	Riot Shield & +3, \emph{Clumsy}, \emph{Heavy} & Ranged, Ballistic \\
	\hline
\end{tabular}
}