\addtocontents{toc}{\protect\setcounter{tocdepth}{1}}
\section{The Role of Storytellers}
\label{sec:sts}
Helping lead every chronicle is a team of Storytellers, whose purpose is to 
improve the game and the player experience by providing opportunities to explore 
rich aspects of the setting at large, create plots and puzzles to be uncovered, 
portray non-player characters, and to make final adjudication when it comes to 
rules and fair play.  In this section we detail what responsibilities the 
Storytellers shoulder and how best to present suggestions.

An important note is that while the Storytellers may create the over-arching plot 
lines and direction of the game, setting the mode and tone of the world in general, 
\emph{Sanguine Dreams} is far from their game---the game belongs to the players, 
and it should be their experience as a whole, not solely the ideas of the 
Storytellers, which defines the game we play every week.

\subsection{A Dark, Wide World}
While the richest and best stories stem from interactions between player-driven 
characters, the World of Darkness is vast and it is the role of the Storytellers 
to present the rest of the world and its many denizens.  They portray everyday 
figures in the street, shadowy antagonists, and others designed to flesh out the 
setting.

In addition to portraying individual or groups of NPCs, the Storytellers are 
responsible for presenting an internally-consistent world to the player-base. As 
written the World of Darkness has more than three thousand years of vampiric 
influence, and many stories and characters have found their way into popular lore 
and kindred culture.  It is one of the major responsibilities for the storytelling 
team to make sure that the rest of the world conforms to the setting, save where 
specifically appropriate.  They are responsible for the whole of the world and the 
many pieces that may come into play during a chronicle, including the personality of 
other kindred, the state of foreign Domains, and even the Camarilla as a whole.

White Wolf has published many, many books on the world's history, most of which are 
presented in a first-person narrative, filled with contradictory and sometimes 
intentionally incorrect information.  The Storytellers decide what aspects of these 
stories are facts, which are merely conjecture, and which are propaganda designed 
to mislead young kindred.

As representatives of the expansive World of Darkness, Storytellers portray and 
direct all non-player-characters, whether they are active in the current chronicle, 
remote in far-off Domains, or lost to ancient history.  If your character wants to 
get a hold of a character not portrayed by a player, or pin down exactly what your 
character may know about the Convention of Thorns, speak to a Storyteller.

\subsection{Plot-lines and Direction}
Rich and fully-fleshed characters will organically grow their own plot as they meet 
and interact, forming alliances and rivalries, but by and large these interpersonal 
stories can't carry a whole game or give it direction.  To facilitate the larger, 
overarching story, a loose but purposeful plot is created that helps give definition 
to a particular chronicle.

Each chronicle tends to last between three and five years and so the metaplot, as it 
is often called, is designed to last about that long, taking characters and players 
through aspects of the setting they may not have encountered before.  Some plots are 
mysteries with many twists and turns, while others are focused on the supernatural, 
or the relationship between kindred and mortals, or even the changing political 
atmosphere that crosses Sect lines.  Whatever the story is, large parts of it are only 
loosely written, leaving a great deal of flexibility so the players can have a real and 
dramatic impact on its direction.

Two of the hardest aspects of creating and maintaining an engaging plot-line include 
keeping it interesting for the players and balancing its involvement in the nightly 
game.  The first because over the course of a chronicle the player base may change, 
and their desires with it.  What started out as a group excited for a mystery plot may evolve 
into one desiring a more straightforward story.  A good Storytelling team needs to be 
able to adapt their plots to fit the current desires of the game as a whole, and not 
just the desires of a vocal minority of players.

It is almost always preferable both from a game-play and player standpoint to have 
player-generated or organic plot take center stage over that which the Storytellers 
have devised, and more often than not the Storytellers will let their ideas take a back-seat 
if the current interpersonal plots or political situation is exciting enough.  The 
over-arching story is an important aspect of the chronicle, but by far the most vital 
part is the collection of stories each player brings to the game.

Similarly, while many times Storytellers will bring in an NPC either for temporary or 
recurring play, it is the focus of all Storytellers to make sure that it is the PCs that 
drive plot forward and are the ones most visibly exploring the world at large; the game 
is for the players, with the Storytellers just providing a framework.  After all, the 
stage hands and directors aren't the stars of a good theatre production.

\subsection{Out of Character Considerations}
A game can't run itself; even beyond the plot threads and ongoing stories there are 
many behind-the-scenes tasks required to keep a game vibrant and exciting, and the 
Storytellers are tasked with performing them, all with the aim of making the best 
game possible.

Storytellers are in charge of almost all out of character aspects of the game, from 
keeping accurate character sheet records, logging experience, taking care of Influences, 
to addressing player concerns, resolving interpersonal conflicts, and more.  Not least 
among those is a knowledge of the rules and systems in which we play.  Though White Wolf 
has given us a solid framework through their \emph{Laws of the Night: Mind's Eye Theatre} 
series, the Storytellers have come together to present this book, and its companion volumes, 
to make sure that there is no confusion regarding rules and that similar situations are ruled 
the same way every time.

No Storyteller knows every rule in this book, let alone the vast esoterica in White Wolf's 
many published volumes, but each is tasked with having a basic familiarity with the system 
as a whole and sound judgment enough to make on-scene calls, when asked to.  Their rules 
calls may not always align 100\% with what is presented in this book or others, but each 
call is made with the intention of bettering the game and moving play along.  If there are 
questions about specific mechanics or interpretations, they should be brought up after the 
scene, preferably submitted in writing so the Storytelling team as a whole can talk about 
your question or suggestion.

\subsection{Organization}
Though there are many different models of how to organize a Storytelling team, after years 
of exploration and experience typically \emph{Sanguine Dreams} uses the following primary 
roles for individual Storytellers:

\begin{description}
	\item[World at Large]--- covering the majority of the meta-plot as well as all NPCs 
	from outside of the local Domain, these Storyteller typically handle ``big picture'' 
	aspects of the game.
	\item[Interpersonal]--- dedicated to the interactions between characters and local 
	NPCs, these Storytellers are also focused on the many secrets and background ties present 
	in every PCs backstory, charged with bringing them to life and making every player 
	feel included in current goings-on.
	\item[Paperwork]--- an absolutely invaluable role, these Storytellers are responsible for 
	the creation, tabulation, maintenance, and storage of the game's records such as character 
	sheets, experience logs, character questionnaires, and more.
\end{description}

\noindent At times other roles can and should be added, particularly as the game grows.  In addition to 
helpers or aids to the above roles, specific players may be designated as ``Narrators,'' which 
means they are trusted by the Storytellers to run scenes, portray specific NPCs, handle rules 
calls, or address some other minor administrative aspect of running the game.

Not every specific Storyteller or Narrator will be intimately familiar with all aspects of 
current plots or background information, but by and large they share a great deal as to the 
ongoing state of game events.  If the Storyteller handling Influences, for example, isn't available 
to answer your question, they should be addressed via the forums or over email to make sure it 
can get a response in a timely fashion and not get lost.

\subsection{Contacting the Storytellers}
While the Storytellers are present each week at game to answer questions, portray NPCs, and 
advance various plot-lines, some questions are more far-reaching or take collaboration to 
answer.  These questions should be submitted to the Storytelling team either through the 
website (\emph{https://sanguinerpg.com}) or by writing them down at game.  Writing down your 
questions or concerns helps ensure that the specifics or intent of your message aren't missed, 
and that there is a record of what you wanted and when you asked.  In this way the Storytellers 
can go over unanswered topics easily and make sure that everyone receives their just attention.

The Storytelling team always encourages all players who have personal or out-of-character conflicts 
with others at game to bring those concerns to their attention so that they can be addressed, 
investigated if necessary, and ultimately resolved.  The Storytellers' primary purpose is to ensure 
a healthy and positive gaming atmosphere for all players.

Being a Storyteller for a game as complex and large as \emph{Sanguine Dreams} can be a difficult 
and stressful task, no matter how experienced or well-organized the team is.  If a question isn't 
answered immediately please be patient and know that there are many more aspects to running the game 
than what any one player sees, and that even if a response seems slow, one is coming.
\addtocontents{toc}{\protect\setcounter{tocdepth}{2}}