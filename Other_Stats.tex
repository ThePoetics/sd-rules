\section{Other Statistics}
\label{sec:other}
In addition to attributes, Abilities, Backgrounds, Disciplines, and 
the rest there are additional elements to a character sheet which 
demand particular attention.  These include Blood, Willpower, and a 
character's Nature and Demeanor. 

\subsection{Blood}
\label{subsec:blood}
Cursed to never again enjoy the taste of regular food and drink, blood 
is the sole nourishment on which a kindred can survive.  In fact, kindred 
cannot ingest any other sustenance, as their body violently rejects it within 
seconds.  Normally obtained by attacking or seducing mortals, more desperate 
kindred may resort to feeding off of animals or even raiding a blood bank, 
though such pales in comparison to the taste of fresh blood taken directly 
from a human.

Once a vampire imbibes blood it becomes \emph{vitae}, imbued with the 
power of the Beast and capable of fueling a kindred's many superhuman 
abilities.  Blood may be used to: \\ 

\begin{description}
	\item[Wake each night:] a kindred automatically expends one blood 
	trait to wake every evening unless they are in torpor.
	\item[Heal wounds:] described in more detail in Chapter~\ref{sec:health}, the only 
	way a vampire can heal is through the expenditure of blood.
	\item[Activate Disciplines:] some Disciplines such as \emph{Celerity} 
	or \emph{Thaumaturgy} require the use of blood to activate.
	\item[Gain Physical traits:] by spending a Blood vampires can gain a bonus trait 
	on ties for Physical challenges.  Bonus traits that do not cause a vampire to exceed 
	their generational maximum (as if Physical were their primary category, see 
	page~\ref{bg:generation}) are retained for the scene or combat.  If a character increases 
	their total over their generational limit, the extra only applies for the first Physical 
	challenge subsequently thrown.  A character may not gain a bonus greater than half their 
	permanent Physical traits, round up.
	\item[Create a ghoul:] by feeding a mortal a full point of Blood they gain some semblance 
	of the vampire's powers.  More information on ghouls can be found on page~\pageref{subsec:ghouls}. 
	\item[Hide their Predatory Nature:] by spending a Blood one can breathe the semblance 
	of life into their skin, negating up to 2 traits of penalty for having low Humanity for 
	a scene or hour.  This effect stacks with the Merit \emph{Blush of Health}.
	\item[Emulate natural processes:] with the expenditure of a Blood a vampire may, for a 
	scene or hour, enliven their body into the semblance of life.  They may have a functioning 
	heartbeat, warm their flesh, or even have intercourse, each of which costs a separate Blood trait.
	\item[Create a Blood Bond:] described further in Chapter~\ref{sec:additional}, even a single 
	taste of \emph{vitae} is enough to form a Blood Bond in another, whether mortal or kindred. \\
\end{description}

\noindent There is a limit to how much Blood a kindred can have within their system 
at any one time which is based on the Generation Background.  More potent 
characters possess a larger potential Blood pool and in some cases can even 
spend more than the usual one Blood per turn in a frightening display of their undead 
prowess.  Spending Blood is a reflexive action and may be done at any time.  
Some powers or situations may require spending more Blood than your 
character's Generation would allow in one turn---in this case the 
expenditures may require multiple turns, with effects taking place after all 
costs are paid.

When a kindred is low on Blood, represented by having five traits or fewer in their system, 
they risk frenzy at the sight or smell of blood as described in Chapter~\ref{sec:morality}.  
If a kindred falls to two or fewer traits in their system they become ravenous and all 
Virtue test difficulties increase by one.  As you cannot bid more traits in a 
Self-Control/Instinct or Courage test than you have Blood in your system, having a low 
Blood pool all but ensures frenzy will follow.

If a kindred is forced to spend Blood when they have none in their system, either 
to wake up or through other means, they instead take a level of unsoakable Lethal 
damage and instantly fall to frenzy.  Characters may not willingly spend Blood they 
do not possess.

Vampires are highly sensitive to substances present in the blood they ingest.  Drugs, 
alcohol, or other substances diffused in the blood may do more than just add taste---there 
may be real consequences for the kindred by way of temporary Negative Traits if they partake 
too heavily of intoxicated prey.  The specific repercussions, as determined by a Storyteller, 
will remain until the traits of \emph{vitae} that caused the drawbacks are used, following 
the last-in, last-out method.  For kindred who spend as little Blood as possible these 
penalties can last a week or more.  

This means that if a vampire gains one trait of tainted Blood that makes them \emph{Lethargic} 
for instance, they cannot be rid of the Negative trait until they have exhausted all of the 
blood that was in their system before they gained the trait as well as the problematic traits 
themselves.

\subsubsection{Starting Blood Pool}
Each character is assumed to arrive at gather at their maximum Blood pool, as determined by their 
Generation, minus any points in Blood- or feeding-related Flaws.  This penalty can be offset by 
Blood- or feeding-related Merits and points in the \emph{Hunting} Ability. Use of the \emph{Herd} 
Background can also offset this starting penalty.  Downtime activities, such as routinely using 
the power of \emph{Earth Meld}, healing a great deal of damage, casting pre-gather Thaumaturgical 
rituals, maintaining Ghouls (see Chapter~\ref{sec:additional}), and other like activities may 
further lower your starting Blood pool; please inform the Storytellers of any actions your 
character has taken during the week that may require Blood expenditures prior to entering play.

If you are unsatisfied with the amount of Blood in your character's system either 
before play or during gather you may ask a Storyteller to run a feeding scene where 
your character goes on the hunt, looking for sustenance.

\subsection{Willpower}
\label{subsec:willpower}
Vampires are often willful creatures, having to resist the dark pull of their 
Beast and the manipulative influence of other predators at every turn.  How 
resolute a character stands in the face of adversity is represented by their 
Willpower pool.

Willpower is a special type of trait that is directly related to your character's 
Generation Background.  Lower generation characters begin play with more Willpower 
and have a higher maximum, while higher generation vampires both begin play with 
fewer and cannot ever amass as much.  Willpower once used, is slow to return, 
refreshing at a rate of one per week, every Sunday before gather.  Additionally, if 
while running a scene with a Storyteller they determine that you have exemplified 
your character's nature, they may award you one additional recovered Willpower.

The primary functions of Willpower are to resist the influence of others and to 
power Disciplines such as \emph{Psychic Projection} or \emph{Quell the Beast}.  
When you are the defender in a Social or Mental challenge you may spend a temporary 
Willpower to gain a retest that cannot be cancelled, representing your determination 
not to be swayed by another.  You may also use Willpower to retest Virtue challenges, 
where your strength of resolve helps keep the Beast at bay.

Willpower however has additional uses and benefits.  Once per night per trait 
category---Physical, Social, and Mental---you may spend a Willpower to completely 
refresh all traits lost due to challenges or other effects, though this can only 
be done outside of challenges.  Similarly if a Storyteller declares that a particular 
challenge requires a certain level of Ability to attempt (e.g. \emph{Security} to pick 
a lock) you may spend a Willpower to be allowed the challenge even if you do not meet 
the prerequisites.

By spending a Willpower trait you may temporarily suppress the effects of an active 
Derangement for the scene, which gives you time to leave the area or otherwise avoid 
the situation which has triggered the malady.  Through a great deal of effort and many 
Willpower expenditures, it may be possible to overcome a permanent Derangement.

Through your force of will, represented by spending a temporary Willpower, you may 
completely ignore all wound penalties up to and including \emph{Incapacitated} and 
also guide yourself in Frenzy or R\"{o}tschreck for a single turn.  In addition if 
you believe a trigger for frenzy may be present in a scene, you may pre-emptively 
spend a Willpower to steel yourself against it, preventing the need for a test, but 
this can only be done before the trigger is present, and then at Storyteller discretion.  
This expenditure represents your character keeping things together just 
long enough to leave the situation.  Frenzy is explored more fully on page~\pageref{subsec:frenzy}.

Mortals rarely possess more than one or two Willpower.

\subsection{Nature and Demeanor}
Every kindred's personality is defined by two aspects:  that which lives at their 
core and that which they show the world.  These archetypes are respectively called 
your Nature and Demeanor and they play important roles both in the way your character 
interacts with others and mechanically from a gameplay perspective.

One's Demeanor may change from week to week, the kindred endeavoring to show a 
personality best suited for a given situation.  Younger characters are more likely to 
change their Demeanors while older or less mercurial kindred have fallen into regular 
and comfortable routines that reliably serve them well.  Natures on the other hand 
almost never change, representing an outlook and methodology to their schemes 
that lays at the center of their being.  Selecting the right Nature and Demeanor at 
character creation can help you shape how you role-play the character and may act 
as guiding principles for the many interactions the character will have with others.

A character's Nature has mechanical benefits and drawbacks as well.  If falling to 
frenzy would cause the character to violate his Nature, once per evening he may
get an additional retest to avoid the ravages of the Beast by drawing on their inner 
motivations.  However if another character discovers their true Nature, they may forever 
bid it as a Negative trait in any applicable challenges, as determined by the Storyteller.  
Kindred are usually very careful to keep their true motives and schemes carefully hidden 
from curious onlookers.

\subsubsection{Sample Archetypes}
\begin{description}[leftmargin=2.5in]
	\item[Architect]--- build something for the future, leave a lasting impression
	\item[Autocrat]--- take charge and do things right
	\item[Bon Vivant]--- you live only for pleasure and sensation
	\item[Bravo]--- others just get in the way unless you rein them in
	\item[Caregiver]--- you strive to comfort and protect others
	\item[Celebrant]--- indulge in your passions, whatever they may be
	\item[Child]--- others must protect and care for you
	\item[Competitor]--- everything is a contest that you must win
	\item[Conformist]--- fall in line, following orders
	\item[Conniver]--- get someone else to do your job
	\item[Curmudgeon]--- everything is terrible and you're going to let everyone know it
	\item[Dabbler]--- never stop learning and trying new things, new experiences
	\item[Deviant]--- you have no care for social etiquette or rules of behavior
	\item[Director]--- impose order on others
	\item[Fanatic]--- you serve an ideal with absolute conviction
	\item[Gallant]--- indulge in excess and exuberance to gain attention
	\item[Idealist]--- you are committed to a greater purpose beyond yourself
	\item[Judge]--- adhere to deep-seated standard of right and wrong
	\item[Loner]--- you don't belong anywhere or with anyone
	\item[Martyr]--- if you shoulder the burden everyone else may succeed
	\item[Masochist]--- test yourself through suffering and enduring
	\item[Monster]--- showcase evil through your nightly life
	\item[Pedagogue]--- you were born to teach and explain
	\item[Penitent]--- you have sinned and cannot find peace until you are forgiven
	\item[Perfectionist]--- there is no excuse for flawed efforts
	\item[Rebel]--- you attempt to break down the system
	\item[Rogue]--- looking out for number one is your best ability
	\item[Scientist]--- rational examination can yield all secrets
	\item[Soldier]--- take pride in accomplishing what you're told to do
	\item[Survivor]--- nothing can stop you
	\item[Thrill-Seeker]--- only the next high can outdo the last
	\item[Traditionalist]--- the old ways are the best ways
	\item[Trickster]--- you rely on humor and irreverence to avoid looking at life
	\item[Visionary]--- there is a grand design in your mind
\end{description}