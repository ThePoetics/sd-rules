\section{Advanced Influence Rules}
Some rules won't be encountered by every player, or cover situations that don't come up during normal play.  
This section covers those situations where the basic rule-set may not be adequate, to ensure rules calls during 
these scenarios are consistent across the board.

\subsection{Capacity and Managing Sphere Limits}
While the rules for each Influence sphere present options for an amazing amount of control and power kindred 
can wield, not every Domain has the resources to fully realize those abilities; perhaps there isn't enough 
organized crime to support any kindred from having 5 Underworld, let alone several, or perhaps there are no major 
routes through which high levels of Transportation could be used.  In all cases, each sphere has a specific limit 
to the amounts of Influence that can be wielded by local kindred.

In most cases these caps only affect the number of kindred who can possess levels 4 and 5 of a particular sphere, 
but in extreme cases may even apply to level 3.  If a sphere is at capacity no new kindred can rise to the capped 
levels, and any \emph{Growth} actions that would have elevate that kindred are wasted.  An enterprising kindred 
wishing to join the ranks of the Influence elite only has two choices:  knock someone down or increase the cap itself.

The ability to \emph{Kill} others' Influence totals is described in a former section, but the methods to manipulate 
a sphere's cap deserves its own section here in advanced rules.  To identify a sphere's existing capacity, and to see 
how close it is to being expanded or restricted, any character can use a one-point \emph{Endeavor} to perform the 
investigation.

A sphere increases its capacity for either level 4 or level 5 characters when enough successful points in 
\emph{Endeavors} have succeeded for the purpose---normally at least 2x the level to be expanded.  When submitting 
the actions, please designate whether you are attempting to raise the cap of level 4 or level 5, so an accurate 
count can be kept.  Once the cap has been raised, which is usually broadcast through Influence results, a single 
slot at the designated level is available to be filled.

Reducing a sphere's capacity works the same way; characters can submit \emph{Endeavors} designed to reduce the 
number of level 4 or 5 slots available for a given Domain, perhaps to counter attempts to expand it or to push out 
political rivals.  Unlike with expanding a sphere, only characters who are currently at levels 3 through 5 can work 
to narrow it---those with level 5 or 4 can reduce the number of 5 slots, while all of them can work to reduce the 
number of 4 slots.

In the event that a sphere has not filled its capacity of level 5 characters, any extra slots will be added to the number 
of level 4s available, until such time as a level 4 kindred climbs to the top spot.  This ensures that a vacant level 5 
slot does not prevent someone else from growing to level 4.

In the event that a sphere's capacity is lowered to the point where there are more kindred at a particular level than 
available slots, all characters at and above the capacity line will receive constant \emph{Erosion} effects, resisted 
by \emph{Reinforce} as they struggle to maintain control over a supersaturated system.  The \emph{Erosions} may affect 
those with greater holdings more severely than those with fewer, but in any case it is cause for concern and immediate 
attention.

Efforts to expand or constrict Influence spheres often have tangible in-game effects.  Where kindred are attempting 
to promote High Society, more nightclubs, fashion shows, and high-end boutiques will spring up.  If kindred try to 
restrict Police, crime will rise as fewer beat cops patrol the streets and fewer detectives investigate crime.  For a 
Domain trying to maintain a particular image, these factors may weigh heavily on their consideration.

\subsection{Slowing Down Endeavors}
Occasionally a character wants to have more actions available when performing an \emph{Endeavor}, perhaps in 
order to \emph{Stealth} or \emph{Conceal} their activities, without relying on others to \emph{Combine} with them 
for added effect.

To do so the character may spend some, but not all, of the points required by a particular \emph{Endeavor}, which 
will begin the process of acquiring what they wish as normal, save that anyone who successfully \emph{Watches} or 
\emph{Follows} them will see that an \emph{Endeavor} is being deliberately slowed.  The downside to instituting 
this even slower method of accomplishing tasks is that any \emph{Endeavor} thus slowed requires twice the number 
of points it would otherwise need to complete, which must be spent in successive cycles or the \emph{Endeavor} is 
lost. \\

\emph{\textbf{Example:} Maria is using her \emph{Police} Influence to receive a riot vest, which is a level 
4 \emph{Endeavor}.  She only has level 4 \emph{Police} however and doesn't want anyone to know that she is 
acquiring such equipment.  She may spend any number of her points this cycle to start the \emph{Endeavor}, 
saving the rest for other actions, but will now not see delivery of her riot gear until she spends a total of 
8 points toward this \emph{Endeavor}, which she can do as quickly or slowly as she likes, so long as she 
continues to put points towards it every cycle until it completes.}

\subsection{Transferring Influences}
At times kindred are charitable, or at least present themselves as such, and may gift or sell their hard-won 
Influences to other characters.  More complicated than the signing of a few documents, this special \emph{Endeavor} 
represents the legwork and planning to actually introduce the new kindred to the vast network of connections within 
a given sphere. 

To transfer some or all of one's Influences to another character, a special \emph{Endeavor} must be logged 
with the Storytellers which takes two cycles to complete.  The cost for this \emph{Endeavor} is a number of points 
equal to the amount of Influence being transferred, with the end result being half of which (round up) being removed 
from the initiator's sheet and added to the recipient's, provided they do not already have that much Influence.

All transfers must be performed while both characters are alive and active in order to facilitate the changeover. \\

\emph{\textbf{Example:} Sally is tired of managing her \emph{Finance} Influence and wants to give most of her
five dots away to Frank.  She institutes a 5-point \emph{Endeavor} to that effect. If it succeeds Frank will end 
up with 3 total dots of \emph{Finance} Influence, and all five dots are removed from Sally's sheet.  He will only 
receive this benefit if he had less than three points himself at the time the transfer concludes, though Sally 
loses her gifted Influence in any case.}

\subsection{``Unfinished Business''}
Occasionally characters die or are forced to flee the Domain in a hurry, and the question becomes ``what happens 
to their Influences?''  In short their Influences start to decay in the same way a character who isn't using theirs 
would, though perhaps on an accelerated time line, depending on circumstances.  Since they are not performing 
actions there is nothing to see with \emph{Watch} or \emph{Follow}, and there is nothing to \emph{Attack}. Their 
Influence may however be \emph{Killed} by those who are able, tearing down their empire in order to make room for 
others who may see opportunity to increase their own power.

Any actions instituted within the cycle the character departs or is killed are lost, not able to be seen through to 
completion.  For actions that take multiple cycles, the action fails for lack of maintenance if they have leave or 
are killed before the first full cycle is completed.  If after that point the action completes as normal, with 
specific outcomes being determined by the Storytellers. \\

\emph{\textbf{Example:} Thomas is planning on making a quick getaway from the Domain, but he has just started to 
gift his \emph{Political} Influence to another character (see above).  If he leaves during the same cycle in which 
the \emph{Endeavor} began, the effort will fail and no Influences will be transferred.  To see them through he needs 
to remain in the Domain at least through the end of the current cycle, but he has to ask himself whether the 
increased risk to life and limb is worth it.}