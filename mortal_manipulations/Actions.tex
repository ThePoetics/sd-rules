\section{Actions: What Influences can do for You}

Every time a character wants to utilize their Influences, they must frame it in the right 
context.  There are goals which are hostile or aggressive, those which are defensive or 
passive, and those which don't directly affect others at all.  When a character uses their 
Influence, they are allocating their actions to a specific task they wish to see succeed.

During every Influence cycle each character receives a number of actions equal to their 
dots in a given sphere.  For example someone with \emph{Bureaucracy 3} would have 3 actions 
they can use toward Bureaucracy tasks every cycle.  The real meat of the game of Influences 
lies in the cat-and-mouse play between characters, all jockeying for control over a given 
aspect of the county, and then trying to exert control over that sphere while keeping away 
those who would see their plans fail.

Most actions can have more than one point allocated to it, which can increase its effectiveness, 
such as a two-point \emph{Watch} having the potential to see more than someone who just used 
one point.  The details of how actions interact and whether they can be utilized in this way are 
described below.  Some actions are listed as requiring a target which means these actions have no 
effect on their own and only function in conjunction with another action or character. \medskip

\indent\textbf{Neutral Actions:} Combine, Endeavor, Growth, Hasten, Watch \\
\indent\textbf{Offensive Actions:} Attack, Block, Follow, Kill, Trace \\
\indent\textbf{Defensive Actions:} Boost, Conceal, Defend, Reinforce, Stealth

\subsection{Neutral Actions}
Some actions are neither aggressive nor defensive, and are often the most common types of actions 
used during the Influences cycle.  Unless specified neutral actions requiring a target must target your 
own, not those of others. \\

\begin{description}
	\item[Combine] \emph{(Requires Target Character)} \hfill \\
		Occasionally vampires do actually work together.  The \emph{Combine} action allows you to 
		gift one or more of your actions to someone else for a particular purpose, such as \emph{Growth} 
		or \emph{Concealing} a previous action.  For every two \emph{Combine} actions you use your target 
		receives one of the corresponding action, which can be any type of action listed in this packet.  
		Note that providing someone extra \emph{Endeavor} actions does not actually give them the ability to 
		perform higher-level \emph{Endeavors}, just that they have more points available for the actions they 
		can already undertake. 	When they submit Influences they must detail that they are using the actions 
		you're giving them or the effort is wasted.
				
		\textbf{Resolution Time:} One cycle
		
		\emph{\textbf{Example:}  Tina wants to really help Brick's \emph{Growth} this cycle so she tells the 
		Storyteller that she is using two levels of the \emph{Combine} action to give Brick an additional 
		\emph{Growth} action, which he must log if he wants to use it.} \\
		
	\item[Endeavor] \hfill \\
		\emph{Endeavors} are the most noteworthy and powerful uses of Influences in the game, and can affect 
		a wide range of game-changing outcomes.  Each Sphere has specific \emph{Endeavors} which can be 
		undertaken at each level, detailed more fully in the next section.  Unlike most other Influence actions 
		\emph{Endeavors} normally take two full Cycles to resolve, but only require you to spend points on them 
		when the action is initiated.  Each \emph{Endeavor} requires a number of points equal to the action's 
		level, as described in the next chapter.  Some \emph{Endeavors}, such as sweeping political reform, may 
		succeed but still take time to affect the greater world, as defined by the Storytellers.  Logging multiple 
		actions toward the same goal may help reduce this time.

		\textbf{Resolution Time:} Two cycles
		
		\emph{\textbf{Example:} Gabby wants to acquire a corpse, which is a \emph{Health 4} action, and so she 
		logs that she is spending four points of \emph{Health} Influence to do so. Provided it is not stopped, 
		after two full cycles she will receive the corpse. Next cycle she will have all of her actions available for 
		other \emph{Endeavors} or tasks as she may want.} \\
		
	\item[Growth] \hfill \\
		The process of gaining more Influence, with all the glad-handing and petty bribes required, is represented 
		by this action.  Each \emph{Growth} is a single action costing one point, though multiple \emph{Growths} can 
		be submitted per cycle.  Unlike most other actions \emph{Growth} takes two cycles to resolve, and requires multiple 
		applications to successfully attain the next level in a Sphere---you must have a number of successful 
		\emph{Growth} actions equal to three times your current level to go up, as shown in the following table:
		
\begin{center}
{\footnotesize
\begin{tabular}{| c | c | c |}
	\hline
	\textbf{Level} & \textbf{Growth for Next} & \textbf{Total Growth Actions} \\
	\hline
	1 & 3 & 0 -- 2 \\
	2 & 6 & 3 -- 8 \\
	3 & 9 & 9 -- 17 \\
	4 & 12 & 18 -- 29 \\
	5 & N/A & 30 -- 35 \\
	\hline
\end{tabular}
}
\end{center}

		Note that if you are at level 5 you cannot store more than 5 additional \emph{Growth}.
		
		\textbf{Resolution Time:} Two cycles
		
		\emph{\textbf{Example:} Severun wants more \emph{Police} Influence, tired of just having two.  He logs 
		two \emph{Growth} actions.  If the they are not stopped, in two cycles he will be one-third of the way 
		to having three dots of \emph{Police} Influence.} \\
		
	\item[Hasten] \emph{(Requires Target Action)} \hfill \\
		Sometimes waiting around for an action to resolve just isn't good enough---a character needs results more 
		quickly than normal.  The \emph{Hasten} action makes any \emph{Endeavor}, \emph{Kill}, or \emph{Growth} 
		resolve in one cycle rather than two.  There is a high cost to this expeditiousness however, as you must 
		spend one more point on \emph{Hasten} than the action to be sped up.  A large benefit to \emph{Hastened} 
		actions is that there is little chance of someone stopping you as there is no time to \emph{Attack} the 
		action.
		
		\textbf{Resolution Time:} One cycle
		
		\emph{\textbf{Example:} Margarit really wants to obtain small-time contraband (a level two Street \emph{Endeavor}) 
		sooner rather than later so as she logs her \emph{Endeavor} she also logs a three-point \emph{Hasten}.  This way 
		she's able to get her results at the end of this cycle rather than the next.} \\
		
	\item[Watch] \hfill \\
		One of the most common actions in the Influences game, \emph{Watch} allows you to see the goings-on of a particular 
		sphere, including actions which have either started or are ending in the current cycle.  While this action won't 
		give all the particulars, you will be able to see the following information:
		
		Total number of \emph{Watchers}, current \emph{Blocks}, attempts to \emph{Attack} other actions, the presence of 
		\emph{Combines}, and any \emph{Growth}, \emph{Endeavor}, \emph{Erosion}, or \emph{Kill} actions starting or ending 
		this cycle.  In addition you will be alerted if someone is \emph{Tracing} an action of yours or \emph{Following} you.
		
		\emph{Watch} actions are countered by \emph{Stealth}, with ties in the number of points allocated going to \emph{Stealth}.  
		For this reason those who wish to make sure they're seeing everything going on in a Sphere may want to log a higher-level 
		\emph{Watch} than normal.  Note that to affect an action (such as with \emph{Trace} or \emph{Attack}) you must have 
		\emph{Watched} the action yourself---you cannot rely on second-hand information.
		
		\textbf{Resolution Time:} One cycle
		
		\emph{\textbf{Example:} Bryan keeps his ears open and is interested in knowing what's going on in the world 
		of \emph{Finance}.  He tells the Storyteller that he wants to \emph{Watch} for what others are doing 
		this cycle, and will receive a report detailing just that.} \\
\end{description}

\subsection{Offensive Actions}
Occasionally people want to prevent others from accomplishing their goals, or figure out who 
managed to slip one past them.  These actions are aggressive in nature and represent actions 
taken directly against others' Influences.  In comparison against defensive actions, ties go to 
the defender except where specified.  While most commonly used against others, it is mechanically 
possible to use these actions against your own interests. \\

\begin{description}
	\item[Attack] \emph{(Requires Target Action)} \hfill \\
		After a character has successfully observed a specific \emph{Endeavor} or \emph{Growth} in progress, 
		through \emph{Watch} or \emph{Follow} actions, they can use the \emph{Attack} action to prevent it 
		from succeeding.  To do so they must spend more on the \emph{Attack} than the target has on \emph{Defend}.
		
		\textbf{Resolution Time:} One cycle
		
		\emph{\textbf{Example:} Luke saw that someone started a \emph{Growth} action last cycle through a 
		successful \emph{Watch} action.  Wanting to be the only kid on the block with and not 
		caring who he may be angering, he \emph{Attacks} with the hope of stopping that \emph{Growth}.} \\
		
	\item[Block] \emph{(Requires Target Action or Level)} \hfill \\
		Sometimes a character wants to put a hold on a specific \emph{Endeavor} or level of \emph{Endeavors}, 
		preventing anyone else from starting it.  By spending a number of points in \emph{Block} as the \emph{Endeavor} 
		would normally take, you can be sure nobody can start that \emph{Endeavor} this cycle.  Note that 
		\emph{Block} actions do not stop \emph{Endeavors} which are already underway.  A \emph{Block} can be 
		affected by and is countered by \emph{Boost} actions, with ties going to the defender.
		
		\textbf{Resolution Time:} One cycle
		
		\emph{\textbf{Example:} Marion decides that she doesn't want anyone performing level 2 Finance 
		\emph{Endeavors} this cycle, and as such puts a level 2 \emph{Block} in place.  Anyone trying 
		to do so without the benefit of \emph{Boost} will find it prevented.} \\
		
	\item[Follow] \emph{(Requires Target Character)} \hfill \\
		A specialized kind of \emph{Watch}, this action alerts you to all of the actions a particular character 
		takes in a given sphere.  To \emph{Follow} someone you must have successfully \emph{Traced} or 
		\emph{Followed} them in this sphere during the past four Influence cycles, and doing so provides the 
		following information for the current cycle:
		
		The level of any \emph{Watch} actions, with whom they were \emph{Combining} and for what purpose, any 
		\emph{Endeavors} or \emph{Growth} they start this cycle, what action they are \emph{Attacking}, 
		\emph{Blocks} they put in place, that they are \emph{Following} others, whom they are attempting to \emph{Kill}, 
		that they are \emph{Tracing}, how much \emph{Erosion} they are suffering, any action they are \emph{Defending}, 
		and whether or not they are using \emph{Reinforce}.
		
		\emph{Follow} will not see any actions protected by an equal or greater number of points in \emph{Stealth}, and 
		a \emph{Follow} is not considered to be successful if the target has submitted no actions that cycle.
		
		\textbf{Resolution Time:} One cycle
		
		\emph{\textbf{Example:} Garrett knows that Ari has Police Influence, from a previous \emph{Trace}.
		This cycle he is \emph{Following} her to find out everything she does.} \\
	
	\item[Kill] \emph{(Requires Target Character)} \hfill \\
		Sometimes just stopping someone's \emph{Endeavors} isn't enough; you want to utterly destroy their 
		control over an Influence sphere.  After having successfully \emph{Traced} or \emph{Followed} them in 
		the past four cycles you may attempt to \emph{Kill} their Influences.  Taking two full cycles to 
		complete, a successful \emph{Kill} destroys at least one stored \emph{Growth} action, lowering their 
		Influence total if they do not have enough \emph{Growth} stored to withstand the attack.  No Influence 
		may be reduced below one dot.
		
		\emph{Kill} is resisted by \emph{Reinforce}, where only extra points go through---a level 3 \emph{Kill} 
		against a level 2 \emph{Reinforce} will yield 1 successful point of \emph{Kill}.
		
		\textbf{Resolution Time:} Two cycles
		
		\emph{\textbf{Example:} Orion hates that Misty has so much \emph{Transportation}, an Influence 
		he wants more of, and subsequently successfully completes two \emph{Kill} actions against her. 
		Unfortunately for Misty she didn't have any saved \emph{Growth} actions logged and so she loses 
		one dot of her \emph{Transportation} Influence, going from 4 to 3. Since to go from 3 to 4 takes 
		9 \emph{Growth} actions, and the \emph{Kill} was worth 2, she drops to level 3 with 7 stored 
		\emph{Growth}.} \\
	
	\item[Trace] \emph{(Requires Target Action)} \hfill \\
		Sometimes kindred will take a particular interest in who was behind a particular Influence result, 
		and the \emph{Trace} action provides that information.  Anything observed through \emph{Watch} can 
		be targeted with this action, which will tell you who originated it.  This detective work is only 
		valid for up to two cycles after you noticed the action, but also provides the target's general level 
		of control over an Influence sphere (``a little,'' ``some,'' or ``a lot'').  \emph{Trace} actions are 
		countered by \emph{Conceal}.
		
		\textbf{Resolution Time:} One cycle
		
		\emph{\textbf{Example:} Davis is rather irked that his last \emph{Endeavor} failed due to someone's 
		\emph{Attack}. This cycle he is attempting to \emph{Trace} that action to see who was 
		behind the unprecedented aggression.  If successful he will receive the person's name 
		and their general level of Influence in this sphere.} \\
\end{description}

\subsection{Defensive Actions}
Just as there are actions to attack or discover the secrets behind others' goals, there are 
likewise actions one can take to reinforce their own Influences or prevent them from ever being 
discovered in the first place.  These defensive actions directly counter the hostile actions 
listed above.  In all cases, ties (such in \emph{Attack} versus \emph{Defend}), go to the defender.  
These cannot be used to benefit the actions of others; except where specified they can only help you own. \\

\begin{description}
	\item[Boost] \emph{(Requires Target Action)} \hfill \\
		Some times characters may be alerted that their action may be \emph{Blocked} during the next 
		cycle and that it will take some extra help to succeed.  The \emph{Boost} action provides 
		protection against such obstruction.  Similarly if someone wishes their \emph{Block} to be 
		more effective, they can \emph{Boost} their \emph{Block}.  In the case of ties, the non-\emph{Blocking} 
		character wins.
		
		\textbf{Resolution Time:} One cycle
		
		\emph{\textbf{Example:}  Royal heard that the Malkavians were going to \emph{Block} all efforts to 
		obtain a firearm this cycle, so in addition to logging that action he adds a \emph{Boost} to make sure he 
		can accomplish his goals in time.} \\
		
	\item[Conceal] \emph{(Requires Target Action)} \hfill \\
		Ensuring one's Influence results do not lead back to them is the function of the \emph{Conceal} 
		action, which actively confuses anyone attempting to \emph{Trace} the character.  This 
		does not protect against the action being spotted in the first place, only preventing it from 
		being tracked back to the character who initiated it.  In the case of ties, \emph{Conceal} wins.
		
		Since \emph{Trace} can be used for multiple cycles after an action was witnessed, it is often wise 
		to \emph{Conceal} for up to two full cycles after a given action completes.
		
		\textbf{Resolution Time:} One cycle
		
		\emph{\textbf{Example:} Samael made an unprovoked \emph{Attack} against someone's Influence last cycle 
		and only too late realized it was the Prince's.  To keep himself from being in trouble, he 
		tells the Storyteller that he is \emph{Concealing} his last action with every dot he has, making 
		it far less likely that the Prince will find out who stopped her plans.  He will do the same next cycle 
		to make to cover his tracks.} \\
		
	\item[Defend] \emph{(Requires Target Action)} \hfill \\
		Sometimes people know their \emph{Endeavors} or \emph{Growth} are unpopular and are likely to be attacked.  
		Utilizing the \emph{Defend} action provides protection against anyone attempting to \emph{Attack} them 
		during the same Influences cycle.  Ties go to the person using \emph{Defend}. 
		
		\textbf{Resolution Time:} One cycle
		
		\emph{\textbf{Example:} Peggy found out that someone \emph{Watched} her start an important 
		\emph{Endeavor} last cycle and believes they mean to stop it.  This cycle she uses her 
		actions to \emph{Defend} it, hoping it will be enough to nullify anyone wishing to see her fail.} \\
	
	\item[Reinforce] \hfill \\
		Some particularly nasty individuals may want to \emph{Kill} another's Influence, either to tear 
		them down or take their place at top of the food chain. The only way to defend against a successful 
		\emph{Kill} action is to \emph{Reinforce} one's Influence, which protects from these targeted attacks
		for one cycle.
		
		\textbf{Resolution Time:} One cycle
		
		\emph{\textbf{Example:} Ronin noticed last cycle that someone instituted a \emph{Kill} action and fears it 
		could be targeted at him.  Not wanting to lose all he had gained, he devotes all of his actions to 
		\emph{Reinforce}, hoping it's enough to stop the onslaught.} \\
	
	\item[Stealth] \emph{(Requires Target Action)} \hfill \\
		The counter to \emph{Watch} and \emph{Follow}, this protection hides another action from view.  Any 
		action successfully hidden with \emph{Stealth} will not show up in Influence results.  Multiple points 
		can be spent on \emph{Stealth} in an effort to hide it from stronger observers, and each logged 
		\emph{Stealth} action affects a single target action.
		
		\textbf{Resolution Time:} One cycle
		
		\emph{\textbf{Example:} Orion wants to be sneaky with his \emph{Block} and to make sure nobody sees that 
		one was in place, so he adds levels of \emph{Stealth} to foil anyone attempting to \emph{Watch} 
		the sphere in general or \emph{Follow} him in particular.} \\
\end{description}