\section{Derangements}
\label{sec:derangements}
Derangements are terrifying maladies which twist the perceptions of those suffering from them, 
and should not be chosen lightly.  Most derangements have specific triggers or situations that 
cause them to activate, including stressful events such as those listed in Chapter~\ref{sec:morality} 
under ``Virtue Tests.''

The mechanical effects of a Derangement may be suppressed temporarily (for the scene or hour) 
with the expenditure of a Willpower trait, but to be truly cured takes months and perhaps years 
of dedicated rehabilitation.  This section explores many, but not all, of the potential 
afflictions which can cripple kindred.  Other potential Derangements may be found in various 
Clanbooks but require specific Storyteller approval before they are accepted for play.  \\

\textbf{Antisocial} --- This label might be considered a misnomer, as people with antisocial personality disorder are not necessarily shy or reclusive, nor do they lack social skills.  Oftentimes, the opposite is the case---a person is very socially domineering, even aggressive.  Antisocial personality disorder refers to a lack of compassion, empathy, and superego/conscience.  The person simply cannot care about any being other than himself, and almost always places his own wants and needs above those of others.  People with this disorder casually lie, cheat, steal, and exploit, and a few even rape and kill.  Most psychologists and law enforcers consider this the ``serial killer disorder.'' Ted Bundy, John Wayne Gacy, and other monsters are classic examples of offenders who, while legally sane and capable of making moral distinction, simply don't care. \\

\textbf{Avoidant} --- \emph{Not compatible with Histrionic.} Avoidant personality disorder is characterized by extreme social anxiety.  Vampires with this condition often feel inadequate, avoid social situations, and seek isolation for their nightly routines.  They fear rejection and social humiliation, often needlessly so.  They prefer routine and exaggerate the potential difficulties of new situations to rationalize avoiding them.  Some display an imaginative bent, creating fantasy worlds in which they are dominant, loved, or triumphant.  Avoidant people yearn for social relations yet feel unable to attain them.  They are frequently depressed and have low self-confidence. \\

\textbf{Berserk} --- You have a tremendous difficulty controlling your anger and frustration.  When confronted with stressful situations, you often lose control, lashing out at those unfortunate enough to cross your path, whether friend or foe.  You must win or tie two Simple tests to avoid frenzy, thrown after the initial Self-Control/Instinct challenge. \\

\textbf{Blood Sweats} --- When you're stressed, you get so nervous and agitated that your state of mind affects your body.  Much as a mortal might become jumpy and break out in cold sweats, you become likewise ill at ease.  The sweat in your case, however, is blood that works its way to your skin.  This blood is very obvious in your sleeping environs, not to mention staining your clothes and making you a disturbing sight to vampires and mortals alike.  Blood sweats cause you to begin play each night an additional point down in your blood pool.  In addition, the blood is quite obvious (your clothes are stained soon after changing and you perspire continuously), and you're often nervous and twitchy.  You are down 2 Blood each gather. \\

\textbf{Blood Taste} --- You hate the taste of blood.  Your body craves it, but you think it tastes disgusting.  You go out of your way to feed on drunks (the alcohol helps kill the taste), even hanging out in bars.  You must win or tie a static test each time you feed to see if you spit out the blood.  If you fail the test, you must burn a Willpower if you wish to keep the noxious fluid in your mouth. Suffer a -4 penalty to your starting Blood pool each gather. \\

\textbf{Borderline} --- Vampires with borderline personality suffer from mood instability and low self-image.  Constant sudden mood swings and bouts of anger characterize sufferers of this disability, particularly when frustrated.  Borderline victims sometimes take out their anger on themselves, masochistically inflicting injury or even trying to kill themselves.  They think in very black-and-white terms and often form intense, conflict-ridden relationships. \\

\textbf{Bulimia} --- \emph{Not compatible with Gluttony.}  Bulimic characters salve their guilt and insecurity by overindulging in activities that comfort them---in this instance, consuming food (or blood, for vampires).  Characters with this affliction will gorge themselves as much as possible when under stress, then purge their systems through dramatic means and consume more.  Characters with this derangement must make a static Conscience/Conviction test against four traits when feeding; failure means the vampire violently expels all blood in their system save four.  This derangement will severely affect a character's starting blood pool. \\

\textbf{Crimson Rage} --- A character with this derangement is prone to fits of anger with little provocation.  While the two bear certain resemblances, this state is quite different from frenzy; frenzy is the instincts of the vampiric Beast while Crimson Rage is a character's own feelings of helplessness and inadequacy.  Characters with Crimson Rage are not protected from being pushed over the edge into frenzy while insane, however.  Whenever this derangement is active, the character gains the Negative traits \emph{Violent x2} and \emph{Impatient}. \\

\textbf{Desensitization} --- Effectively an emotional amputee, vampires with this affliction cannot truly feel any sort of strong emotion, be it joy, sadness, anger, or love; they are simply unable to form any meaningful emotion.  Unable to fully believe in their own ideals, all Conscience/Conviction difficulties are increased by one. \\

\textbf{Disassociative Blood-Spending} --- One of the more internal maladies, this affliction inhibits a vampire's ability to consciously spend blood.  Vampires with this Derangement have an embarrassing tendency to spend blood when wholly inappropriate, such as to boost their physical abilities or burn for Celerity.  Storytellers may decide once per session that the character has unconsciously spent blood, or that the character wakes up with fewer traits than expected, and need not elaborate on how or when the blood was spent.  Players are also encouraged to take the initiative by spending blood at inopportune times as well, without Storyteller interaction. \\

\textbf{Fugue} --- Characters with this affliction react to stress by adopting a specific set of behaviors; in the process they suffer blackouts or periods of memory loss.  Whenever confronted by extreme stress the character must make a static Willpower challenge; failure means the character blacks out and the player must role-play the character's trance-like state.  Otherwise, control of the character passes to a narrator for a scene, who dictates the actions the character takes in order to remove the stress.  At the end of the fugue, the character `regains consciousness' with no memory of his actions. \\

\textbf{Gluttony} --- \emph{Not compatible with Bulimia.}  Gluttonous vampires have difficulty taking their sustenance in moderation.  Why stop when one is merely sated?  Why not drink in the heady vitae until there is no more?  This derangement is particularly prevalent among elder vampires who have indulged their vices for so long that they lack the ability to control their hunger.  Vampires suffering from this derangement must spend a Willpower trait when they wish to stop feeding if not yet full.  A gluttonous vampire automatically frenzies when confronted with the sight, smell, or taste of blood when hungry (blood pool at 5 or less), and may be continuously snacking, despite not needing more. \\

\textbf{Histrionic} --- \emph{Not compatible with Avoidant.}  People with histrionic personalities need to be the center of attention at all times, often interrupting others in order to dominate conversations.  They use florid language even when describing something mundane events, and they seek constant praise.  They may dress provocatively or exaggerate injuries to gain attention.  They also tend to inflate social relationships, believing that everyone loves them; they describe the most casual acquaintances as dear friends.  They need to be adored and use and manipulate others to provoke this result.  Nothing infuriates them more than being ignored. \\

\textbf{Hysteria} --- Characters with this derangement are unable to properly control their emotions when subjected to stress or pressure, becoming vulnerable to violent mood swings and fits of intense violence.  The vampire must test to resist frenzy anytime stress is present; in addition, whenever the vampire fails in a particularly stressful or prominent instance, she enters frenzy automatically. \\

\textbf{Immortal Terror} --- This madness stems from the vampire's inability to deal with the true scope of his own immortality.  Terrified by the implications of really living forever, the vampire copes by developing a strong unconscious death wish.  Whenever the character is confronted by direct evidence of his immortality, such as attending a funeral or watching a mortal ally die, the character must make a static Willpower challenge (versus 4 Traits) to avoid undertaking actions that might result in his immediate destruction.  Such actions can be as indirect as breaching the Masquerade by telling a reporter about Kindred society, as long as the act carries potentially deadly consequences.  Note that the vampire is not consciously aware that he seeks his own destruction, and he resists attempts to persuade him otherwise. \\

\textbf{Manic Depression} --- This Derangement causes a character to suffer devastating mood swings.  Whenever the character fails to achieve a personal goal, she must win a Static Willpower challenge or fall into a depressive state for a number of scenes determined by a Storyteller.  While depressed, her temporary Willpower traits are considered halved (round down, minimum of one), and she may not spend Blood to raise her Physical Traits.  After that, she enters a period of upbeat energy and excitement, pursuing her goals obsessively for a number of scenes equal to the time spent in depression.  During this time she is one Trait down to resist frenzy of any kind. \\

\textbf{Megalomania} --- These individuals have made power the focus of their existence, and such characters must always be the most potent in their environment; where the power stems from is irrelevant, just so long as they are dominant over all others.  They believe that other people are divided into two classes: lesser beings and beings elevated above their worth.  Rivals are considered competition.  Due to their supreme conscience they are considered one Trait up on all Willpower tests while their derangement is active, but they must also make a Willpower test (difficulty six Traits) to resist any opportunity to commit diablerie during that time. \\

\textbf{Narcissistic} --- Narcissistic personality disorder is characterized by extreme self-centeredness.  As with histrionic disorder, people with this condition seek attention and praise.  They grossly exaggerate their accomplishments, expecting others to acknowledge them as superior.  They tend to be choosy about picking friends, since they believe that not just anyone is worthy of them.  They tend to make good first impressions, yet have difficulty maintaining long-lasting relationships.  They are generally uninterested in the feelings of others and may take advantage of them. \\

\textbf{Obsessive/Compulsive} --- Obsession = cleanliness.  Compulsion = cleaning.  Characters suffering from this derangement are driven to control their environment.  Obsessive characters keep one thing in their life constant---for example personal cleanliness or keeping things quiet.  Compulsive characters perform specific actions or sets of actions, such as washing their hands constantly or always feeding on mortals in a ritualistic fashion.  Obsessive/Compulsive characters are one trait up to resist any attempt to Dominate or otherwise coerce them from their set behaviors, but they frenzy automatically if forcibly prevented from adhering to their derangement. \\

\textbf{Paranoia} --- Paranoid beings believe that all their woes stem from an outside source.  Many paranoid beings come up with intricate theories about just who is against them and why; those they suspect of being their cause are often subject to swift and brutal violence.  Paranoid characters trust no one, not even those blood bound to them, and they have a difficult time interacting with others.  They are one trait down on all social tests while their derangement is active, and any sign of suspicious activity forces them to test to resist frenzy. \\

\textbf{Phobia} --- Something in your past affected you deeply.  It may be the sight of a shovel or the feeling of cold water around your body, but you have the instinctual trigger that causes you to frenzy.  If this trigger affects you, you must test as if for R\"{o}tschreck.  This phobia frenzy, if triggered, does not cause any lasting negative traits. \\

\textbf{Regression} --- Characters suffering from this affliction avoid facing responsibilities or consequences by retreating to a younger state where less was required of them.  During this state they may alternate between times of whimsy and temper tantrums, but they will always seek to put a more powerful individual between them and whatever is plaguing them.  Victims are two traits down in all Mental challenges. \\

\textbf{Sanguinary Animism} --- This illness is unique to vampires, a response to guilt for feeding on mortals.  Afflicted vampires do not believe they merely consume the victim's blood, but a part of his soul as well.  The character hears her victim's voice inside her head and is assaulted by ``memories'' of the victim's life, all created by the vampire's subconscious.  Whenever the vampire feeds on a mortal, they must make a static Willpower challenge; success means she is distracted as above and is one trait down on all challenges for the remainder of the scene.  Failure means the character gains a second angry, reproachful personality bent on driving her to ruin.  The character is at a one Trait penalty to all actions for the duration of the madness, and must role-play the inner conflict involved; this madness lasts until moments just before dawn. \\

\textbf{Schizophrenia} --- A victim of Schizophrenia is afflicted in a number of ways.  The individual most notably experiences hallucinations on the edge of their perceptions; manifestations as whispers that the character hears around the corners of buildings or behind them, shadowy figures moving just on the edge of their vision, or the feeling of insects crawling over their skin are all common.  Victims of Schizophrenia truly feel that their hallucinations are real; there is no doubt in their mind that the voices telling them to kill their mother are really there, and should be obeyed.  The character affected by this derangement is highly encouraged to role-play the true intensity of this malady.