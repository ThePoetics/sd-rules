\section{The Camarilla}
\label{sec:camarilla}
Standing staunchly against the forces of its enemies for more than five centuries, the 
Camarilla, sometimes called the Ivory Tower, bills itself as a civilized and ordered 
society, the only defense against the ravages of the Beast beating within the chest of 
every kindred.  Its proponents laud its stability and structure, extolling the virtues 
of powerful Domains which have remained firmly in Camarilla hands almost since its inception.
Detractors however point to the heavy-handed brutality and overwhelming disparity between the 
haves and the have-nots, the structure closely resembling the medieval fiefdoms popular in the 
era in which it was birthed---the eldest among kindred have almost limitless control while 
the newest neonate has little chance to survive, let alone take power for themselves.

Within the Camarilla each Domain acts independently, operating as a sort of city-state.  While 
the base laws are the same from Domain to Domain, specifics in how the laws are interpreted, 
punishments meted out, and the whims and dictates of various officers can vary greatly.  As 
the quality of a kindred's life is directly proportionate to the amount of power and resources 
at their disposal, frequent travel is frowned upon and viewed with suspicion, in addition to its 
inherent dangers, as someone who would willingly start over time and time again must either be 
incompetent or not invested in a Domain's prosperity.

The Camarilla is supported by three principle pillars: the Traditions, Boons, and Status.  
Before any new kindred is allowed to explore the world they must learn the intricacies of these 
all-important aspects of nightly life; someone who has a poor grip on any of them will quickly 
become an outcast and disgraced, a pariah among predators.  While there are other customs and 
standard practices, it is these three which have come to define the Camarilla and its role in 
every kindred's life.

\subsection{The Traditions}
The Camarilla has remained strong for the past five centuries by its rigid conformity to six 
immutable laws which form the cornerstone of its society.  They are the Traditions and while 
every city's Prince may decide how to interpret these laws, the wording has never changed.  
Any who dare break even a single one of these laws risks the severest of punishments as an 
example for all others who would think to dare follow suit. \\

\begin{description}
	\item[The Masquerade:] Thou shalt not reveal thy true nature to those not of the Blood. 
	Doing so shall renounce thy claims of Blood.
	\item[The Domain:] Thy domain is thy concern. All others owe thee respect while in it. 
	None may challenge thy word in thy domain.
	\item[The Progeny:] Thou shalt sire another one with permission of thine elder. If thou 
	createst another without thine elder's leave, both thou and thy progeny shall be slain.
	\item[The Accounting:] Those thou create are thine own childer. Until thy progeny shall be 
	released, thou shalt command them in all things. Their sins are thine to endure.
	\item[Hospitality:] Honor one another's domain. When thou comest to a foreign city, thou 
	shalt present thyself to the one who ruleth there. Without the word of acceptance, thou 
	art nothing.
	\item[Destruction:] Thou art forbidden to destroy another of thy kind. The right of 
	destruction belongeth only to thine elder. Only the eldest among thee shall call the 
	blood hunt.
\end{description}

\subsection{Officers of the Camarilla}
From time immemorial Camarilla Domains have chosen select members of the populace to serve 
in specific capacities in ensuring stability and adjudicating conflicts, social or physical 
as they may be.  The idea of ``elections'' or ``voting'' for any of these positions is as 
alien a concept to modern kindred as it would be to Dark Ages nobility, the society of which 
many Camarilla Domains endeavor to emulate.

\subsubsection{The Prince}
For most members of the Camarilla, the Prince is the most important and powerful individual 
they will ever encounter.  The kindred who claims Praxis over a city or county holds almost 
absolute authority within the Domain and is the final decision-maker when it comes to dispatching 
Camarilla justice within those bounds.  Often the Prince is the primary contact for important 
members of foreign Domains and spends many nights examining the long view of the region's health.

Princes are responsible for granting others the right to stay within their borders, a process 
called ``Acknowledgment,'' as well as choosing many of the other officers who will keep their 
laws and decrees enforced.  It is said that ``there is no such thing as former Princes,'' and with 
good reason---once having attained the highest kindred position of the local level, few future 
Princes would risk that individual staying in their Domains, lest the urge to rule again rears its 
head.

An important duty of the Prince is to reward the good behavior of their subjects by the granting of 
Status, even as other officers strip it away.  A Prince who gives out too little Status may find 
themselves with too few allies to fend off challengers to the throne, but one who gives out too much 
may be seen as weak or easily manipulated.  Those who last in the grand chair have learned the delicate 
balance of praise and admonishment, of perception and reality.

Often addressed by a deferent ``Your Majesty,'' long-standing Princes are traditionally viewed with 
great reverence and fear, owing to their almost unilateral ability to control the Domain over which 
they rule.

\subsubsection{Seneschal}
Assisting the Prince as a second-in-command or master of the house, a Seneschal's duty is to 
attend to the night-to-night affairs of the Domain, and to ensure it is running smoothly.  
Often addressing conflicts between Clans or officers, acting in the Prince's stead when they are 
unavailable, and keeping proper counts of feeding rights and territories, the Seneschal's 
usual occupation is to make sure the Prince does not need to get involved in any matter that 
arises, busy as they are with more important matters.

The Prince chooses their Seneschal, who acts as a chief advisor and confidant, and who may 
act in their stead when necessary.

\subsubsection{Keeper of Elysium}
The long history of the Camarilla would be nothing without its traditions.  The Prince 
charges the Keeper of Elysium with upholding those most sacred places, the Elysia 
where violence and intemperance are not permitted; safe havens for all.  In addition 
they are the visible protectors of the Masquerade, the First Tradition, and are responsible 
for admonishing those who would flout the society's most sacred law.  

For those who would threaten or break either of these traditions, the Keeper is charged with 
making public their crimes and the punishment, often social in nature, relating to the offenses.  
The Keeper is assisted in their task by Guardians of Elysium which serve a similar function, to 
a lesser degree.

\subsubsection{Sheriff}
Though other officers may curtail those who would act out socially, it is a rare Prince who 
lasts long without a strong Sheriff, chosen for their unflinching loyalty and unquestionable 
efficacy in administering the Prince's justice.  Directly tasked with the investigation into 
and punishment of those who would flaunt the Prince's decrees, including many of the core 
Traditions, the Sheriff is at once private detective and police captain.

Often very martially capable, some Sheriffs operate more as organizers, letting their various 
Deputies handle the physical conflicts that often arise from the Sheriff performing their duty.  
Some kindred, often Toreador and Ventrue, wrinkle their nose at the idea of a Sheriff regularly 
attending gathers, as such may suggest that they expect trouble to arise at any moment.

\subsubsection{Scourge}
Not every kindred belongs in a proper Camarilla Domain---enemies of the Sect, notorious 
interlopers, or even mortal Hunters may make their home in a Prince's Domain.  To weed out 
these harmful influences a Prince will often choose a Scourge, charged with removing those 
elements that are neither Acknowledged nor desired in a Domain.

This position has fallen out of fashion since its height in the late middle ages, but with 
recent movements by Anarchs, the Sabbat, and other forces, some Princes feel it is necessary 
to remind these unaligned kindred that their Domain is not a safe haven for their indulgences.

Assisted by Huntsmen, the Scourge is an even less popular sight at gather than the Sheriff for 
what their position represents.  Most often their duties are extremely martial in nature, and 
Scourges are chosen for their brutality, as it doesn't take long for word of a Domain's ability 
to clean itself of undesirables to spread to others who may have otherwise desired to visit.

\subsubsection{The Primogen Council}
Not every position in the Camarilla is chosen by the Prince.  While in more traditional 
European Domains the Primogen Council is more a collection of advisors chosen specifically 
by the Prince, in the more progressive Americas the Primogen Council has come to represent 
the most important members of each Clan, where they collectively address Domain issues and 
advise the Prince on the will of the Populace, which he may ignore at his own peril.

Each Primogen is seen as the embodiment of their Clan---if someone strikes a Primogen, it is 
often seen as equivalent to striking each member thereof.  Similarly, if a Primogen lashes out, 
be it socially or physically, against another member, some see that as a declaration of war 
between that individual and an entire Clan.

Issues often before the Council often include feeding territories, grievances between Domain 
members that need not involve other officers, and the long-term stability of the region, whether 
through cooperation or by staying away from each others' projects.

Each Clan has their own method of choosing their Primogen, but often the individual who sits 
upon the Council is the oldest or most powerful member of a given Clan.

\subsubsection{Harpies}
Analagous to a modern high school cheerleading squad, each Domain has within it those kindred 
who delight in pointing out the social failings and missteps of others.  These individuals are 
called ``Harpies,'' and the label is not a positive one, referencing the winged horrors that 
cackled at helpless travelers in ancient Greece before descending on them to rend their flesh.

One of the Primogen Council's most important duties is to select a Lead Harpy, which not only 
serves as a major point of contact for other Domains, but also is tasked with the publicizing of 
all noteworthy scandals within a Domain---who yelled at whom, who arrived insultingly underdressed 
for a grand event, who was caught discussing treason, and so forth.

The singular purpose of this Lead Harpy is to maintain the social order by way of Status---making 
sure that those who have it are continuously acting in ways congruent with their labels, striking 
down those who reach too high above their station, and maintaining the status quo where, over time, 
the truly great will rise while all others will stay small and unimportant.

Harpies are often unliked by their targets, but universally recognized as being essential for the 
prosperity of a Domain; those in power want to ensure their rivals number very few, while those 
young or low kindred want to see the high and mighty be taken down a notch or two.

Often releasing reports about the goings-on within a Domain, the news they bring is not always of 
scandal, but also of boons owed and events from far-off Domains.

\subsubsection{The Others}
While a Prince may name individuals to other positions, the Camarilla as a whole only recognizes 
those listed above as being worthy of Status and recognition.  There are however other accepted 
positions whose power dwarfs even that of the Prince, and it is a smart kindred who forgets they 
ever heard titles like ``Archon'' or ``Justicar,'' the very mention of which may bring unflinching 
and unrelenting wrath down upon them.

\subsection{Other Customs and Procedures}
In addition to the all-important Traditions and Status within the Camarilla, there are other 
common or well-defined practices which almost every kindred should be aware of.  These include 
how favors are traded, the process through which one sires a childe, places of safety and respite, 
properly arriving in a new Domain, and the ultimate penalty for crimes against the Camarilla.

\subsubsection{Acknowledgment}
Acknowledgment is the process through which a kindred receives both permission to 
stay within a given Domain and protection under the laws and Traditions of the 
Camarilla.  While the formality of asking permission to stay is as varied as the 
Princes who grant it, typically a kindred who desires to visit or reside in a new 
home will write to their Primogen or the Seneschal, who will bring the request 
to the Prince.

When the kindred arrives in their hopeful new home it is traditional for them to 
make themselves known at Elysium where they will wait for permission to visit the 
Prince.  If they are expected they will often find their Primogen or the Keeper 
waiting for them.  Unexpected kindred will likely be found by the Keeper or a 
Guardian, who will then inform the Primogen or Prince that a visitor has arrived.

If a kindred possesses positive standing Acknowledgment is usually little more 
than a formality, though a vital one.  The Prince or his officers will ascertain 
why the kindred is relocating, their purpose for choosing this Domain in particular, 
and whether the kindred is likely to be a benefit to the area.  If so, they are 
welcomed as a member of the local Camarilla, entitled to the rights and privileges 
thereof but also beholden to its rules and responsibilities.

Some Princes may demand some proof be made of the kindred's worthiness, either 
by the support of their new Primogen or through the exchange of a Boon.  While such 
a process could be an insult to well-known or respectable kindred, younger 
individuals with little proven value can expect to be met with some scrutiny.

A kindred seeking Acknowledgment within a Domain is functionally asking permission to 
stay, use the resources of, and contribute to their new city.  While a short-term 
visitor must receive the Prince's blessing to exist within his borders, as it is his 
Domain, they may not necessarily ask for formal Acknowledgment, as their primary 
allegiance rests in their home Domain.  There is no precedent for requesting or granting 
``temporary'' Acknowledgment; either someone is welcome in a Domain or they are not.

Kindred who are not a part of the Camarilla, either by Clan affiliation or personal 
choice, must strike a difficult balance with the local establishment.  Kindred who are 
Acknowledged are members of the Camarilla; those desiring to live outside its protections 
and strictures must find other means of guaranteeing their survival.  They have no rights 
under the Camarilla Traditions, but often either formal treaties or local custom might give 
them some manner of protection from hostile or aggressive actions.  In truth, the only 
protection an Independent or unaligned kindred has within a Camarilla court is the 
populace's fear of reprisal, either from the Prince or from the victim's Clan.

\subsubsection{Holding Courts and Gathers}
Within a Domain local kindred meet to discuss the politics of the night, air 
grievances, and make deals to further their own aims.  Such a gathering may be hosted 
by almost any resident, and usually all Acknowledged kindred are invited.  Private 
meetings are rarely called gatherings unless a large number of individuals are invited.

The host of a formal gather may earn prestige and Status for providing a pleasant or 
entertaining evening, seeing to their guests' needs, and taking care of any small issues 
which may arise.  However should the gather go poorly due to their shortcomings or 
shortsightedness they may receive scorn and derision, particularly from the Harpy 
who is always watching for kindred who err in the social arena.

Traditionally gathers hosted or called by a Prince are called courts and designate
that the sovereign will receive petitions, hold official meetings, Acknowledge new 
kindred, and make announcements.  In America many Princes set aside specific or 
recurring times to hold court, even at other kindred's gathers, a departure from the 
European tradition of holding proper court only by special decree.

\subsubsection{Embrace and the Accounting}
While every Clan has its own traditions and customs when it comes to the selection and 
embrace of a new childe, let alone what happens afterward, there are some aspects of 
the process which are standard throughout the Camarilla.  

The Right of Progeny can only be granted by a Domain's Prince, which permits a kindred 
to embrace a single individual.  Often granted in exchange for Boons or services 
rendered, permission may be given to embrace a particular individual or anyone in 
general.  Once negotiated for and received, the would-be sire may then embrace.

The embrace begins by the sire draining their target of all blood.  At the moment of 
death they are fed a small amount of the sire's vitae, which will bring the Beast to 
life within the new vampire.  However nothing is without risk and the success of the 
embrace is not guaranteed.

Once the childe awakens, normally within a few minutes, they feel the first ravages 
of the Beast.  Unused to its power they are utterly incapable of stopping themselves 
from lashing out at their surroundings; forward-thinking sires have sources of vitae 
handy both to feed the new vampire and also as targets for the blind rage of their 
progeny.

As the awakened vampire returns to their senses, instruction begins.  This period, 
called the Accounting, usually lasts between five and twenty years and is when the 
sire educates their childe in both how to survive as a vampire and the laws and rules 
of Camarilla society.  While we as players may read a game book and understand the 
society, to actually live within it takes endless nights of practice and patience.

The Accounting is a dangerous time for both sire and childe---not only is the sire 
giving up much of their time and resources to raise the fledgling vampire but they 
are also held responsible for any mistakes or missteps their childe takes.  The 
Camarilla is not a forgiving society and in this instance the sins of the childe 
are certainly the sins of the sire, both being severely punished for any slights of 
tradition or law.

Once the sire believes the childe is ready to be released from Accounting and 
presented to the Camarilla as a fully-fledged member, further meetings with the 
Prince are requested, wherein which the Prince may test the knowledge and education 
of the new kindred.  If the Prince's expectations are met the childe is granted 
formal Acknowledgment before the entire court and is now responsible for 
themselves, though often sires will keep a tight leash on their progeny for 
several more years as missteps would speak ill of their tutelage.

\subsubsection{Elysium}
One of the longest-lasting and most respected customs among all kindred-kind, 
rumored even to predate the Camarilla, is that of Elysium, the creation and maintenance 
of safe harbors for all kindred regardless of origin or affiliation.  Such locations, 
protected by the Keeper of Elysium, are often places of great beauty or cultural impact, 
important structures for local residents both mortal and eternal.

While some Keepers of Elysium allow politics to be discussed while in these 
refuges from the trials and tribulations of the night, just as often such talk is 
forbidden, the Elysia declared safe havens from all forms of conflict.  Unless 
otherwise specified no Disciplines, save those required to preserve the 
Masquerade, may be used, particularly on or against other visitors.  In all cases 
across the world, violence is never permitted on Elysium save the Keeper or their 
Guardians bringing down someone who has flagrantly violated this supreme dictate.

So important is the idea of Elysium to the Camarilla and its basis as a neutral 
ground for meetings or conference with kindred of every variety that the 
protections against violence or disruption include all those traveling directly to 
or from Elysium.  An unknown visitor being found by the Scourge need only say they 
are heading directly to Elysium to be given free passage to do so, though usually 
with armed escort to ensure they arrive as declared.

In short, Elysia are the only true safe havens for kindred and the Camarilla takes 
great pains to ensure their prestige and splendor are never diminished.

\subsubsection{Punishment within the Camarilla}
While squabbles between kindred or even Clans can be settled privately with 
back-room deals and behind-the-scenes arrangements, once the Domain's officers 
step into a situation it is much more difficult to keep a matter under wraps.  

If a kindred acts in a way that is disruptive or disrespectful to the society 
or those of high station the Harpy may levy punishment by stripping positive 
Status or bestowing negative Status upon the rude upstart.  Other officers such 
as the Sheriff, Scourge, and Keeper of Elysium can also chastise members of the 
Domain under certain circumstances, listed in a following section.  If Primogen 
or the Prince have granted Status to the individual in question they may revoke 
their support, showing their displeasure.  If the kindred has Status from other 
Domains letters may be sent and phone calls made to convince the backer to 
withdraw their support in the face of new scandal.

Not all punishments are based around Status however, though thrashing someone's 
standing is the most popular method.  Often Boons or favors are encouraged to 
make reparations.  Though no kindred can be forced into agreeing to a Boon, the 
alternatives for not accepting the agreement can be even more horrifying, all but 
ensuring the miscreant conforms.  Sometimes when a kindred has committed harm 
against important members of the populace the Prince may decree that the injured 
party may blood bond the aggressor one step.  This is a very unpopular punishment 
among Clans that encourage individuality such as the Brujah or Gangrel, but it 
guarantees similar actions will not be taken again.

For more serious crimes or transgressions against the core tenets of the 
Camarilla or Prince, punishments only escalate in severity.  Temporary banishment 
from gather or restrictions on where and when someone may feed are common ways to 
enforce the Prince's will, backed up by his officers, but even these harsh 
penalties, often coinciding with social repercussions, are not sufficient for 
some crimes.  More brutal and totalitarian Princes may elect to torture the 
subject in private, often forcing the criminal to leave their wounds unhealed for a 
set period of time.  Such a display is greatly distasteful for more humane kindred, 
but there is no doubt it gets the point across that dissent will not be tolerated.

If someone has been a recurring and constant pain, particular to another important 
individual, the Prince may retain their Acknowledgment but grant select individuals 
the Rite of Destruction over them, meaning that they have full permission to end that 
kindred's existence at any time.  Most kindred on the receiving end of such a 
threat quickly find themselves a new home in a new Domain.  Alternatively the 
Prince may exile a transgressor not just from gather but from the Domain as a whole, 
revoking their Acknowledgment.  Such a person, were they to stay in the local area, 
would not have the protections of the Traditions, including the prohibition against 
killing other kindred---a very disadvantageous position for them to be in.

The most serious punishment a Prince can levy upon one of his subjects is called the 
\emph{Lextalionis}, or ``Blood Hunt.''  This decree, once made, may never be revoked 
and so is only used in the most dire of circumstances against the most troubling and 
violent of offenders.  The Prince holds a special court gathering at Elysium, calling 
every Acknowledged member to hear the proclamation.  Once assembled, the Prince reads 
the list of the criminal's transgressions and orders that all kindred Acknowledged 
in that Domain, from that night until the end of nights, are to make all efforts to hunt 
down and destroy the named kindred within that Domain.  The Prince then signs the scroll 
in blood, casts it into a flaming brazier, and declares the Hunt begun.  Contrary to 
almost any other situation in Camarilla society, the traditional prize for taking down a 
Blood Hunted kindred is the right to diablerize him, unless explicitly forbidden by the 
Prince at the time the announcement is made.  In modern nights most Princes specifically 
deny this historic privilege, though they are expected to offer other rewards.

Any kindred found to be harboring a fugitive of justice may be subject to terrible 
punishments levied by the Scourge, Sheriff, or other officers.  In the case of a 
Blood Hunted kindred being protected, they themselves will become the subject of a 
new Blood Hunt.  Camarilla justice is brutal, merciless, and unforgiving.

\subsubsection{Boons}
In a world where kindred have the capacity to amass more wealth and resources than 
many small countries, the exchange of favors is far more valuable than material goods.  
These favors are called Boons and their use is an essential aspect of every Camarilla 
kindred's unlife.  Boons are inviolable, and to break an agreement once made is to 
forever be outcast, hunted and hated in every respectable Domain across the globe.  Often 
the exchange of Boons or their use is called ``prestation.''  Boons typically come in five 
varieties, in order of increasing severity:

\begin{description}
	\item[Trivial Boons] are exceedingly small and without risk or discomfort.  A kindred 
	of standing holding the door for one that has none, or offering a moment of their time for 
	a private conversation.
	\item[Minor Boons] represent favors that may cause some small hardship or giving up a limited 
	resource.  Recommending someone for a position or teaching a basic level of a common Discipline, 
	for example.
	\item[Major Boons] are large favors that can represent severe sacrifices of time or 
	effort which and may bring problems for the kindred in debt.  Supporting political actions 
	that harm one's own Clan, teaching hidden secrets or higher-levels of Disciplines, and 
	coming to the social defense of a deplorable kindred are examples of Major Boons.
	\item[Blood Boons] are only appropriate where someone risked very real danger to help or 
	aid another, such as entering into combat against a lesser opponent, sacrificing a truly 
	significant portion of one's holdings or political capital, or taking an action that causes 
	them real and lasting political harm.  At this level a kindred may even demand that the debtor 
	drink from them, establishing or furthering a Blood Bond.
	\item[Life Boons] are the rarest of boons, issued only when someone has truly saved another's 
	life and/or put their life on the line in service to another.  This is a terrible debt to owe, 
	for anything and everything can be demanded in exchange, including the murder of other kindred.
\end{description}

\noindent Boons are a valuable resource not just to collect from others but also to owe; if your potential 
service is valuable, the person whom you owe will have a vested interest in your well-being and 
continued good standing, keeping their boon powerful.  If you are an outcast your ability to affect 
real change may be limited, and thus the value of your debts is decreased.  What may be a Major Boon 
from an unimportant kindred may be a Minor Boon from a Prince or Primogen, owing to their unique 
position and high social standing.

To engage in Boons, an agreement between two kindred must be made for that purpose.  Either an 
immediate exchange or the promise of future service must be detailed, with payment explicitly 
defined as a particular boon.  It is a very good idea to record the date and circumstances 
surrounding such an exchange so in future there is no question about what was owed and to whom.  
Some kindred also inform the Harpy of boons owed, not only to ensure that the kindred in question 
is held accountable but also so there is public record of their debts---it is a function of 
Camarilla politics that if an individual is slain other than by the Prince's decree all boons 
they owed transfer to their killer.  Owing boons can in this way save your life, if others are 
more reticent to pay your debts than slay you.

When a Boon is called in and the favor demanded is appropriate for the type of Boon owed, 
it becomes the immediate and pressing need for the debtor to make every effort to fulfill 
the task.  Such is the nature of kindred favors---they may not always come at convenient 
times, but must be followed to the letter, when and where cashed in.

Boons are such an important aspect of Camarilla society and to the setting as a whole that 
they are both an in- and out-of-character mechanic.  If a character breaks a Boon they owe, 
even if no other character discovers the truth, they are forever branded with the negative 
Status \emph{Boon Breaker} and all Boons owed to them are nullified, though they are still 
responsible for other Boons they owe.  While mistakes happen and we as players do not have 
perfect recall, all effort should be made to rectify the situation agreeably to all parties.  
This is another reason making detailed notes of the Boons your character owes and is owed can 
be so valuable, both in- and out-of-character.  That said however, a crafty kindred may still 
find ways around the wording or intent of a Boon, but to play such games may place the 
character's reputation at stake as above, at Storyteller discretion.

Sometimes debts will be called in at cross-purposes with each other; for instance a Primogen 
who is tasked by one character to support a particular Harpy candidate, while someone else 
demands they support someone else.  In this case the first Boon called in takes precedence.  
In this example the debtor can tell the second kindred that they are unable to do so, though 
they need not explain that it is a matter of prestation.  This denial can also happen if someone 
demands a service that is ill-suited to the level of the Boon being called, such as demanding 
to be granted the Right of Progeny for a Trivial Boon.  The debtor is well within their power to 
suggest that the individual re-think their demands.  When in doubt, please talk to a Storyteller.

For administration purposes all Major, Blood, and Life boons should be submitted to the 
Storytellers in writing during the same session the agreement is made, with both participants 
signing the card.  This greatly helps reduce confusion and disagreements about what was or 
wasn't agreed upon at a later date.  Minor boons may be similarly recorded if desired.

\subsection{Hostile Threats}

A Camarilla Domain is besieged by enemies on all sides dedicated to its ruin.  The Sabbat is 
filled with an almost religious zeal for the destruction of the Camarilla way of life, vowing 
to erase it from the Earth.  Any Domain wishing to survive longer than a few years must 
prepare its defenses for the eventual arrival of its ancient enemies.

In addition to the visible and violent dangers posed by the Sabbat there are Independent Clans 
and kindred who see the Camarilla as infringing on their territory, rights, or existence.  
Usually lacking a strong martial force, these vampires work with subtlety, sowing discord and 
disharmony within a Domain, hoping to cause it to collapse from within.  The Camarilla has never 
trusted outsiders, though at times it can accept them as neutral neighbors, keeping a wary eye 
on them and any deals struck with local kindred.

Not all dangers come from without however, as the violent or political machinations of even 
loyal Camarilla kindred can lead to its downfall.  A Domain without a strong centralized power 
base is likely to be picked off or crumble, leaving its corpse to be picked at by the circling 
jackals nearby.  When there is a protracted question of leadership, Domains fall.

Sometimes when a Domain gets sloppy, or too many near-breaches of the Masquerade occur, mortal 
hunters will descend on the city, hoping to stem the tide of kindred influence on humanity, 
usually by destroying every vampire they can find.  Some are even rumored to possess truly 
terrifying gifts, from the Camarilla's perspective, even going so far as to mimic true Disciplines.  
Vehemently maintaining the Masquerade is a Domain's best defense against attracting their attention.

Vampire legends tell of other creatures, other supernatural entities which prowl the night.  
Travel is dangerous and deadly, and no kindred leaves the relative safety of the city without 
good reason and better planning.  The woods are home to predators eager to tear kindred asunder, 
and there are worse things still in those places where kindred fear to tread.

%
%	STATUS
%

\section{Status}
\label{sec:status}
Status is the measure of a kindred within the Camarilla---by its dictates the 
worth and value of every vampire is measured and weighed, with those at the top 
fighting to keep others down and those below fighting amongst themselves for the 
scraps offered by their betters.

In a world where mythical creatures can perform miracles by the power of their 
blood, ``evidence'' doesn't hold much weight when charges or allegations of 
misconduct are levied; all that matters is the weight of someone's standing.  This 
makes the elders almost unassailable and every young vampire merely pawns in the 
machinations of their older brethren.  When one makes a formal claim however they 
are both relying on and risking their Status; if what they say is patently false 
(e.g. ``by my status the moon is made of purple yarn'') or countermanded by someone 
with greater standing, the Harpy or other officer is likely to strip their standing 
for being so wrong and not worthy of the favor they had previously received.

All Status falls into one of three categories---Age, Patronage, or Positional---and 
represents how widely respected and well-known a given character is.

\subsection{What Status Means}
Status is the foundation of the Camarilla, and all Elders' power.  Kindred with great 
standing are ``always'' right in disputes, while those with less may never speak out 
against their betters lest they risk of swift and brutal retaliation.  Those with high 
standing wish to see few rise as challengers, while those of low standing seek only 
to gain favor.  As the Prince and Primogen grant their support, the Harpy tears it 
away, keeping a fine social balance of who is respected and who sits unimportant in the 
wings.

Status can be both positive and negative, and even though someone may possess great 
renown, finding themselves slandered with Negative Status can undo years of hard work 
and good will.  One who possesses Negative Status might not be welcomed at gather 
until it is removed.  Those who have earned two Negative Status traits are normally 
ejected from the Domain or at the very least lose their Acknowledgment, as no Prince 
would wish to entertain such creatures at their court.

Status is very fluid and the relative value of the Domain's kindred is always changing.  
It is both an in- and out-of-character mechanic in that characters respect those with 
greater station and Status provides mechanical benefits in challenges.

\subsubsection{Visiting Another Sect}

While every sect has some form of Status system, confirmed members of other Sects and 
those who are truly Independent only respect Age-related Status Traits; their enemies' 
fancy titles or reputation does not impress them.

\subsubsection{Status in Challenges}
When defending in a Social Challenge against another kindred, one's Status total (positive 
minus negative) may be added bonus traits for the purpose of tie resolution, and any 
Negative Status may be called just like Negative Social traits.  Status may also be 
added to the aggressor's total if their presence is known (e.g. not hiding in Obfuscate).  
Simple ignorance of someone's station or identity is no defense against its effects.

Status typically only applies to Contested challenges, and never against mortals.  Status 
may apply against ghouls, at Storyteller discretion.

\subsection{Age-Related Status}

\begin{center}
{\footnotesize
   \begin{tabular}{| c | c | c | l |}
	\hline
	\textbf{Kindred Age} & \textbf{Generation} & \textbf{Title} & \textbf{Status Traits} \\
	\hline
	0 - 100 & $\leq$ 11$^{th}$ & Neonate & \textit{(None)} \\
	100 - 250 & 11$^{th}$ - 9$^{th}$ & Ancilla & Recognized \\
	250 - 450 & 9$^{th}$ - 8$^{th}$ & Venerate & Potent, Venerated \\
	450+ & 8$^{th}$ - 7$^{th}$ & Elder & \footnotesize{Potent, Venerated, Established} \normalsize \\
	\hline
	\multicolumn{4}{| c |}{{\footnotesize All age-related titles require Storyteller approval }} \\ 
	\hline
   \end{tabular}
}
\end{center}

\noindent \textbf{Neonates} are easily the most visible and numerous members of kindred society, 
across all Sects.  Usually less than one hundred years deceased, these fledgling vampires are the 
youngest and most inexperienced predators, still learning the ins and outs of undead life.  Often 
looked down upon by their elders for their immaturity and lack of finesse when dealing with kindred 
politics, the harsh and top-down structure of the Camarilla guarantees that any Neonate wishing to 
get ahead will have to be truly exceptional.  Usually these kindred think in human time-frames and 
as such their plans rarely extend longer than a year or two, though their first-hand connection to the 
modern nights often makes their advice or ability to adapt to new situations valuable to more aged or 
static kindred.

\textbf{Ancillae} are those kindred who, by benefit of their age and strength of blood, have garnered 
fame and respect, at least enough to begin climbing the social ladder of kindred society.  Most often 
between 125 and 250 years dead, these kindred have shown a deep understanding of kindred politics as 
well as the uncanny ability to outwit, outplay, and simply outlast other individuals.  If Neonates are 
seen as children, an Ancilla is a young adult, able to make their own decisions and largely able to 
comport himself without making a mockery of his Clan or lineage.  In respect for their accomplishments 
and their ability to execute plans in the order of decades, the Camarilla \emph{Recognizes} their value 
and rewards them with Status.

\textbf{Venerates} form the rigid backbone of kindred society.  Between 250 and 450 years old, these 
powerful vampires have proven themselves to be true predators, with the strength of blood to match 
their commitment to survival in an environment when so many do not reach half this old.  Venerate 
kindred use Ancillae and Neonates as their pawns in their great, sometimes unfathomable, machinations 
against their enemies.  Kindred of this age realize that direct conflict and violence against their foes 
will too often risk their own destruction; such are the tactics of younger and weaker minds.  Having 
served the tenets of the Camarilla for so long and so faithfully, their elders make public record of 
their \emph{Potency} and the \emph{Veneration} owed them by other, lesser, kindred.

\textbf{Elders} are those truly rare individuals whose existence predates the Camarilla itself.  
Their plans sublime, the power contained in their blood almost unimaginable, Elders form the truly 
elite among vampire society.  They wield their younger allies with the skill and precision of a 
tactical surgeon, making waves and continences that extend for multiple decades if not longer, organizing 
whole lineages of kindred, who are often ignorant of the fact that they have been manipulated at all, 
to see their will done.  These kindred have their fingers on the pulse of the continental or global 
stage, and little occurs that escapes their notice or foreknowledge.  These kindred, so long-lived and 
their resources almost limitless, are fiercely defensive, knowing better than any other that a single 
moment of weakness or ill-preparation could end everything they have built.  These aged vampires rarely 
appear at gather, greatly preferring to send agents on their behalf, even when dealing with such 
important individuals as Princes.  Remembering the very birth of their respective Sects, Elders are 
rewarded for their \emph{Potency}, the very real awe and \emph{Veneration} all owe them, and the fact 
that their guidance has lead to the \emph{Established} norm in the fabric of kindred culture.

Of course there are older kindred, the mention of which make even the truly old shudder; those titans 
of the world whose very footsteps created and felled whole nations and empires, by whose wisdom the 
very Camarilla itself was founded when tonight's Elders were mere Neonates.  None can say anything 
about their plans, their goals, or even if such human concepts are still applicable to these predators 
who have survived to become the almost literal immortal elite of the world.  What games these Luminary 
and Sacrosanct kindred play is an utter mystery to even the most perceptive of Elders, who fear that 
they themselves are merely pawns in some greater Jyhad.

Age-related titles and Status are not automatically granted merely for existing, or for having a 
certain potency of blood or prestige of lineage; the character must have the public support of a 
number of his betters who accept him among their number before the Camarilla will recognize their 
accomplishment.  To be raised in age category, for example from Neonate to Ancilla, the petitioner 
must have the public and vocal support of 10 Ancillae, or 5 Venerates, or 3 Elders, or a like 
combination thereof.  

If on the other hand someone has proven themselves truly unworthy of the respect owed someone of their 
age, a like number of kindred organized in decrying the individual will cause him to drop in age 
category, and likely cause much larger scandal to be placed upon him for such visible missteps.

\subsection{Patronage Status}

Patronage Status traits are those given to a character by other, more respected, kindred, and may be 
either positive or negative.  They represent either accolade or punishment, and the weight of the 
person who gave the status can weigh heavily on those seeking to revoke it.  Traditionally any simple 
adjective may be used, but it is common practice to not grant someone Status that could otherwise be 
granted by a Camarilla position. The rules governing Patronage Status in the Camarilla are as follows: 

\begin{itemize}
   \item Venerate and younger kindred may have up to 3 positive Patronage Status
   \item Traditionally non-Camarilla Clans may only have 2 positive Patronage Status traits.
   \item All kindred may receive up to 2 Negative Patronage Status traits.
   \item Caitiff automatically receive an additional Negative Patronage Status trait ``\emph{Caitiff}'' which can never be removed and does not count against the normal cap of 2 Negative Status traits.
   \item Lasombra automatically receive the additional positive Patronage Status trait ``\emph{Respected}'' which can never be removed and does not count against their positive Status trait cap.
   \item Any character may grant another kindred a Patronage Status trait (either positive or negative) if their total status is more than twice the total of the recipient. This requires the permanent expenditure of a positive Patronage Status trait by the granter. 
   \item Ignoring another's Status for one challenge or outburst requires the permanent expenditure of a positive Patronage Status trait; status is too important to be easily ignored.  If a character has no 
   positive Patronage Status to burn, they must publicly proclaim how \emph{Rude} they are to do so.
   \item To remove a Negative Patronage Status trait from yourself you may either permanently expend a positive Patronage Status trait to cancel it out or convince someone else to erase it, following the usual rules for granting Status.
   \item Revoking a Patronage Status trait you yourself have granted costs nothing, whether positive or 
   negative, though any expenditures made to grant the initial status are not returned.
   \item A character in possession of Negative Status may not receive further positive Patronage until such has been removed, either by burning a positive Patronage or by the powers of another.
   \item Patronage status must be unique; one cannot be \emph{Dignified x2}, for example.
\end{itemize}

\noindent While any positive adjective that is not already provided by a given Officer Position (see below) is generally 
acceptable for Patronage Status, over the years there are some labels which have become common or traditional 
to bestow on deserving kindred.  This list is far from exhaustive and should be used as a guide when giving out 
positive Status: \\

\begin{center}
\begin{minipage}{0.4\textwidth}
	\begin{flushleft}
		\begin{itemize}
			\item Admired
			\item Adored
			\item Brave
			\item Cherished
			\item Courageous
			\item Courteous
			\item Favored
		\end{itemize}
	\end{flushleft}
\end{minipage}
\begin{minipage}{0.4\textwidth}
	\begin{flushright}
		\begin{itemize}
			\item Insightful
			\item Praised
			\item Resourceful
			\item Respected
			\item Trustworthy
			\item Well-Connected
			\item Wise
		\end{itemize}
	\end{flushright}
\end{minipage}
\end{center}

\subsection{Positional Status}

Each Camarilla position grants Positional Status, which is lost when the position 
is vacated. All positions also confer the ability to grant or strip Patronage Status 
to an individual as detailed below.  All primary positions confer ``permanent'' Positional 
Status traits and a single ``removable'' Positional Status trait; assistant positions only 
confer a single ``removable'' Positional Status trait.  Only Positional Status may be used 
to affect Status per the powers of a given position.

Permanent Positional Status may never be lost while a character holds a given position, 
while removable Positional Status may be stripped by the individual(s) who appoint the 
position as a means of punishment; a Prince picks the Sheriff, an entire Clan chooses 
their Primogen for example.

In the following list the underlined Status trait is removable. Positional Status ``spent'' to 
affect another's are not lost, and so count for standing and Social Challenges, but may not be used 
again to affect status in the same month. \\

\begin{itemize}
   \item \textbf{Prince} -- \emph{Well-Known, Famous, Exaulted,} and \underline{Loyal}. May grant Positive to or remove Negative Status from any individual within her Domain.
   \item \textbf{Seneschal} -- \emph{Cherished, Esteemed}, and \underline{Influential}.  Speaks with the Prince's voice and ability when the Prince is unavailable. 
   \item \textbf{Sheriff} -- \emph{Feared} and \underline{Trusted}.  May strip Positive Status from or grant Negative Status to those who refuse official questioning or judgment.
   \item \textbf{Keeper of Elysium} -- \emph{Honorable} and \underline{Just}. May strip Positive Status from or grant Negative Status to those who threaten or breach the Masquerade or Elysium.
   \item \textbf{Scourge} -- \emph{Feared} and \underline{Trusted}.  May strip Positive Status from or grant Negative Status to those who harbor willfully unacknowledged kindred.
   \item \textbf{Primogen} -- \emph{Revered} and \underline{Dutiful}.  May grant Positive Status to or remove Negative Status from their Clan.
   \item \textbf{Lead Harpy} -- \emph{Influential} and \underline{Well-Known}.  May strip Positive Status from or grant Negative Status to any member of the Domain who has been involved in a publicized scandal.
   \item \textbf{Assistants} -- Guardian, Huntsmen, Harpies, and Deputies receive \underline{Trusted}. Clan Whips receive \underline{Dutiful}, and all possess the powers of their senior officers, when invoked on their behalf.
\end{itemize}

\subsubsection{Limits on Officers' Abilities to Affect Status}

\begin{itemize}
   \item No officer may affect the Patronage Status of kindred who are of equal or greater 
   standing, or affect the same individual's standing more than once a week.
   \item No officer may affect the standing of others more times per story than the 
   number of Status traits their office provides (e.g. a Primogen twice).
   \item For determining who the Lead Harpy can affect or how often they may levy Status 
   per month, add the number of sitting Primogen to their Positional Status total.  This extra 
   Status does not count in Challenges or in any other situation; it is solely used to determine 
   who the Harpy can affect.
   \item Assistant positions wield the abilities of their bosses when acting as such.  For 
   example in absence of the Primogen, a Whip may grant Status to a member of their Clan, using 
   the Primogen's standing in place of their own. Any primary officer may reverse a Status adjustment 
   levied by an assistant in their name without cost, at any time.
   \item If an officer levies Status upon someone else and then is removed from position, they may 
   only freely remove that Status if they continue to have more standing than the recipient.
   \item ``Acknowledged'' is a unique Positional Status that comes from being a member of the 
   Camarilla and accepted into a Domain, and is granted (or revoked) by the Prince at no cost to himself.
\end{itemize}

\subsection{Examples of Status Totals}
\begin{description}
	\item[Venerate Prince:]  Maximumn of 11 Status -- \\ 
	4 from position, 3 unique patronage, 3 age, Acknowledged
	\item[Ancilla Primogen:]  Maximum of 7 Status --  \\ 
	2 from position, 3 unique patronage, 1 age, Acknowledged
	\item[Neonate Caitiff Huntsman:]  Maximum of 3 effective Status -- \\ 
	1 from position, 2 unique patronage, Acknowledged, -1 for the permanent Negative trait \emph{Caitiff}
\end{description}

